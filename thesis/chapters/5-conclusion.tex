\chapter{Conclusion}\label{sec:conclusion}

We refined and formalized several correctness properties of LYG and successfully proved them in Lean.
However, we parametrized these correctness properties over a correct function $\mathcal{G}$ that semi-decides emptiness of refinement types.
While LYG defines such a function, we did not prove that it indeed is such a correct function $\mathcal{G}$.

Even though these correctness properties look seemingly easy to prove, it turned out to be a very involved undertaking.
After all, it took us 48 definitions and 143 lemmas and theorems to formalize these proofs in Lean!

We believe that this complexity is caused by the amount of details required to describe LYG and the rigorousness of Lean.
In fact, we discovered a minor flaw in LYG's definition of $\mathcal{U}$, buried in the details of the let binding semantics.
Luckily, this flaw has no impact on the GHC implementation of LYG, as the implementation uses a different encoding of refinement types.
Still, this flaw was not discovered in peer reviews of the LYG paper,
showing that LYG's correctness is not obvious at all and making a strong point for verification, yet formal verification.

As our proofs are formally verified by Lean, it is highly unlikely that LYG as formalized by this thesis has any other flaws,
except in the definition of the presented function $\mathcal{G}$ that we did not check.
Nonetheless, our formally verified proofs alone cannot guarantee that our formalization correctly reflects LYG and that our chosen abstractions are general enough, as this is subject to human interpretation. We hope that our informal clarifications support our claim that they do.\\

We can strongly recommend to use Lean for formal verification and suggest to formally verify the correctness of $\mathcal{G}$ too!