\chapter{Formal Definitions}\label{sec:formalization}

Before any property of LYG can be proven or even stated in Lean, all relevant definitions must be formalized.
Since nothing can be left vague in Lean, a lof of decisions had to be made to back up LYG by a fully defined model. 
This chapter discusses these decisions.

\section{Abstracting LYG: The Guard Module}
LYG does not specify an exact guard or expression syntax.
Instead, the notation ``$...$'' is often used to indicate a sensible continuation to make guards powerful enough to model all Haskell constructs.
This is rather problematic for a precise formalization and presented the first big challenge of this thesis.
As we wanted to avoid formalizing Haskell and its semantics, we had to carefully design an abstraction that is as close as possible to LYG
while pinning down guards to a closed but extendable theory.

\subsubsection{The Result Monad}

First, we defined a generic $\mathtt{Result}$ monad to capture the result of an evaluation.
Due to laziness, evaluation of guard trees can either end with a specific right hand side, not match any guard or diverge:

\begin{minted}{Lean}
inductive Result (α: Type)
| value: α → Result
| diverged: Result
| no_match: Result
\end{minted}

A $\mathtt{bind}$ operation can be easily defined on $\mathtt{Result}$ to make it a proper monad with $\mathtt{Result.value}$ as unit function:

\begin{minted}{Lean}
def Result.bind { α β: Type } (f: α → Result β): Result α → Result β
| (Result.value val) := f val
| Result.diverged := Result.diverged
| Result.no_match := Result.no_match
\end{minted}

\subsubsection{Denotational Semantics for Guards}

For some abstract environment type $\mathtt{Env}$, we would like to have a denotational semantics $\mathtt{Grd.eval}$ for guards $\mathtt{Grd}$:
\begin{minted}{Lean}
Grd.eval : Grd → Env → Result Env
\end{minted}

Abstracting $\mathtt{Grd.eval}$ would unify all guard constructs available in Haskell and those used by LYG.
However, LYG needs to recognize all guards that can lead to a diverged evaluation:
Removing all RHSs behind such a guard would inevitably remove the guard itself.
As this might change the semantics of the guard tree, LYG cannot mark all such RHSs as redundant unless there is a proof that the guard will never diverge.
As a consequence, $\mathtt{Grd.eval}$ cannot be abstracted away.

Instead, we explicitly distinguish between non-diverging (\textit{total}) $\mathtt{tgrd}$s and possibly diverging $\mathtt{bang}$ guards:
\begin{minted}{Lean}
inductive Grd
| tgrd (tgrd: TGrd)
| bang (var: Var)
\end{minted}

While $\mathtt{TGrd}$s classically represent guards and $\mathtt{Grd}$s represent guards with side effects (introduced by $\mathtt{bang}$ guards) in this context,
we decided to follow the naming conventions of LYG and chose the name $\mathtt{TGrd}$ for side-effect free (non-diverging) guards
rather than renaming $\mathtt{Grd}$.

In order to define a denotational semantics on $\mathtt{Grd}$, we postulated the functions $\mathtt{tgrd\_eval}: \mathtt{TGrd} \to \mathtt{Env} \to \mathtt{option}\;\mathtt{Env}$ and $\mathtt{is\_bottom}: \mathtt{Var} \to \mathtt{Env} \to \mathtt{bool}$ as well as a type $\mathtt{Var}$ that represents variables. While $\mathtt{tgrd}$s can change the environment,
$\mathtt{bang}$ guards cannot:

\begin{minted}{Lean}
def Grd.eval : Grd → Env → Result Env
| (Grd.tgrd grd) env :=
    match tgrd_eval grd env with
    | none := Result.no_match
    | some env' := Result.value env'
    end
| (Grd.bang var) env :=
    if is_bottom var env
    then Result.diverged
    else Result.value env
\end{minted}

Alternatively,
we could have set $\mathtt{Var} := \mathtt{Env} → \mathtt{bool}$ and $\mathtt{TGrd} := \mathtt{Env} → \mathtt{option}\;\mathtt{Env}$ and replace $\mathtt{is\_bottom}$ and $\mathtt{tgrd\_eval}$ with $\mathtt{id}$,
yielding the following definition:
\begin{minted}{Lean}
inductive Grd'
| tgrd (grd: Env → option Env)
| bang (test: Env → bool)
\end{minted}
However, this could make the set of guard trees and refinement types uncountable.
While this is not problematic for aspects explored by this thesis,
it could make implementing a correct function $\mathcal{G}$ impossible, as it cannot reason anymore about guards in a computable way if $\mathtt{Env}$ is instantiated with a non-finite type.

\bigskip

\subsubsection{The Guard Module}

In Lean, type classes provide an ideal mechanism to define such ambient abstractions.
They can be opened, so that all members of the type class become implicitly available in all definitions and theorems.
Every implicit usage pulls the type class into its signature so that consumers can provide a concrete implementation of the type class.

We defined and opened a type class \textit{GuardModule} that describes the presented abstraction:

\begin{minted}{Lean}

class GuardModule :=
    (Rhs : Type)
    [rhs_decidable: decidable_eq Rhs]
    (Env : Type)
    (TGrd : Type)
    (tgrd_eval : TGrd → Env → option Env)
    (Var : Type)
    (is_bottom : Var → Env → bool)

variable [GuardModule]
open GuardModule
\end{minted}

We also postulated a type $\mathtt{Rhs}$ to refer to right hand sides. For technical reasons, equality on this type must be decidable.
This abstracts from the numbers that are used in LYG to distinguish right hand sides.
We also require most types to be inhabited so that we can construct module-independent examples.

All following definitions and theorems implicitly make use of this abstraction.

\section{Guard Trees}\label{chap:formalGuardTrees}

\subsubsection{Syntax of Guard Trees}

With the definition of $\mathtt{Grd}$, guard trees are defined as inductive data type:

\begin{minted}{Lean}
inductive Gdt
| rhs (rhs: Rhs)
| branch (tr1: Gdt) (tr2: Gdt)
| grd (grd: Grd) (tr: Gdt)
\end{minted}

\subsubsection{Semantics of Guard Trees}

$\mathtt{Gdt.eval}$ defines a denotational semantics on guard trees, using the semantics of guards.
It returns the first RHS that matches a given environment. If a guard diverges, the entire evaluation diverges. Otherwise, if no RHSs matches, \textit{no-match} is returned.

\begin{minted}{Lean}
def Gdt.eval : Gdt → Env → Result Rhs
| (Gdt.rhs rhs) env := Result.value rhs
| (Gdt.branch tr1 tr2) env :=
    match tr1.eval env with
    | Result.no_match := tr2.eval env
    | r := r
    end
| (Gdt.grd grd tr) env := (grd.eval env).bind tr.eval
\end{minted}

\subsubsection{RHSs in Guard Trees}

Every guard tree contains a (non-empty) finite set of right hand sides:
\begin{minted}{Lean}
def Gdt.rhss: Gdt → finset Rhs
| (Gdt.rhs rhs) := { rhs }
| (Gdt.branch tr1 tr2) := tr1.rhss ∪ tr2.rhss
| (Gdt.grd grd tr) := tr.rhss
\end{minted}

In LYG, it is implicitly assumed that the right hand sides of a guard tree are numbered unambiguously.
This has to be stated explicitly in Lean with the following recursive predicate:

\begin{minted}{Lean}
def Gdt.disjoint_rhss: Gdt → Prop
| (Gdt.rhs rhs) := true
| (Gdt.branch tr1 tr2) :=
        disjoint tr1.rhss tr2.rhss
        ∧ tr1.disjoint_rhss ∧ tr2.disjoint_rhss
| (Gdt.grd grd tr) := tr.disjoint_rhss
\end{minted}

\subsubsection{Removing RHSs in Guard Trees}

$\mathtt{Gdt.remove\_rhss}$ defines how a set of RHSs can be removed from a guard tree.
This definition is required to state that all redundant RHSs can be removed without changing semantics.
Note that the resulting guard tree might be empty when all RHSs are removed!

\begin{minted}{Lean}
def Gdt.branch_option : option Gdt → option Gdt → option Gdt
| (some tr1) (some tr2) := some (Gdt.branch tr1 tr2)
| (some tr1) none := some tr1
| none (some tr2) := some tr2
| none none := none

def Gdt.grd_option : Grd → option Gdt → option Gdt
| grd (some tr) := some (Gdt.grd grd tr)
| _ none := none

def Gdt.remove_rhss : finset Rhs → Gdt → option Gdt
| rhss (Gdt.rhs rhs) := if rhs ∈ rhss then none else some (Gdt.rhs rhs)
| rhss (Gdt.branch tr1 tr2) :=
    Gdt.branch_option
        (tr1.remove_rhss rhss)
        (tr2.remove_rhss rhss)
| rhss (Gdt.grd grd tr) := Gdt.grd_option grd (tr.remove_rhss rhss)
\end{minted}

Finally, to deal with the semantics of empty guard trees,
$\mathtt{Gdt.eval\_option}$ lifts $\mathtt{Gdt.eval}$ to $\mathtt{option\;Gdt}$:

\begin{minted}{Lean}
def Gdt.eval_option : option Gdt → Env → Result
| (some gdt) env := gdt.eval env
| none env := Result.no_match
\end{minted}

\section{Refinement Types}
\label{sec:formalizationRefinementTypes}

Refinement types presented another challenge.
Defining refinement types through a proper type system would have required to model Haskell types.
Instead, we tried to rely on the same abstractions used to define guard trees in hope that guard trees and refinement types can be related.

In this formalization, a refinement type $\Phi$ denotes a predicate on environments:

\begin{minted}{Lean}
def Φ.eval: Φ → Env → bool
\end{minted}

With a proper $\mathtt{GuardModule}$ instantiation, the environment can be used to not only carry runtime values, but also their type!
A (well) typed environment can assist in proving a refinement type to be empty.

\subsubsection{Variable Scoping Rules / Incorrectness of $\mathcal{U}$}
\label{sec:formalizationVariableBindingRules}

Another problem that had to be solved was the formalization of the unconventional binding mechanism of refinement types through conjunctions, as described in chapter \ref{chap:bckgrndRefinementTypesBinding}.
In particular, this causes $\mathcal{U}$ to be not correct (for some intuitive notion of correctness)
with regards to the guard tree semantics we defined in chapter \ref{chap:formalGuardTrees}.
While the following guard tree $gdt$ does not match for $x = \mathtt{False}$, its uncovered refinement type $\Theta$ computed by $\mathcal{U}$ is empty due to the constraints $x \ntermeq \Conid{False}$ and $x \ntermeq \Conid{True}$ that refer to the same variable and thus represent a contradiction!

\[
    \mathit{gdt} :=
    \vcenter{\hbox{
    \begin{forest}
    	grdtree
    	[
    		[{$\grdlet{x}{\mathtt{True}},\, \grdcon{\mathtt{False}}{x}$} [1]]
    		[{$\grdcon{\mathtt{True}}{x}$} [2]]
    	]
    \end{forest}
    }}
\]
\begin{align*}
    \Theta := \mathcal{U}(\true, \mathit{gdt}) =& \reft{ x{:}\ensuremath{\Conid{Bool}} }{
        ((\grdlet{x}{\Conid{True}} \andtheta
        x \ntermeq \Conid{False}) \andtheta x \ntermeq \Conid{True}
    } \\
    =& \reft{ x{:}\ensuremath{\Conid{Bool}} }{
        \grdlet{x}{\Conid{True}} \land
        (x \ntermeq \Conid{False} \land x \ntermeq \Conid{True})
    }
\end{align*}

In the example, the let binding brings a variable $x$ into scope that shadows an outer variable.
Due to the definition of $\mathcal{U}$ and the scoping rules of refinement types,
this shadowing binding of $x$ overrides the outer variable $x$ in contexts where it is incorrect to do so.

In particular, the term $\unc(\unc(\Theta, t_1), t_2)$ in the branch case of $\mathcal{U}$ is problematic:
All bindings introduced in $\Theta$ should still be exposed by $\unc(\Theta, t_1)$ so that variables in $t_2$ are resolved correctly.
However, variable definitions introduced in $t_1$ must not be visible to $t_2$ and thus must not be exposed by $\unc(\Theta, t_1)$!
This is clearly violated by the term $\unc(\Theta \andtheta (\ctlet{x}{e}), t)$ that defines the let-case in $\mathcal{U}$ - it exposes both all bindings from $\Theta$ and the new binding $\ctlet{x}{e}$.

Shadowing is unproblematic for the presented semantics of guard trees though: If the first guard tree of a branch fails to match, its environment just before the failing guard is discarded and with it possible shadowing bindings.
The second branch is always evaluated with the same environment that the first guard tree has been evaluated with.
This is consistent with Haskells semantics of pattern match expressions.

There are several ways of how this problem can be addressed.
\begin{itemize}
    \item Replace the term $\unc(\unc(\Theta, t_1), t_2)$ in the definition of $\mathcal{U}$ with $\unc((\unc(\Theta, t_1) \cup \false) \andtheta \Theta, t_2)$.
    
        ``$\andtheta$'' stops at ``$\cup$'', so the $\cup$-operator acts as scope boundary.
        To bring the variables defined by $\Theta$ into scope again, $\Theta$ is joined a second time, potentially causing a refinement type of exponential size.
        While we believe that $\mathcal{U}$ is correct with this updated definition,
        we decided against this solution as the construction to limit the scope feels like a band aid and is unnecessarily complex.
    
    \item Adjust the semantics of guard trees so that variables defined in a branch override shadowed variables in all later branches.
    
        We prove correctness of $\mathcal{U}$ as stated in LYG with this updated semantics of guard trees.
        However, this semantics is not only very unconventional, but also dramatically increases the complexity
        when reasoning about the effect of removing an inaccessible RHS.
        
        Since variable bindings introduced by guards that guard only inaccessible RHSs stay visible
        until the evaluation ends (and are not only relevant for the inaccessible RHSs),
        removing such guards almost always causes a different final environment.
        In this sense, almost no inaccessible RHSs are redundant - which is not the intention of the analysis and clearly not the case for the GHC implementation of LYG. To make the analysis more meaningful, we could require each variable name to be unique.
        With this assumption, such environment modification should have no impact on the evaluation result.
        Due to the high complexity of this approach, we decided against it too.
    
    \item Limit the scope of variables in refinement types.
    
        A data constructor $\mathtt{Φ.tgrd\_in}:
        \mathtt{TGrd} \to Φ \to Φ$ is introduced that limits the scope of the guard to the nested refinement type and any scoping behavior of the $\land$-operator is removed.
        This simplifies the scoping mechanism, but requires to adapt $\mathcal{U}$, as done in chapter \ref{sec:formalizationUncoveredAnalysis}.
        We chose this approach due to its clear modeling, in hope to reduce the complexity of the formal proofs.
        
\end{itemize}

This problem does not arise in the GHC implementation of LYG as it uses a different encoding for refinement types.

\subsubsection{Syntax of Refinement Types}

Finally, this is our formalized syntax of refinement types:

\begin{minted}{Lean}
inductive Φ
| false
| true
| tgrd_in (tgrd: TGrd) (ty: Φ)
| not_tgrd (tgrd: TGrd)
| var_is_bottom (var: Var)
| var_is_not_bottom (var: Var)
| or (ty1: Φ) (ty2: Φ)
| and (ty1: Φ) (ty2: Φ)
\end{minted}

Since the negation of a guard cannot bind variables,
it does not need to have a nested refinement type that would see bound variables.
The same applies to $\mathtt{var\_is\_bottom}$ and its negation.

\subsubsection{Semantics of Refinement Types}

The semantics of refinement types is easily defined and implicitly uses the guard module:

\begin{minted}{Lean}
def Φ.eval: Φ → Env → bool
| Φ.false env := ff
| Φ.true env := tt
| (Φ.tgrd_in grd ty) env := match tgrd_eval grd env with
    | some env := ty.eval env
    | none := ff
    end
| (Φ.not_tgrd grd) env :=
    match tgrd_eval grd env with
    | some env := ff
    | none := tt
    end
| (Φ.var_is_bottom var) env := is_bottom var env
| (Φ.var_is_not_bottom var) env := !is_bottom var env
| (Φ.or t1 t2) env := t1.eval env || t2.eval env
| (Φ.and t1 t2) env := t1.eval env && t2.eval env
\end{minted}

With this definition the evaluation of the second operand of a conjunction is obviously independent
of any environment effects applied in the evaluation of the first operand!

\subsubsection{Definition of \texttt{is\_empty}}

A refinement type $\Phi$ is called \textit{empty} if it does not match any environment.
This is formalized by the predicate $\Phi\mathtt{.is\_empty}$:

\begin{minted}{Lean}
def Φ.is_empty (ty: Φ): Prop := ∀ env: Env, ¬(ty.eval env)
\end{minted}

\subsubsection{Definition of \texttt{can\_prove\_empty}}

Instead of a partial function $\generate$ with $\generate(Φ) = \varnothing$ if and only if $Φ$ is empty,
we define a total function $\mathtt{can\_prove\_empty}$ and a predicate $\mathtt{correct\_can\_prove\_empty}$ that ensures
its correctness. This abstracts from the in this context unneeded generation of inhabitants. It also avoids dealing with partial functions, which are not supported by Lean.
\begin{minted}{Lean}
variable can_prove_empty: Φ → bool
def correct_can_prove_empty : Prop :=
    ∀ ty: Φ, can_prove_empty ty = tt → ty.is_empty
\end{minted}

The subtype $\mathtt{CorrectCanProveEmpty}$ bundles a correct $\mathtt{can\_prove\_empty}$ function:
\begin{minted}{Lean}
def CorrectCanProveEmpty := {
    can_prove_empty : Φ → bool
    // correct_can_prove_empty can_prove_empty
}
\end{minted}

\section{Uncovered Analysis}
\label{sec:formalizationUncoveredAnalysis}

As discussed in chapter \ref{sec:formalizationRefinementTypes}, LYG's definition of
$\mathcal{U}$ has problems with guard trees that define shadowing bindings.
LYG defined $\mathcal{U}$ as follows (see chapter \ref{chap:bckgrdUncoveredAnalysis} for the discussion of this definition):

\[
\begin{array}{lcl}
	\unc(\reft{\Gamma}{\Phi}, \gdtrhs{n})                                    & = & \reft{\Gamma}{\false}                                                                                                 \\
	\unc(\Theta, \gdtseq{t_1}{t_2})                                          & = & \unc(\unc(\Theta, t_1), t_2)                                                                                          \\
	\unc(\Theta, \gdtguard{\grdbang{x}}{t})                                  & = & \unc(\Theta \andtheta (x \ntermeq \bot), t)                                                                           \\
	\unc(\Theta, \gdtguard{\grdlet{x}{e}}{t})                                & = & \unc(\Theta \andtheta (\ctlet{x}{e}), t)                                                                              \\
	\unc(\Theta, \gdtguard{\grdcon{\genconapp{K}{a}{\gamma}{y:\tau}}{x}}{t}) & = & \Theta \andtheta (x \ntermeq K) \uniontheta \unc(\Theta \andtheta (\ctcon{\genconapp{K}{a}{\gamma}{y:\tau}}{x}), t) \\
\end{array}
\]

Equipped with the data constructor $\mathtt{Φ.tgrd\_in}$ we can fix the shadowing problem problem and formalize $\mathcal{U}$ now.
Instead of using $\Phi$ as accumulator type, our formalization uses the type $\Phi \to \Phi$:
The new accumulator explicitly applies a context to a refinement type.
This happens implicitly in LYG's definition through the use of $\Theta \andtheta \cdot$.

Note that all occuring accumulator functions are homomorphic under the semantics of refinement types, i.e. $\mathtt{Gdt.eval}\;(f\;(a.\mathit{and}\;b)) = \mathtt{Gdt.eval}\;((f\;a).\mathit{and}\;(f\;b))$.
We carefully make use of this to get formalized definitions of $\mathcal{U}$ and $\mathcal{A}$ that can be interleaved, as done in LYG.

\begin{minted}{Lean}
def ω_acc : (Φ → Φ) → Gdt → Φ
| acc (Gdt.rhs _) := Φ.false
| acc (Gdt.branch tr1 tr2) := (ω_acc ((ω_acc acc tr1).and ∘ acc) tr2)
| acc (Gdt.grd (Grd.bang var) tr) :=
    ω_acc (acc ∘ (Φ.var_is_not_bottom var).and) tr
| acc (Gdt.grd (Grd.tgrd grd) tr) :=
            (acc (Φ.not_tgrd grd))
        .or (ω_acc (acc ∘ (Φ.tgrd_in grd)) tr)

def ω : Gdt → Φ := ω_acc id
\end{minted}

\section{Redundant / Inaccessible Analysis}

\subsubsection{Formalization of Annotated Trees}

The formalization of annotated trees is straightforward.
However, we allow arbitrary annotations rather than only accepting refinement types.
This will become useful in formal proofs when we no longer care about
the specific refinement types but only if they are empty.

\begin{minted}{Lean}
inductive Ant (α: Type)
| rhs (a: α) (rhs: Rhs): Ant
| branch (tr1: Ant) (tr2: Ant): Ant
| diverge (a: α) (tr: Ant): Ant
\end{minted}

\subsubsection{Formalization of $\mathcal{A}$}

Similar to the formalization of $\mathcal{U}$ in chapter \ref{sec:formalizationUncoveredAnalysis}, we also need to address the shadowing problem when formalizing $\mathcal{A}$.
This is LYG's definition of $\mathcal{A}$ as stated in chapter \ref{chap:backgrdRedundantInaccAnalysis}:

\[
	\begin{array}{lcl}
		\ann(\Theta,\gdtrhs{n})                                                  & = & \antrhs{\Theta}{n}                                                                       \\
		\ann(\Theta, \gdtseq{t_1}{t_2})                                          & = & \antseq{\ann(\Theta, t_1)}{\ann(\unc(\Theta, t_1), t_2)}                                 \\
		\ann(\Theta, \gdtguard{\grdbang{x}}{t})                                  & = & \antbang{\Theta \andtheta (x \termeq \bot)}{\ann(\Theta \andtheta (x \ntermeq \bot), t)} \\
		\ann(\Theta, \gdtguard{\grdlet{x}{e}}{t})                                & = & \ann(\Theta \andtheta (\ctlet{x}{e}), t)                                                 \\
		\ann(\Theta, \gdtguard{\grdcon{\genconapp{K}{a}{\gamma}{y:\tau}}{x}}{t}) & = & \ann(\Theta \andtheta (\ctcon{\genconapp{K}{a}{\gamma}{y:\tau}}{x}), t)                  \\
	\end{array}
\]

Our formalization in Lean follows. Analogous to our formalization of $\mathcal{U}$, instead of contextualizing refinement types by
combining them with the accumulator through $\andtheta$, we model the accumulator as an explicit function
that contextualizes its argument:

\begin{minted}{Lean}
def π_acc : (Φ → Φ) → Gdt → Ant Φ
| acc (Gdt.rhs rhs) := Ant.rhs (acc Φ.true) rhs
| acc (Gdt.branch tr1 tr2) :=
    Ant.branch
        (π_acc acc tr1)
        (π_acc ((ω_acc acc tr1).and ∘ acc) tr2)
| acc (Gdt.grd (Grd.bang var) tr) :=
    Ant.diverge
        (acc (Φ.var_is_bottom var)) 
        (π_acc (acc ∘ ((Φ.var_is_not_bottom var).and)) tr)
| acc (Gdt.grd (Grd.tgrd grd) tr) :=
    (π_acc (acc ∘ (Φ.tgrd_in grd)) tr)

def π : Gdt → Ant Φ := π_acc id
\end{minted}

Note that in the branch case, $\mathcal{A}\mathtt{\_acc}$ and $\mathcal{U}\mathtt{\_acc}$
are called with the same arguments. Even more, both functions have the same recursion structure, which
makes it possible to interleave both functions.
This is done in chapter \ref{sec:formalizationInterleaving}.


%-- returns (accessible, inaccessible, redundant) rhss, given that `can_prove_empty` is correct.

\subsubsection{Formalization of $\mathcal{R}$}

It remains to formalize the function $\mathcal{R}$ that partitions
all right hand sides of an annotated guard tree into accessible, inaccessible and redundant right hand sides,
by using the function $\mathtt{can\_prove\_empty}$.

This is $\mathcal{R}$ as presented in LYG and chapter \ref{chap:backgrdRedundantInaccAnalysis}:
\[
	\begin{array}{lcl}
		\red(\antrhs{\Theta}{n})  & = & \begin{cases}
			(\epsilon, \epsilon, n), & \text{if $\generate(\Theta) = \emptyset$} \\
			(n, \epsilon, \epsilon), & \text{otherwise}                          \\
		\end{cases}                                                                                                                     \\
		\red(\antseq{t}{u})       & = & (\overline{k}\,\overline{k'}, \overline{n}\,\overline{n'}, \overline{m}\,\overline{m'}) \hspace{0.5em} \text{where} \begin{array}{l@{\,}c@{\,}l}
			(\overline{k}, \overline{n}, \overline{m})    & = & \red(t) \\
			(\overline{k'}, \overline{n'}, \overline{m'}) & = & \red(u) \\
		\end{array} \\
		\red(\antbang{\Theta}{t}) & = & \begin{cases}
			(\epsilon, m, \overline{m'}), & \text{if $\generate(\Theta) \not= \emptyset$ and $\red(t) = (\epsilon, \epsilon, m\,\overline{m'})$} \\
			\red(t),                      & \text{otherwise}                                                                                     \\
		\end{cases}                                                                                                                     \\
	\end{array}
\]

This definition has a surprisingly direct representation in Lean:
\begin{minted}{Lean}
def ρ : Ant Φ → list Rhs × list Rhs × list Rhs
| (Ant.rhs ty n) :=
    if can_prove_empty ty
    then ([], [], [n])
    else ([n], [], [])
| (Ant.branch tr1 tr2) :=
    match (ρ tr1, ρ tr2) with
    | ((k, n, m), (k', n', m')) := (k ++ k', n ++ n', m ++ m')
    end 
| (Ant.diverge ty tr) :=
    match ρ tr, can_prove_empty ty with
    | ([], [], m :: ms), ff := ([], [m], ms)
    | r, _ := r
    end
\end{minted}

\section{Interleaving $\mathcal{U}$ and $\mathcal{A}$}
\label{sec:formalizationInterleaving}

Since $\mathcal{A}\mathtt{\_acc}$ and $\mathcal{U}\mathtt{\_acc}$ have
the same recursion structure, they can be combined into a single function that shares the recursive invocations.
The following function $\mathcal{UA}\mathtt{\_acc}$ computes the uncovered refinement type and the annotated guard tree for a given guard tree at the same time.
This improves performance if a lazy evaluation strategy is used in combination with sharing as the accumulator can be fully shared.

\begin{minted}{Lean}
def ωπ_acc : (Φ → Φ) → Gdt → Φ × Ant Φ
| acc (Gdt.rhs rhs) := (Φ.false, Ant.rhs (acc Φ.true) rhs)
| acc (Gdt.branch tr1 tr2) :=
    let (U1, A1) := ωπ_acc acc tr1,
        (U2, A2) := ωπ_acc (U1.and ∘ acc) tr2
    in  (U2, Ant.branch A1 A2)
| acc (Gdt.grd (Grd.bang var) tr) :=
    let (U, A) := ωπ_acc (acc ∘ (Φ.var_is_not_bottom var).and) tr
    in (U, Ant.diverge (acc (Φ.var_is_bottom var)) A)
| acc (Gdt.grd (Grd.tgrd grd) tr) := 
    let (U, A) := ωπ_acc (acc ∘ (Φ.tgrd_in grd)) tr
    in ((acc (Φ.not_tgrd grd)).or U, A)
\end{minted}

It is surprisingly easy to show that this function is really interleaving $\mathcal{A}\mathtt{\_acc}$ and $\mathcal{U}\mathtt{\_acc}$:

\begin{minted}{Lean}
theorem ωπ_acc_eq (acc: Φ → Φ) (gdt: Gdt):
    ωπ_acc acc gdt = (ω_acc acc gdt, π_acc acc gdt) :=
by induction gdt generalizing acc;
    try { cases gdt_grd }; simp [ωπ_acc, ω_acc, π_acc, *]
\end{minted}


\chapter{Correctness Statements}
\label{sec:formalizationCorrectnessStmts}

As we have all the required definitions at this point, we can state and formalize what we expect of the presented pattern match analyses to be considered correct.
We provide proofs for all correctness propositions on GitHub \cite{leanProof}. Chapter \ref{sec:proof} will discuss parts of these proofs in more detail.

\section{Correctness of the Uncovered Analysis}
\label{sec:formalizationSemanticU}

$\unc$ should compute a refinement type that denotes exactly all values that are not covered by a given guard tree.
This does not include values under which the execution diverges!

The following theorem states correctness of $\mathcal{U}$ in Lean:

\begin{minted}{Lean}
theorem ω_semantic: ∀ gdt: Gdt, ∀ env: Env,
        (ω gdt).eval env ↔ (gdt.eval env = Result.no_match)
\end{minted}

As an obvious consequence, a guard tree always matches (or diverges) if and only if the refinement type computed by $\mathcal{U}$ is empty.
If a correct function $\mathcal{G}$ or $\mathtt{can\_prove\_empty}$ proves emptiness of such a computed refinement type, there are no uncovered cases by this theorem. Otherwise, a warning of potential uncovered cases should be issued!

Hence, this theorem implies correctness of the uncovered analysis:
The uncovered analysis should rather report a false positive than not detect an uncovered case.

Note that this theorem carries over to all semantically equivalent definitions of $\mathcal{U}$.

\subsection{Comparison to LYGs Notion Of Correctness}

LYG states that ``[...] LYG will never fail to report uncovered clauses (no false negatives), but it may report false positives'' \cite{10.1145/3408989}.
Our statement of $\mathcal{U}$s correctness is stronger:
The function $\mathcal{U}$ computes a refinement type that covers exactly all environments that are not covered by the guard tree. If $\mathcal{G}$ is assumed to be correct and used to semi-decide whether the refinement type computed by $\mathcal{U}$ is empty,
LYGs claim follows.

\section{Correctness of the Redundant/Inaccessible Analysis}
\label{sec:formalizationSemanticRA}

For a given guard tree and a given correct function $\mathtt{can\_prove\_empty}$ (which corresponds to $\generate$ in LYG), $\red$
should compute a triple $(a, i, r)$ of accessible, inaccessible and redundant right hand sides.
Whenever the given guard tree evaluates to a RHS, this RHS must be accessible and neither inaccessible nor redundant.
RHSs that are redundant can be removed from the guard tree without changing the semantics of the guard tree.
This expresses correctness of the redundant and inaccessible analysis.

\begin{minted}{Lean}
theorem ρ_semantic:
    ∀ can_prove_empty: CorrectCanProveEmpty,
    ∀ gdt: Gdt, gdt.disjoint_rhss → (
        let ⟨ a, i, r ⟩ := ρ can_prove_empty.val (π gdt)
        in
                (∀ env: Env, ∀ rhs: Rhs,
                    gdt.eval env = Result.value rhs
                      → rhs ∈ a \ (i ++ r)
                )
            ∧
                Gdt.eval_option (gdt.remove_rhss r.to_finset)
                = gdt.eval

        : Prop
    )
\end{minted}

Note that redundant RHSs could be marked as inaccessible or even accessible instead without violating this theorem.
The opposite is not true: Not all accessible RHSs can be marked as inaccessible and not all inaccessible RHSs can be marked as redundant - see chapters \ref{sec:intro} and \ref{sec:background} for counterexamples. However, we conjecture that $a$ contains no inaccessible and $i$ no redundant RHSs if $\mathtt{can\_prove\_empty}$ is both correct and complete.

\subsection{Comparison to LYGs Notion Of Correctness}

LYG states correctness of the redundant/inaccessible analysis as following:
``Similarly, LYG will never report accessible clauses as
redundant (no false positives), but it may fail to report clauses which are redundant when the code
involved is too close to undecidable territory.'' \cite{10.1145/3408989}.
Furthermore, LYG also states
``A redundant equation can be removed from a
function without changing its semantics, whereas an inaccessible equation cannot, [...]''.

We both improved the precision of LYGs notion of correctness by formally defining every involved concept,
but also made it more complete by stating that RHSs identified as redundant by LYG are indeed redundant.
