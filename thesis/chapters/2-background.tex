\chapter{Background}\label{sec:background}

\section{Lower Your Guards}

Lower Your Guards (LYG) \cite{10.1145/3408989} describes algorithms that analyze pattern matching expressions and report uncovered cases, but also redundant and inaccessible right hand sides.

LYG was designed for use in the Glasgow Haskell Compiler,
but the algorithm and its data structures are so universal
that they can be leveraged for other programming languages with pattern matching constructs too.

All definitions and some examples of this chapter are taken from LYG \cite{10.1145/3408989}.

\subsection{Inaccessible vs. Redundant RHSs}

A closer look at figure \ref{lst:haskell} reveals that, while both RHS 3 and 4 are inaccessible,
the semantics of $\mathtt{f}$ changes if both are removed.
This means that an automated refactoring cannot just remove all inaccessible leaves!

The reason for this is the term $t := \mathtt{f}\;\mathtt{Case3}\;\mathtt{undefined}$ and the fact that Haskell uses a lazy evaluation strategy.
If both RHSs $3$ and $4$ are removed, $t$ evaluates to $5$ - the term $\mathtt{undefined}$ is never evaluated as no pattern matches against it.
However, if nothing or only one of the RHSs $3$ or $4$ is removed, $\mathtt{undefined}$ will be matched with $\mathtt{True}$ and thus $t$ will throw a runtime error!

To communicate this difference, LYG introduces the concept of $\mathit{redundant}$ and $\mathit{inaccessible}$ RHSs:
A redundant RHS can be removed from its pattern matching expression without any observable difference.
An inaccessible RHS is never evaluated, but its removal might lead to observable changes.
This definition implies that redundant RHSs are inaccessible.

As for listing \ref{lst:haskell}, LYG will mark RHS $3$ as inaccessible and RHS $4$ as redundant.
This choice is somewhat arbitrary, as RHS $3$ could be marked as redundant and RHS $4$ as inaccessible as well, and will be discussed in more detail in chapter \ref{chap:isRedundantSet}. However, for the reasons mentioned above, not both RHSs can be marked as redundant.

\subsection{Guard Trees}

For all analyses, LYG first transforms Haskell specific pattern match expressions to simpler \textit{guard trees}.
This transformation removes a lot of complexity, as many different Haskell constructs can be desugared
to the same guard tree. Guard trees also simplify adapting LYG to other programming languages
and they enable studying LYG mostly independent from Haskell.
Their syntax is defined in figure \ref{fig:guardTrees}. \textit{Con} refers to data constructors, \textit{TyCt} to type constraints.

Guard trees (\textit{Gdt}s) are made of three elements: Uniquely numbered right hand sides, \textit{branches} and \textit{guarded trees}.
Guarded trees refer to Haskell specific guards (\textit{Grd}) that control the execution.
\textit{Let guards} can bind a term to a variable in a new lexical scope.
\textit{Pattern match guards} can destructure a value into variables if the pattern matches or otherwise prevent the
execution from entering the tree behind the guard.
Finally, \textit{bang guards} can stop the entire execution when the value of a variable does not reduce to a head normal form (like $\mathtt{undefined}$).

\begin{figure}[htbp]
	\caption{Definition of Guard Trees}
	\label{fig:guardTrees}
	\centering
	\[ \textbf{Guard Syntax} \]
	\[
		\begin{array}{cc}
			\begin{array}{rlcl}
				k,n,m       \in & \mathbb{N} &           &                                                 \\
				K           \in & \Con       &           &                                                 \\
				x,y,a,b     \in & \Var       &           &                                                 \\
				\tau,\sigma \in & \Type      & \Coloneqq & a \mid ...                                      \\
				e \in           & \Expr      & \Coloneqq & x \mid  \genconapp{K}{\tau}{\gamma}{e} \mid ... \\
			\end{array} &
			\begin{array}{rlcl}
				\gamma \in & \TyCt & \Coloneqq & \tau_1 \typeeq \tau_2 \mid ...               \\
				g \in      & \Grd  & \Coloneqq & \grdlet{x:\tau}{e}                           \\
				           &       & \mid      & \grdcon{\genconapp{K}{a}{\gamma}{y:\tau}}{x} \\
				           &       & \mid      & \grdbang{x}                                  \\
			\end{array}
		\end{array}
	\]

	\[ \textbf{Guard Tree Syntax} \]
	\[
		\begin{array}{rcll}
			t \in \Gdt & \Coloneqq & \gdtrhs{k} \mid \gdtseq{t_1}{t_2} \mid \gdtguard{g}{t} \\
		\end{array}
	\]
\end{figure}

The evaluation of a guard tree selects the first right hand side that execution reaches.
If the execution stops at a bang guard, the evaluation is said to \textit{diverge}, otherwise, if execution falls through, the evaluation ends with a \textit{no-match}.
A formal semantics for guard trees will be defined in chapter \ref{chap:formalGuardTrees}.

The transformation from Haskell pattern matches to guard trees is not of much interest for this thesis and can be found in LYG \cite{10.1145/3408989}.
To preserve semantics, it is important that the transformation inserts bang guards whenever a variable is matched against a data constructor.

Figure \ref{fig:desugaringExample} presents the transformation of figure \ref{lst:haskell} into a guard tree.


\begin{figure}[htbp]
	\caption{Desugaring Example}
	\label{fig:desugaringExample}
	\centering
	\begin{minted}{Haskell}
data Case = Case1 | Case2 | Case3 | Case4

f :: Case -> Bool -> Integer
f Case1 _ = 1
f Case2 _ = 2
f x True | Case1 <- x = 3
         | Case2 <- x = 4
f Case3 _ = 5
\end{minted}

	$\Downarrow$

	\begin{forest}
		grdtree
		[
		[{$\grdbang{x_1},\, \grdcon{\mathtt{Case1}}{x_1}$} [1]]
			[
				[{$\grdbang{x_1},\, \grdcon{\mathtt{Case2}}{x_1}$} [2]]
					[
						[{$\grdlet{x}{x_1},\, \grdbang{x_2},\, \grdcon{\mathtt{True}}{x_2}$}
									[{$\grdbang{x},\, \grdcon{\mathtt{Case1}}{x}$} [3]]
									[{$\grdbang{x},\, \grdcon{\mathtt{Case2}}{x}$} [4]]
							]
							[{$\grdbang{x_1},\, \grdcon{\mathtt{Case3}}{x_1}$} [5]]
					]
			]
		]
	\end{forest}
\end{figure}

\vspace{\baselineskip}

It is usually straightforward to define a transformation from pattern matching expressions to guard trees
that also preserves uncovered cases and inaccessible and redundant RHSs.
This makes guard trees an ideal abstraction for the following analysis steps.

\subsection{Refinement Types}

\textit{Refinement types} \cite{10.1145/113445.113468} describe vectors of values $x_1, ..., x_n$ that satisfy a given predicate $\Phi$.
Their syntax is defined in figure \ref{fig:refinementTypes}.

\begin{figure}[htbp]
	\caption{Definition of Refinement Types}
	\label{fig:refinementTypes}
	\centering
	\[
		\begin{array}{rcll}
			\Gamma  & \Coloneqq & \varnothing \mid \Gamma, x:\tau \mid \Gamma, a                                                                                                  & \text{Context}         \\
			\varphi & \Coloneqq & \true \mid \false \mid \ctcon{\genconapp{K}{a}{\gamma}{y:\tau}}{x} \mid x \ntermeq K \\
			        &           & \mid x \termeq \bot \mid x \ntermeq \bot \mid \ctlet{x}{e} & \text{Literals}        \\
			\Phi    & \Coloneqq & \varphi \mid \Phi \wedge \Phi \mid \Phi \vee \Phi                                                                                               & \text{Formula}         \\
			\Theta  & \Coloneqq & \reft{\Gamma}{\Phi}                                                                                                                             & \text{Refinement type} \\
		\end{array}
	\]
    \[ \textbf{Operations on $\Theta$} \]
    \[
    \begin{array}{lcl}
        \Phi \andtheta \varphi &=&
        \begin{cases}
            \Phi_1 \wedge (\Phi_2 \andtheta \varphi) \; \textrm{if }\Phi = \Phi_1 \wedge \Phi_2 \\
            \Phi \wedge \varphi \; \textrm{otherwise} \\
        \end{cases}\\
        \reft{\Gamma}{\Phi} \andtheta \varphi &=&
            \reft{\Gamma}{\Phi \andtheta \varphi}\\
        \reft{\Gamma}{\Phi_1} \uniontheta \reft{\Gamma}{\Phi_2} &=& \reft{\Gamma}{\Phi_1 \vee \Phi_2} \\
    \end{array}
    \]
\end{figure}

Refinement predicates are built from literals $\phi$ and closed under conjunction and disjunction.
The literal $\true$ refers to ``true'', while $\false$ refers to ``false''. For example:
$$
	\begin{array}{rcl}
		\reft{ x{:}\ensuremath{\Conid{Bool}}}{ \true }                                                                                                           & \text{denotes} & \{ \bot, \ensuremath{\Conid{True}}, \ensuremath{\Conid{False}} \}                       \\
		\reft{ x{:}\ensuremath{\Conid{Bool}}}{ x \ntermeq \bot }                                                                                                 & \text{denotes} & \{ \ensuremath{\Conid{True}}, \ensuremath{\Conid{False}} \}                             \\
		\reft{ x{:}\ensuremath{\Conid{Bool}}}{ x \ntermeq \bot \wedge \ctcon{\ensuremath{\Conid{True}}}{x} }                                                     & \text{denotes} & \{ \ensuremath{\Conid{True}} \}                                                         \\
		\reft{ mx{:}\ensuremath{\Conid{Maybe}\;\Conid{Bool}}}{ mx \ntermeq \bot \wedge \ctcon{\ensuremath{\Conid{Just}\;\Varid{x}}}{mx}  } & \text{denotes} & \ensuremath{\Conid{Just}\;}\{ \ensuremath{\bot}, \ensuremath{\Conid{True}}, \ensuremath{\Conid{False}} \} \\
	\end{array}
$$

\subsection{Binding Mechanism Of Refinement Types}\label{chap:bckgrndRefinementTypesBinding}

Refinement type literals, such as the let-literal or the pattern-match-literal, can bind one or more variables.
Unconventionally however, a binding is in scope of a literal if and only if
the binding literal is the left operand of a parent conjunction.

Thus, $(\ctlet{x}{y} \land x \ntermeq \bot) \land x \ntermeq \bot$ is semantically equivalent to $y \ntermeq \bot \land x \ntermeq \bot$.
Clearly, $\land$ is not associative!

To utilize this behavior, the operator $\andtheta$ replaces the rightmost operand of the top conjunction tree of the left argument (figure \ref{fig:refinementTypes}).

\subsection{$\generate$enerating Inhabitants}
LYG also describes a partial function $\generate$
with $\generate(\Theta) = \varnothing \Rightarrow (\Theta \text{ denotes } \varnothing)$ for all refinement types $\Theta$.
$\generate$ is used to $\generate$enerate inhabitants of a refinement type to build elaborate error messages, or, more importantly, to get a guarantee that a refinement type is empty.
A total correct function $\generate$ is uncomputable, as there are expressions (making use of recursively defined functions) that match a certain data constructor if and only if a given turing machine halts!
This thesis just assumes that ``interesting'' computable and correct functions $\generate$ exist,
so the details of $\generate$ as proposed by LYG do not matter.
In general, all proposed correctness statements should allow for an empty function $\mathcal{G}$.

\subsection{$\mathcal{U}$ncovered Analysis}
\label{chap:bckgrdUncoveredAnalysis}

The goal of the uncovered analysis is to detect all cases that are not handled by a given guard tree.
Refinement types are used to capture the result of this analysis.

The function $\unc(\reft{\Gamma}{\true}, \cdot)$ in figure \ref{fig:U}
computes a refinement type that captures all uncovered values for a given guard tree.
This refinement type is empty if and only if there are no uncovered cases.
If $\generate$ is used to test for emptiness, this already yields an algorithm to test for uncovered cases.
It can be verified that the uncovered refinement type of the guard tree in
figure \ref{fig:desugaringExample} ``semantically'' equals $\reft{ x_1{:}\ensuremath{\Conid{Case}}, \, x_2{:}\ensuremath{\Conid{Bool}}}{ x_1 \ntermeq \bot \andtheta x_1 \ntermeq \mathtt{Case1} \andtheta x_1 \ntermeq \mathtt{Case2} \andtheta x_1 \ntermeq \mathtt{Case3} }$ and denotes $x_1 = \mathtt{Case4}$.

\begin{figure}[htbp]
	\caption{Definition of $\unc$}
	\label{fig:U}
	\[ \ruleform{ \unc(\Theta, t) = \Theta } \]
	\[
		\begin{array}{lcl}
			\unc(\reft{\Gamma}{\Phi}, \gdtrhs{n})                                    & = & \reft{\Gamma}{\false}                                                                                                 \\
			\unc(\Theta, \gdtseq{t_1}{t_2})                                          & = & \unc(\unc(\Theta, t_1), t_2)                                                                                          \\
			\unc(\Theta, \gdtguard{\grdbang{x}}{t})                                  & = & \unc(\Theta \andtheta (x \ntermeq \bot), t)                                                                           \\
			\unc(\Theta, \gdtguard{\grdlet{x}{e}}{t})                                & = & \unc(\Theta \andtheta (\ctlet{x}{e}), t)                                                                              \\
			\unc(\Theta, \gdtguard{\grdcon{\genconapp{K}{a}{\gamma}{y:\tau}}{x}}{t}) & = & \Theta \andtheta (x \ntermeq K) \uniontheta \unc(\Theta \andtheta (\ctcon{\genconapp{K}{a}{\gamma}{y:\tau}}{x}), t) \\
		\end{array}
	\]
\end{figure}

\subsection{$\mathcal{A}$nnotated Guard Trees}

\textit{Annotated guard trees} represent simplified guard trees that have been annotated with refinement types $\Theta$.
They are made of RHSs, branches and bang nodes. Their syntax is defined in figure \ref{fig:annotatedGuardTree}. Annotated guard trees are the result of the function $\mathcal{A}$, which is discussed in the next chapter. 

\begin{figure}[htbp]
	\caption{Definition of Annotated Guard Trees}
	\label{fig:annotatedGuardTree}
	\centering
	\[
		u \in \Ant \Coloneqq \antrhs{\Theta}{k} \mid \antseq{u_1}{u_2} \mid \antbang{\Theta}{u}
	\]
\end{figure}

\subsection{$\mathcal{R}$edundant/Inaccessible Analysis}
\label{chap:backgrdRedundantInaccAnalysis}

The goal of the redundant/inaccessible analysis is to report as many RHSs
as possible that are redundant or inaccessible.
This is done by annotating a guard tree with refinement types and then checking these refinement types for emptiness.
If a RHS is associated with an empty refinement type, the RHS is inaccessible and in some circumstances even redundant.
The refinement type of a bang node describes all values under which an evaluation will diverge.
Figure \ref{fig:A} defines a function $\ann$ that computes such an annotation for a given guard tree.
Figure \ref{fig:AExample} shows the annotated tree of the introductory example in figure \ref{lst:haskell} with abbreviated refinement types.

Such an annotated guard tree is then passed to a function $\red$ as defined in figure \ref{fig:R}.
$\red$ uses $\generate$ to compute redundant and inaccessible RHSs. All other RHSs are assumed to be accessible,
even though, due to $\generate$ being a partial function, not all of them actually are accessible.

Figure \ref{fig:RExample} computes inaccessible and redundant leaves for an annotated tree that is
$(\mathcal{G}=\varnothing)$-equivalent to the annotated tree from figure $\ref{fig:AExample}$
for sensible functions $\mathcal{G}$.
It states that RHS 4 in \ref{lst:haskell} is redundant and can be removed, while RHS 3 is just inaccessible.

\begin{figure}[htbp]
	\caption{Definition of $\ann$}
	\label{fig:A}
	\[ \ruleform{ \ann(\Theta, t) = u } \]
	\[
		\begin{array}{lcl}
			\ann(\Theta,\gdtrhs{n})                                                  & = & \antrhs{\Theta}{n}                                                                       \\
			\ann(\Theta, \gdtseq{t_1}{t_2})                                          & = & \antseq{\ann(\Theta, t_1)}{\ann(\unc(\Theta, t_1), t_2)}                                 \\
			\ann(\Theta, \gdtguard{\grdbang{x}}{t})                                  & = & \antbang{\Theta \andtheta (x \termeq \bot)}{\ann(\Theta \andtheta (x \ntermeq \bot), t)} \\
			\ann(\Theta, \gdtguard{\grdlet{x}{e}}{t})                                & = & \ann(\Theta \andtheta (\ctlet{x}{e}), t)                                                 \\
			\ann(\Theta, \gdtguard{\grdcon{\genconapp{K}{a}{\gamma}{y:\tau}}{x}}{t}) & = & \ann(\Theta \andtheta (\ctcon{\genconapp{K}{a}{\gamma}{y:\tau}}{x}), t)                  \\
		\end{array}
	\]
\end{figure}

\begin{figure}[htbp]
	\caption{Examplary Evaluation of $\mathcal{A}$}
	\label{fig:AExample}
    \[\ann(\true,
    \vcenter{\hbox{
    \begin{forest}
		grdtree
		[
		[{$\grdbang{x_1},\, \grdcon{\mathtt{Case1}}{x_1}$} [1]]
			[
				[{$\grdbang{x_1},\, \grdcon{\mathtt{Case2}}{x_1}$} [2]]
					[
						[{$\grdlet{x}{x_1},\, \grdbang{x_2},\, \grdcon{\mathtt{True}}{x_2}$}
									[{$\grdbang{x},\, \grdcon{\mathtt{Case1}}{x}$} [3]]
									[{$\grdbang{x},\, \grdcon{\mathtt{Case2}}{x}$} [4]]
							]
							[{$\grdbang{x_1},\, \grdcon{\mathtt{Case3}}{x_1}$} [5]]
					]
			]
		]
	\end{forest}}}
	) = \]
	\[
	\begin{forest}
	    anttree
		[
		[{$\reft{\Gamma }{ x_1 \termeq \bot }$ \lightning} [$\reft{\Gamma }{ x_1 \ntermeq \bot, x_1 \termeq \mathtt{Case1} }$ 1]]
			[
				[{$\reft{\Gamma }{ x_1 \ntermeq \bot, x_1 \termeq \bot }$ \lightning} [$\reft{\Gamma }{ ..., x_1 \ntermeq \mathtt{Case1}, x_1 \termeq \mathtt{Case2}  }$ 2]]
					[
						[{$\reft{\Gamma }{ x_2 \termeq \bot }$ \lightning}
									[{$\reft{\Gamma }{ \false }$ \lightning} [$\reft{\Gamma }{ ..., x_1 \ntermeq \mathtt{Case1}, ..., x_1 \termeq \mathtt{Case1}  }$ 3]]
									[{$\reft{\Gamma }{ \false }$ \lightning} [$\reft{\Gamma }{ ..., x_1 \ntermeq \mathtt{Case2}, ..., x_1 \termeq \mathtt{Case2}  }$ 4]]
							]
							[{$\reft{\Gamma }{ \false }$ \lightning} [$\reft{\Gamma }{ ..., x_1 \termeq \mathtt{Case3}  }$ 5]]
					]
			]
		]
	\end{forest}
	\]
\end{figure}

\begin{figure}[htbp]
	\caption{Definition of $\red$. $\red$ partitions all RHSs into could-be-accessible $(\overline{k})$, inaccessible $(\overline{n})$ and $\red$edundant $(\overline{m})$ RHSs. }
	\label{fig:R}
	\centering
	\[ \ruleform{ \red(u) = (\overline{k}, \overline{n}, \overline{m}) } \]
	\[
		\begin{array}{lcl}
			\red(\antrhs{\Theta}{n})  & = & \begin{cases}
				(\epsilon, \epsilon, n), & \text{if $\generate(\Theta) = \emptyset$} \\
				(n, \epsilon, \epsilon), & \text{otherwise}                          \\
			\end{cases}                                                                                                                     \\
			\red(\antseq{t}{u})       & = & (\overline{k}\,\overline{k'}, \overline{n}\,\overline{n'}, \overline{m}\,\overline{m'}) \hspace{0.5em} \text{where} \begin{array}{l@{\,}c@{\,}l}
				(\overline{k}, \overline{n}, \overline{m})    & = & \red(t) \\
				(\overline{k'}, \overline{n'}, \overline{m'}) & = & \red(u) \\
			\end{array} \\
			\red(\antbang{\Theta}{t}) & = & \begin{cases}
				(\epsilon, m, \overline{m'}), & \text{if $\generate(\Theta) \not= \emptyset$ and $\red(t) = (\epsilon, \epsilon, m\,\overline{m'})$} \\
				\red(t),                      & \text{otherwise}                                                                                     \\
			\end{cases}                                                                                                                     \\
		\end{array}
	\]
\end{figure}

\begin{figure}[htbp]
	\caption{Examplary Evaluation of $\mathcal{R}$}
	\label{fig:RExample}
    \[
    \red(
    \vcenter{\hbox{
    	\begin{forest}
    	    anttree
    		[
    		[{$\reft{\Gamma }{ \true }$ \lightning} [$\reft{\Gamma }{ \true }$ 1]]
    			[
    				[{$\reft{\Gamma }{ \false }$ \lightning} [$\reft{\Gamma }{ \true  }$ 2]]
    					[
    						[{$\reft{\Gamma }{ \true }$ \lightning}
    									[{$\reft{\Gamma }{ \false }$ \lightning} [$\reft{\Gamma }{ \false }$ 3]]
    									[{$\reft{\Gamma }{ \false }$ \lightning} [$\reft{\Gamma }{ \false }$ 4]]
    							]
    							[{$\reft{\Gamma }{ \false }$ \lightning} [$\reft{\Gamma }{ \true }$ 5]]
    					]
    			]
    		]
    	\end{forest}
	}}
	) = (1 2 5, 3, 4)
	\]
\end{figure}

\clearpage

\section{Lean}

\subsection{The Lean Theorem Prover}

Lean is an interactive theorem prover that is based on the calculus of inductive constructions \cite{leanWebsite} \cite{moura15} and is developed by Microsoft Research.
It features dependent types, offers a high degree of automation through tactics and can also be used as a programming language.
Due to the Curry-Howard isomorphism, writing function definitions intended to be used in proofs,
writing proofs and writing proof-generating custom tactics is very similar.

We use Lean 3 for this thesis and want to give a brief overview of its syntax. See \cite{leanDocs} for a detailed documentation.

Inductive data types can be defined with the keyword $\mathtt{inductive}$. The $\mathtt{\#check}$ instruction can be used to type-check terms:
\begin{minted}{Lean}
inductive my_nat : Type
| zero : my_nat
| succ : my_nat → my_nat

#check my_nat.succ my_nat.zero
#check my_nat.zero.succ -- equivalent term, using dot notation
\end{minted}

The keyword $\mathtt{def}$ can be used to bind terms and define recursive functions:
\begin{minted}{Lean}
def my_nat.add : my_nat → my_nat → my_nat
-- Patterns can be used in definitions
| my_nat.zero b := b
| (my_nat.succ a) b := (a.add b).succ
\end{minted}

Likewise, $\mathtt{def}$ can be used to bind proof terms to propositions.
Propositions are stated as type and proved by constructing a term of that type.
$\Pi$-types are used to introduce generalized type variables:
\begin{minted}{Lean}
-- This type states that for all a, a + zero = a
def my_nat.add_zero_eq : Π a: my_nat, a.add my_nat.zero = a :=
    -- Proof by induction
    @my_nat.rec
        -- Induction Hypothesis
        (λ a, a.add my_nat.zero = a)
        -- Case Zero
        (my_nat.add.equations._eqn_1 my_nat.zero)
        -- Case Succ
        (λ a h,
            @eq.subst my_nat
                (λ x, (my_nat.succ a).add my_nat.zero = x.succ)
                (a.add my_nat.zero)
                a
                h
                (my_nat.add.equations._eqn_2 a my_nat.zero)
        )
\end{minted}

Proofs are usually much shorter when using Leans tactic mode.
Also, definitions can be parametrized (which generalizes the parameter) and the keywords $\mathtt{lemma}$ and $\mathtt{theorem}$ can be used instead of $\mathtt{def}$:
\begin{minted}{Lean}
lemma my_nat.add_zero' (a: my_nat): a.add my_nat.zero = a :=
begin
    induction a,
    { refl, },
    { simp [my_nat.add, *], },
end
\end{minted}

\subsection{The Lean Mathematical Library}

\textit{Mathlib} \cite{mathlibOverview} is a community project that offers a rich mathematical foundation for many theories in Lean 3.
Its theories of finite sets, lists, boolean logic and permutations have been very useful for this thesis.

Mathlib also offers many advanced tactics like $\mathtt{finish}$, $\mathtt{tauto}$ or $\mathtt{linarith}$.
These tactics help significantly in proving trivial lemmas.
