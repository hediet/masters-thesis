\chapter{Introduction}\label{sec:intro}

In functional programming, pattern matching is a very popular feature.
This is particularly true for Haskell, where you can define algebraic data types
and easily match on them in function definitions.
With increasingly complex data types and function definitions however,
pattern matching can be yet another source of mistakes.

Figure \ref{lst:haskell} showcases common types of mistakes that can arise with pattern matching.

Most importantly, the function $\mathtt{f}$ is not defined on all values:
Evaluating $\mathtt{f}\;\mathtt{Case4}$ will cause a runtime error!
In other words, the pattern match used to define $\mathtt{f}$ is not exhaustive,
as the input $\mathtt{Case4}$ is uncovered.
This is usually an oversight by the programmer and should be brought to their attention with an appropriate warning.

Also, $\mathtt{f}$ will never evaluate to $3$ or $4$ - replacing these values with any other value would not change any observable behaviour of $\mathtt{f}$. Such right hand sides (\textit{RHS}s) are called \textit{inaccessible}.
Inaccessible RHSs indicate a code smell and should be avoided too. Sometimes, such RHSs can simply be removed from the pattern.

\begin{figure}[htbp]
	\caption{A Pattern Matching Example In Haskell}
	\label{lst:haskell}
    \begin{minted}{Haskell}
    data Case = Case1 | Case2 | Case3 | Case4
    
    f :: Case -> Bool -> Integer
    f Case1 _ = 1
    f Case2 _ = 2
    f x True | Case1 <- x = 3
             | Case2 <- x = 4
    f Case3 _ = 5
    \end{minted}
\end{figure}


Lower Your Guards (LYG) \cite{10.1145/3408989} is a compiler analysis that is able to detect such mistakes and also can deal with the intricacies of lazy evaluation.

However, LYG is only checked empirically so far: Its implementation in the Glasgow Haskell Compiler just \textit{seems to work}.

Obviously, LYG would be incorrect if it marks a RHS as inaccessible even though it actually
is accessible. This could have fatal consequences: A programmer acting on such misinformation
might delete a RHS that is very much in use!

As LYG does not give a complete characterization of correctness,
we first want to establish a precise notion of correctness and then check that these algorithms indeed comply with it. At the very least, a verifying tool should be verified itself!

The large number of case distinctions made in the algorithms motivates the use of a theorem prover;
a natural proof would not be very trustworthy due to the high technical demand and risk of missing edge cases.

\vspace{\baselineskip}

The main contributions of this thesis are as follows:
\begin{itemize}
\item We formalized the uncovered and redundant/inaccessible analysis of LYG in Lean 3. This formalization is discussed in detail in chapter \ref{sec:formalization}.
We noticed an inaccuracy in how variable scopes are handled in refinement types
and present a counter-example to $\mathcal{U}$'s correctness in chapter \ref{sec:formalizationVariableBindingRules} by exploiting shadowing variable bindings.
We suggest a more explicit variable scoping mechanism of refinement types.
\item We establish a notion of correctness of LYG. Its formalization in Lean is discussed in chapter \ref{sec:formalizationCorrectnessStmts}. This notion of correctness is more precise and more complete than the notion of correctness presented in LYG.
\item We present formal proofs that the redundant/inaccessible analysis of LYG satisfies
    this notion of correctness if our suggestion of a more explicit scoping operator is applied. Details of this proof are discussed in chapter \ref{sec:proof}.

\end{itemize}