\documentclass[parskip=no,12pt,a4paper,twoside,headings=openright]{scrreprt}
% switch to scrbook if you want roman page numbers for the front matter
% however scrbook has no 'abstract' environment!
% if your thesis is in english, use "parskip=no" instead

% binding correction (BCOR) von 1cm für Leimbindung
\KOMAoptions{BCOR=1cm}
\KOMAoptions{draft=yes}

\usepackage[utf8]{inputenc} % encoding of sources
\usepackage[T1]{fontenc}
\usepackage{./thesis/style/studarbeit}
\usepackage{listings}
\usepackage[edges]{forest}
\usepackage{xspace}
\usepackage{mathtools}
\usepackage{amssymb}
\usepackage{ifsym}
\usepackage{wasysym}
\usepackage{minted}
\usepackage{fontspec}
\usepackage{unicode-math}
\usepackage{mathdesign}
\usepackage{svg}
\usepackage[ngerman, num]{isodate}

% https://tex.stackexchange.com/questions/343494/minted-red-box-around-greek-characters/343506#343506
\makeatletter
\AtBeginEnvironment{minted}{\dontdofcolorbox}
\def\dontdofcolorbox{\renewcommand\fcolorbox[4][]{##4}}
\makeatother

% https://leanprover.github.io/lean4/doc/syntax_highlight_in_latex.html
\usepackage{newunicodechar}
\newfontfamily{\freeserif}{DejaVu Sans}
\newunicodechar{∈}{\freeserif{\smallin}}
\newunicodechar{∧}{\freeserif{\land}}
\newunicodechar{∨}{\freeserif{\lor}}
\newunicodechar{∪}{\freeserif{\cup}}
\newunicodechar{∘}{\freeserif{\circ}}
\newunicodechar{ω}{\ensuremath{\mathcal{U}}}
\newunicodechar{ρ}{\ensuremath{\mathcal{R}}}
\newunicodechar{π}{\ensuremath{\mathcal{A}}}
\newunicodechar{∀}{\ensuremath{\forall}}
\newunicodechar{↔}{\ensuremath{\leftrightarrow}}
\newunicodechar{⟨}{\ensuremath{\langle}}
\newunicodechar{⟩}{\ensuremath{\rangle}}
\newunicodechar{∅}{\ensuremath{\varnothing}}
\newunicodechar{∉}{\ensuremath{\not\in}}
\newunicodechar{∩}{\ensuremath{\cap}}
\newunicodechar{⊆}{\ensuremath{\subseteq}}
\newunicodechar{∃}{\ensuremath{\exists}}
\newunicodechar{≠}{\ensuremath{\neq}}
\newunicodechar{ℕ}{\ensuremath{\mathbb{N}}}
\newunicodechar{Γ}{\ensuremath{\Rightarrow}}


\defaultfontfeatures{Scale = MatchLowercase}
\setmainfont{CMU Serif}[Scale = 1.0]
\setsansfont{CMU Sans Serif}
\setmonofont{CMU Typewriter Text}
\setmathfont{Latin Modern Math}

\title{Formal Verification of Pattern Matching Analyses}
\author{Henning Dieterichs}
\thesistype{Masterarbeit}
\zweitgutachter{Prof.~Dr.~rer.~nat.~Bernhard~Beckert}
\betreuer{M.~Sc.~Sebastian~Graf}
\coverimage{thesis/cover.png}
\abgabedatum{{\year=2021 \month=4 \day=6 \today}}


\begin{document}

\begin{otherlanguage}{ngerman} % Titelseite ist immer auf Deutsch
	\mytitlepage
\end{otherlanguage}

\begin{abstract}
	\begin{center}\Huge\textbf{\textsf{Zusammenfassung}}
	\end{center}
	\vfill
    Die in Lower Your Guards vorgestellten Algorithmen analysieren Pattern Matching Definitionen
	und erkennen nicht abgedeckte Fälle, aber auch unzugängliche und redundante rechte Seiten.
	
	Ihre Implementierung in GHC entdeckte erfolgreich bisher unbekannte Fehler in Haskell Quellcode.
    Während die empirische Validierung über eine große Menge von Haskell-Code die Behauptung der Korrektheit untermauert,
    fehlt den Autoren eine präzise Formalisierung sowie ein Beweis für diese Behauptung.
	
	Diese Arbeit etabliert einen präzisen Begriff von Korrektheit und
	präsentiert formale Beweise, dass diese Algorithmen tatsächlich korrekt sind.
	Diese Beweise sind in Lean 3 formalisiert.

	\vfill

	The algorithms presented in Lower Your Guards analyze pattern matching definitions
	and detect uncovered cases, but also inaccessible and redundant right hand sides.
	
	Their implementation in the GHC spotted previously unknown bugs in real world code.
    While empirical validation over a large corpus of Haskell code corroborates the claim of correctness, the authors lack a precise formalization as well as a proof of that claim.
	
	This thesis establishes a precise notion of correctness and
	presents formal proofs that these algorithms are indeed correct.
	These proofs are formalized in Lean 3.
	\vfill

\end{abstract}

\tableofcontents



%%%%%%%%% HERE THE MACROS START
% Highlight changes and keyword undone
\newcommand{\highlight}[1]{\setlength{\fboxsep}{2pt}\colorbox[gray]{0.8}{\ensuremath{#1}}} % less height

\newcommand{\TODOI}[1]{\textcolor{red}{#1}}
\newcommand{\UNDONE}{\begin{color}{red}{\bf UNDONE }\end{color}}
\newcommand{\DONE}{\begin{color}{green}{\bf DONE}\end{color}}
\newcommand{\NOTE}[1]{\bf #1}
\newcommand{\TODO}[1]{{\bf{\begin{color}{red}{TODO: }\end{color} #1}}}
\newcommand{\lyg}{LYG\xspace}
\newcommand{\gmtm}{GMTM\xspace}

\newcommand{\cf}{cf.\@\xspace}
\newcommand{\eg}{e.g.,\@\xspace}
\newcommand{\ie}{i.e.\@\xspace}
\newcommand{\vs}{vs.\@\xspace}
\newcommand{\etc}{etc.\@\xspace}
\newcommand{\keyword}[1]{\textsf{\textbf{#1}}}
\newcommand{\id}[1]{\textsf{\textsl{#1}}\xspace}

% Useful macros that are often needed
\newcommand{\typeeq}{\sim}          % type equality
\newcommand{\termeq}{\approx}       % positive term equality
\newcommand{\ntermeq}{\not\approx}  % negative term equality

\newcommand{\ruleform}[1]{\fbox{$#1$}}
\newcommand{\rulename}[1]{\textsc{[#1]}}
\newcommand{\freein}{\;\#\;}

\newcommand{\ticket}[1]{\href{https://ghc.haskell.org/trac/ghc/ticket/#1}{\##1}}
\newcommand{\extension}[1]{\texttt{#1}}

% \newcommand{\sg}[1]{\begin{color}{red}{\bf SG:} #1\end{color}}
% \newcommand{\simon}[1]{\begin{color}{purple}{\bf SLPJ:} #1\end{color}}
% \newcommand{\ryan}[1]{\begin{color}{orange}{\bf Ryan:} #1\end{color}}
\newcommand{\sg}[1]{}
\newcommand{\simon}[1]{}
\newcommand{\ryan}[1]{}

\newcommand{\Conid}[1]{\mathit{#1}}
\newcommand{\Varid}[1]{\mathit{#1}}

% Types and Grd syntax
\newcommand{\ty}[1]{\textsf{#1}\xspace}
\newcommand{\Pat}{\ty{Pat}}
\newcommand{\Grd}{\ty{Grd}}
\newcommand{\Con}{\ty{Con}}
\newcommand{\Var}{\ty{Var}}
\newcommand{\Expr}{\ty{Expr}}
\newcommand{\Type}{\ty{Type}}
\newcommand{\Kind}{\ty{Kind}}
\newcommand{\TyCt}{\ty{TyCt}}
\newcommand{\NT}{\ty{NT}}
\newcommand{\PS}{\ty{PS}}
\newcommand{\CL}{\ty{CL}}
\newcommand{\grdlet}[2]{\textsf{let}\;#1\,\textsf{=}\,#2}
\newcommand{\grdbang}[1]{\textsf{!}#1}
\newcommand{\grdcon}[2]{#1 \leftarrow #2}
\newcommand{\ctlet}[2]{\textsf{let}\;#1\,\textsf{=}\,#2}
\newcommand{\ctcon}[2]{#1 \leftarrow #2}
\newcommand{\genconapp}[4]{#1\;\overline{#2}\;\overline{#3}\;\overline{#4}}
\newcommand{\expconapp}[4]{#1\;\overline{#2}\;\overline{#3}\;\overline{#4}}
\newcommand{\deltaconapp}[3]{#1\;\overline{#2}\;\overline{#3}}
\newcommand{\ntconapp}[3]{#1\;\overline{#2}\;#3}
\newcommand{\false}{\times}
\newcommand{\true}{\checked}

% GrdTree Gdt
\newcommand{\Gdt}{\ty{Gdt}}
\newcommand{\gdtrhs}[1]{
	\vcenter{\hbox{\begin{forest}
				grdtree,
				[ [{$#1$}] ]
			\end{forest}}}}
\newcommand{\gdtseq}[2]{
	\vcenter{\hbox{\begin{forest}
				grdtree,
				for tree={delay={edge={-}}},
				[ [{$#1$}] [{$#2$}] ]
			\end{forest}}}}
\newcommand{\gdtguard}[2]{
	\vcenter{\hbox{\begin{forest}
				grdtree,
				grhs/.style={tier=rhs,edge={-}},
				[ [{$#1$} [{$#2$}] ] ]
			\end{forest}}}}
\newcommand{\gdtempty}{\bullet_{\Gdt}}

% AnnTree Ant
\newcommand{\Ant}{\ty{Ant}}
\newcommand{\antrhs}[2]{
	\vcenter{\hbox{\begin{forest}
				anttree,
				[ [{$#1$\,$#2$}] ]
			\end{forest}}}}
\newcommand{\antseq}[2]{
	\vcenter{\hbox{\begin{forest}
				anttree,
				for tree={delay={edge={-}}},
				[ [{$#1$}] [{$#2$}] ]
			\end{forest}}}}
\newcommand{\antbang}[2]{
	\vcenter{\hbox{\begin{forest}
				anttree,
				for tree={delay={edge={-}}},
				[ [{$#1$\,\lightning} [{$#2$}] ] ]
			\end{forest}}}}
\newcommand{\antempty}{\bullet_{\Ant}}

% Graphic notation for trees
\forestset{%
	clausetree/.style={
			for tree={
					grow'=0,
					calign=first,
					anchor=parent,
					line width=0.2mm, % This one only affects layout, not appearance of lines. Hence we set this again in delay
					inner sep=2pt,
					s sep=0pt,
					delay={
							edge={line width=0.2mm}},
					l=0em % fix some excessive horizontal space usage for empty nodes
				},
			forked edges
		},
	grdtree/.style={
			clausetree,
			guards/.style={edge={-Bar}},
			grhs/.style={tier=rhs,edge={->}},
			% Everything except the root and the leaves is guards
			for descendants={delay={if n children=0{grhs}{guards}}}
		},
	anttree/.style={
			clausetree,
			arhs/.style={tier=rhs,edge={->}},
			for descendants={delay={if n children=0{arhs}{}}}
		},
}

% Desugaring function
\newcommand{\ds}{\mathcal{D}}

% Checking functions
\newcommand{\unc}{\mathcal{U}}
\newcommand{\ann}{\mathcal{A}}
\newcommand{\red}{\mathcal{R}}
\newcommand{\uncann}{\unc\hspace{-0.35em}\ann}

% Refinement type functions
\newcommand{\generate}{\mathcal{G}}
\newcommand{\normalise}{\mathcal{N}}
\newcommand{\expand}{\mathcal{E}}

% Refinement type syntax
\newcommand{\reft}[2]{\langle \, #1 \mid #2 \, \rangle}
\newcommand{\andtheta}{\,\dot{\wedge}\,}
\newcommand{\uniontheta}{\,\cup\,}

% Normalised refinement types
\newcommand{\inv}[1]{I#1}
\newcommand{\nreft}[2]{\langle #1 \!\parallel\! #2 \rangle}
\newcommand{\adddelta}{\,\oplus_{\delta}\,}
\newcommand{\restrict}[2]{#1 \! \mid_{#2}}
\newcommand{\addphi}{\,\oplus_{\varphi}\,}
\newcommand{\inhabited}[2]{#1 \vdash #2 \, \mathsf{inh}}
\newcommand{\inhabitedbot}{\textsc{$\vdash$Bot}\xspace}
\newcommand{\inhabitednocpl}{\textsc{$\vdash$NoCpl}\xspace}
\newcommand{\inhabitedinst}{\textsc{$\vdash$Inst}\xspace}
\newcommand{\inhabitednt}{\textsc{$\vdash$NT}\xspace}
\newcommand{\cons}{\mathsf{Cons}}
\newcommand{\inst}{\mathsf{Inst}}
\newcommand{\rep}[2]{#1(#2)}
\newcommand{\repnt}[2]{#1_{\text{\tiny NT}}(#2)}
\newcommand{\addphiv}{\,\dot{\oplus}_{\varphi}\,} % "Vectorised" \addphi, hence \addphiv
\newcommand{\throttle}[2]{\left\lfloor#2\right\rfloor_{#1}}
\chapter{Introduction}\label{sec:intro}

In functional programming, pattern matching is a very popular feature.
This is particularly true for Haskell, where you can define algebraic data types
and easily match on them in function definitions.
With increasingly complex data types and function definitions however,
pattern matching can be yet another source of mistakes.

Figure \ref{lst:haskell} showcases common types of mistakes that can arise with pattern matching.

Most importantly, the function $\mathtt{f}$ is not defined on all values:
Evaluating $\mathtt{f}\;\mathtt{Case4}$ will cause a runtime error!
In other words, the pattern match used to define $\mathtt{f}$ is not exhaustive,
as the input $\mathtt{Case4}$ is uncovered.
This is usually an oversight by the programmer and should be brought to their attention with an appropriate warning.

Also, $\mathtt{f}$ will never evaluate to $3$ or $4$ - replacing these values with any other value would not change any observable behavior of $\mathtt{f}$. Such right hand sides (\textit{RHS}s) are called \textit{inaccessible}.
Inaccessible RHSs indicate a code smell and should be avoided too. Sometimes, such RHSs can simply be removed from the pattern.

\begin{figure}[htbp]
	\caption{A Pattern Matching Example In Haskell}
	\label{lst:haskell}
    \begin{minted}{Haskell}
    data Case = Case1 | Case2 | Case3 | Case4
    
    f :: Case -> Bool -> Integer
    f Case1 _ = 1
    f Case2 _ = 2
    f x True | Case1 <- x = 3
             | Case2 <- x = 4
    f Case3 _ = 5
    \end{minted}
\end{figure}


Lower Your Guards (LYG) \cite{10.1145/3408989} is a compiler analysis that is able to detect such mistakes and also can deal with the intricacies of lazy evaluation.

However, LYG is only checked empirically so far: Its implementation in the Glasgow Haskell Compiler just \textit{seems to work}.

Obviously, LYG would be incorrect if it marks a RHS as inaccessible even though it actually
is accessible. This could have fatal consequences: A programmer acting on such misinformation
might delete a RHS that is very much in use!

As LYG does not give a complete characterization of correctness,
we first want to establish a precise and complete notion of correctness and then check that these algorithms indeed comply with it. At the very least, a verifying tool should be verified itself!

The large number of case distinctions made in the algorithms motivates the use of a theorem prover;
a natural proof would not be very trustworthy due to the high technical demand and risk of missing edge cases.

\vspace{\baselineskip}

The main contributions of this thesis are as follows:
\begin{itemize}
\item We formalized the uncovered and redundant/inaccessible analysis of LYG in Lean 3. This formalization is discussed in detail in chapter \ref{sec:formalization}.
We noticed an inaccuracy in how variable scopes are handled in refinement types
and present a counter-example to $\mathcal{U}$'s correctness in chapter \ref{sec:formalizationVariableBindingRules} by exploiting shadowing variable bindings.
We suggest a more explicit variable scoping mechanism of refinement types.
\item We establish a notion of correctness of LYG. Its formalization in Lean is discussed in chapter \ref{sec:formalizationCorrectnessStmts}. This notion of correctness is more precise and more complete than the notion of correctness presented in LYG.
\item We present formal proofs that the redundant/inaccessible analysis of LYG satisfies
    this notion of correctness if our suggestion of a more explicit scoping operator is applied. Details of this proof are discussed in chapter \ref{sec:proof}.

\end{itemize}
\chapter{Background}\label{sec:background}

\section{Lower Your Guards}

Lower Your Guards (LYG) \cite{10.1145/3408989} describes algorithms that analyze pattern matching expressions and report uncovered cases, but also redundant and inaccessible right hand sides.

LYG was designed for use in the Glasgow Haskell Compiler,
but the algorithm and its data structures are so universal
that they can be leveraged for other programming languages with pattern matching constructs too.

All definitions and some examples of this chapter are taken from LYG \cite{10.1145/3408989}.

\subsection{Inaccessible vs. Redundant RHSs}

A closer look at figure \ref{lst:haskell} reveals that while both RHS 3 and 4 are inaccessible,
the semantics of $\mathtt{f}$ changes if both are removed.
This means that an automated refactoring cannot just remove all inaccessible leaves!

The reason for this is the term $t := \mathtt{f}\;\mathtt{Case3}\;\mathtt{undefined}$ and the fact that Haskell uses a lazy evaluation strategy.
If both RHSs $3$ and $4$ are removed, $t$ evaluates to $5$ - the term $\mathtt{undefined}$ is never evaluated as no pattern matches against it.
However, if nothing or only one of the RHSs $3$ or $4$ is removed, $\mathtt{undefined}$ will be matched with $\mathtt{True}$ and thus $t$ will throw a runtime error!

To communicate this difference, LYG introduces the concept of $\mathit{redundant}$ and $\mathit{inaccessible}$ RHSs:
A redundant RHS can be removed from its pattern matching expression without any observable difference.
An inaccessible RHS is never evaluated, but its removal might lead to observable changes.
This definition implies that redundant RHSs are inaccessible.

As of listing \ref{lst:haskell}, LYG will mark RHS $3$ as inaccessible and RHS $4$ as redundant.
This choice is somewhat arbitrary, as RHS $3$ could be marked as redundant and RHS $4$ as inaccessible as well, and will be discussed in more detail chapter \TODOI{ref}.

\subsection{Guard Trees}

For all analyses, LYG first transforms Haskell specific pattern match expressions to simpler \textit{guard trees}.
This transformation removes a lot of complexity, as many different Haskell constructs can be desugared
to the same guard tree. Guard trees also simplify adapting LYG to other programming languages
and they enable studying LYG mostly independent from Haskell.
Their syntax is defined in figure \ref{fig:guardTrees}.

Guard trees (\textit{Gdt}s) are made of three elements: Uniquely numbered right hand sides, \textit{branches} and \textit{guarded trees}.
Guarded trees refer to Haskell specific guards (\textit{Grd}) that control the execution.
\textit{Let guards} can bind a term to a variable in a new lexical scope.
\textit{pattern match guards} can destructure a value into variables if the pattern matches or otherwise prevent the
execution from entering the tree behind the guard.
Finally, \textit{bang guards} can stop the entire execution when the value of a variable does not reduce to a head normal form.

\begin{figure}[htbp]
	\caption{Definition of Guard Trees}
	\label{fig:guardTrees}
	\centering
	\[ \textbf{Guard Syntax} \]
	\[
		\begin{array}{cc}
			\begin{array}{rlcl}
				k,n,m       \in & \mathbb{N} &           &                                                 \\
				K           \in & \Con       &           &                                                 \\
				x,y,a,b     \in & \Var       &           &                                                 \\
				\tau,\sigma \in & \Type      & \Coloneqq & a \mid ...                                      \\
				e \in           & \Expr      & \Coloneqq & x \mid  \genconapp{K}{\tau}{\gamma}{e} \mid ... \\
			\end{array} &
			\begin{array}{rlcl}
				\gamma \in & \TyCt & \Coloneqq & \tau_1 \typeeq \tau_2 \mid ...               \\
				p \in      & \Pat  & \Coloneqq & \_ \mid K \; \overline{p} \mid ...           \\
				g \in      & \Grd  & \Coloneqq & \grdlet{x:\tau}{e}                           \\
				           &       & \mid      & \grdcon{\genconapp{K}{a}{\gamma}{y:\tau}}{x} \\
				           &       & \mid      & \grdbang{x}                                  \\
			\end{array}
		\end{array}
	\]

	\[ \textbf{Guard Tree Syntax} \]
	\[
		\begin{array}{rcll}
			t \in \Gdt & \Coloneqq & \gdtrhs{k} \mid \gdtseq{t_1}{t_2} \mid \gdtguard{g}{t} \\
		\end{array}
	\]
\end{figure}

The evaluation of a guard tree selects the first right hand side that execution reaches.
If the execution stops at a bang guard, the evaluation is said to \textit{diverge}, otherwise, if execution falls through, the evaluation ends with a \textit{no-match}.
A formal semantic for guard trees will be defined in chapter \ref{chap:formalGuardTrees}.

The transformation from Haskell pattern matches to guard trees is not of much interest for this thesis and can be found in LYG \cite{10.1145/3408989}.
To preserve semantics, it is important that the transformation inserts bang guards whenever a variable is matched against a data constructor.

Figure \ref{fig:desugaringExample} presents the transformation of figure \ref{lst:haskell} into a guard tree.


\begin{figure}[htbp]
	\caption{Desugaring Example}
	\label{fig:desugaringExample}
	\centering
	\begin{minted}{Haskell}
data Case = Case1 | Case2 | Case3 | Case4

f :: Case -> Bool -> Integer
f Case1 _ = 1
f Case2 _ = 2
f x True | Case1 <- x = 3
         | Case2 <- x = 4
f Case3 _ = 5
\end{minted}

	$\Downarrow$

	\begin{forest}
		grdtree
		[
		[{$\grdbang{x_1},\, \grdcon{\mathtt{Case1}}{x_1}$} [1]]
			[
				[{$\grdbang{x_1},\, \grdcon{\mathtt{Case2}}{x_1}$} [2]]
					[
						[{$\grdlet{x}{x_1},\, \grdbang{x_2},\, \grdcon{\mathtt{True}}{x_2}$}
									[{$\grdbang{x},\, \grdcon{\mathtt{Case1}}{x}$} [3]]
									[{$\grdbang{x},\, \grdcon{\mathtt{Case2}}{x}$} [4]]
							]
							[{$\grdbang{x_1},\, \grdcon{\mathtt{Case3}}{x_1}$} [5]]
					]
			]
		]
	\end{forest}
\end{figure}

\vspace{\baselineskip}

It is usually straightforward to define a transformation from pattern matching expressions to guard trees
that also preserves uncovered cases and inaccessible and redundant RHSs.
This makes guard trees an ideal abstraction for the following analysis steps.

\subsection{Refinement Types}

\textit{Refinement types} describe vectors of values $x_1, ..., x_n$ that satisfy a given predicate $\Phi$.
Their syntax is defined in figure \ref{fig:refinementTypes}.

\begin{figure}[htbp]
	\caption{Definition of Refinement Types}
	\label{fig:refinementTypes}
	\centering
	\[
		\begin{array}{rcll}
			\Gamma  & \Coloneqq & \varnothing \mid \Gamma, x:\tau \mid \Gamma, a                                                                                                  & \text{Context}         \\
			\varphi & \Coloneqq & \true \mid \false \mid \ctcon{\genconapp{K}{a}{\gamma}{y:\tau}}{x} \mid x \ntermeq K \\
			        &           & \mid x \termeq \bot \mid x \ntermeq \bot \mid \ctlet{x}{e} & \text{Literals}        \\
			\Phi    & \Coloneqq & \varphi \mid \Phi \wedge \Phi \mid \Phi \vee \Phi                                                                                               & \text{Formula}         \\
			\Theta  & \Coloneqq & \reft{\Gamma}{\Phi}                                                                                                                             & \text{Refinement type} \\
		\end{array}
	\]
    \[ \textbf{Operations on $\Theta$} \]
    \[
    \begin{array}{lcl}
        \Phi \andtheta \varphi &=&
        \begin{cases}
            \varphi_1 \wedge (\varphi_2 \andtheta \varphi) \; \textrm{if }\Phi = \varphi_1 \wedge \varphi_2 \\
            \Phi \wedge \varphi \; \textrm{otherwise} \\
        \end{cases}\\
        \reft{\Gamma}{\Phi} \andtheta \varphi &=&
            \reft{\Gamma}{\varphi \andtheta \varphi}\\
        \reft{\Gamma}{\Phi_1} \uniontheta \reft{\Gamma}{\Phi_2} &=& \reft{\Gamma}{\Phi_1 \vee \Phi_2} \\
    \end{array}
    \]
\end{figure}

Refinement predicates are built from literals $\phi$ and closed under conjunction and disjunction.
The literal $\true$ refers to ``true'', while $\false$ refers to ``false''. For example:
$$
	\begin{array}{rcl}
		\reft{ x{:}\ensuremath{\Conid{Bool}}}{ \true }                                                                                                           & \text{denotes} & \{ \bot, \ensuremath{\Conid{True}}, \ensuremath{\Conid{False}} \}                       \\
		\reft{ x{:}\ensuremath{\Conid{Bool}}}{ x \ntermeq \bot }                                                                                                 & \text{denotes} & \{ \ensuremath{\Conid{True}}, \ensuremath{\Conid{False}} \}                             \\
		\reft{ x{:}\ensuremath{\Conid{Bool}}}{ x \ntermeq \bot \wedge \ctcon{\ensuremath{\Conid{True}}}{x} }                                                     & \text{denotes} & \{ \ensuremath{\Conid{True}} \}                                                         \\
		\reft{ mx{:}\ensuremath{\Conid{Maybe}\;\Conid{Bool}}}{ mx \ntermeq \bot \wedge \ctcon{\ensuremath{\Conid{Just}\;\Varid{x}}}{mx}  } & \text{denotes} & \ensuremath{\Conid{Just}\;}\{ \ensuremath{\bot}, \ensuremath{\Conid{True}}, \ensuremath{\Conid{False}}, \} \\
	\end{array}
$$

\subsection{Binding Mechanism Of Refinement Types}\label{chap:bckgrndRefinementTypesBinding}

Refinement type literals, such as the let-literal or the pattern-match-literal can bind one or more variables.
Unconventionally however, a binding is in scope of a literal if and only if
the binding literal is the left operand of a parent conjunction.

Thus, $(\ctlet{x}{y} \land x \ntermeq \bot) \land x \ntermeq \bot$ is semantically equivalent to $y \ntermeq \bot \land x \ntermeq \bot$.
Clearly, $\land$ is not associative!

To utilize this behaviour, the operator $\andtheta$ replaces the rightmost operand of the top conjunction tree of the left argument (figure \ref{fig:refinementTypes}).

\subsection{$\generate$enerating Inhabitants}
LYG also describes a partial function $\generate$
with $\generate(\Theta) = \varnothing \Rightarrow (\Theta \text{ denotes } \varnothing)$ for all refinement types $\Theta$.
$\generate$ is used to $\generate$enerate inhabitants of a refinement type to build elaborate error messages
and to get a guarantee that a refinement type is empty.
A total correct function $\generate$ is uncomputable, since refinement types can make use of recursively defined functions!
This thesis just assumes that ``interesting'' computable and correct functions $\generate$ exist,
so the details of $\generate$ as proposed by LYG do not matter.
In general, all proposed correctness statements should allow for an empty function $\mathcal{G}$.

\subsection{Uncovered Analysis}
\label{chap:bckgrdUncoveredAnalysis}

The goal of the uncovered analysis is to detect all cases that are not handled by a given guard tree.
Refinement types are used to capture the result of this analysis.

The function $\unc(\reft{\Gamma}{\true}, \cdot)$ in figure \ref{fig:U}
computes a refinement type that captures all uncovered values for a given guard tree.
This refinement type is empty if and only if there are not any uncovered cases.
If $\generate$ is used to test for emptiness, this already yields an algorithm to test for uncovered cases.
It can be verified that the uncovered refinement type of the guard tree in
figure \ref{fig:desugaringExample} ``semantically'' equals $\reft{ x_1{:}\ensuremath{\Conid{Case}}, \, x_2{:}\ensuremath{\Conid{Bool}}}{ x_1 \ntermeq \bot \andtheta x_1 \ntermeq \mathtt{Case1} \andtheta x_1 \ntermeq \mathtt{Case2} \andtheta x_1 \ntermeq \mathtt{Case3} }$ and denotes $x_1 = \mathtt{Case4}$.

\begin{figure}[htbp]
	\caption{Definition of $\unc$}
	\label{fig:U}
	\[ \ruleform{ \unc(\Theta, t) = \Theta } \]
	\[
		\begin{array}{lcl}
			\unc(\reft{\Gamma}{\Phi}, \gdtrhs{n})                                    & = & \reft{\Gamma}{\false}                                                                                                 \\
			\unc(\Theta, \gdtseq{t_1}{t_2})                                          & = & \unc(\unc(\Theta, t_1), t_2)                                                                                          \\
			\unc(\Theta, \gdtguard{\grdbang{x}}{t})                                  & = & \unc(\Theta \andtheta (x \ntermeq \bot), t)                                                                           \\
			\unc(\Theta, \gdtguard{\grdlet{x}{e}}{t})                                & = & \unc(\Theta \andtheta (\ctlet{x}{e}), t)                                                                              \\
			\unc(\Theta, \gdtguard{\grdcon{\genconapp{K}{a}{\gamma}{y:\tau}}{x}}{t}) & = & \Theta \andtheta (x \ntermeq K) \uniontheta \unc(\Theta \andtheta (\ctcon{\genconapp{K}{a}{\gamma}{y:\tau}}{x}), t) \\
		\end{array}
	\]
\end{figure}

\subsection{Annotated Guard Trees}

\textit{Annotated guard trees} represent simplified guard trees that have been annotated with refinement types $\Theta$.
They are made of RHSs, branches and bang nodes. Their syntax is defined in figure \ref{fig:annotatedGuardTree}.

\begin{figure}[htbp]
	\caption{Definition of Annotated Guard Trees}
	\label{fig:annotatedGuardTree}
	\centering
	\[
		u \in \Ant \Coloneqq \antrhs{\Theta}{k} \mid \antseq{u_1}{u_2} \mid \antbang{\Theta}{u}
	\]
\end{figure}

\subsection{Redundant/Inaccessible Analysis}
\label{chap:backgrdRedundantInaccAnalysis}

The goal of the redundant/inaccessible analysis is to report as much RHSs
as possible that are redundant or inaccessible.
This is done by annotating a guard tree with refinement types and then checking these refinement types for emptiness.
If a RHS is associated with an empty refinement type, the RHS is inaccessible and in some circumstances even redundant.
The refinement type of a bang node describes all values under which an evaluation will diverge.
Figure \ref{fig:A} defines a function $\ann$ that computes such an annotation for a given guard tree.
Figure \ref{fig:AExample} shows the annotated tree of the introductory example in figure \ref{lst:haskell} with abbreviated refinement types.

Such an annotated guard tree is then passed to a function $\red$ as defined in figure \ref{fig:R}.
$\red$ uses $\generate$ to compute redundant and inaccessible RHSs. All other RHSs are assumed to be accessible,
even though, due to $\generate$ being a partial function, not all of them actually are accessible.

Figure \ref{fig:RExample} computes inaccessible and redundant leaves for an annotated tree that is
$(\mathcal{G}=\varnothing)$-equivalent to the annotated tree from figure $\ref{fig:AExample}$
for sensible functions $\mathcal{G}$.
It states that RHS 4 in \ref{lst:haskell} is redundant and can be removed, while RHS 3 is just inaccessible.

\begin{figure}[htbp]
	\caption{Definition of $\ann$}
	\label{fig:A}
	\[ \ruleform{ \ann(\Theta, t) = u } \]
	\[
		\begin{array}{lcl}
			\ann(\Theta,\gdtrhs{n})                                                  & = & \antrhs{\Theta}{n}                                                                       \\
			\ann(\Theta, \gdtseq{t_1}{t_2})                                          & = & \antseq{\ann(\Theta, t_1)}{\ann(\unc(\Theta, t_1), t_2)}                                 \\
			\ann(\Theta, \gdtguard{\grdbang{x}}{t})                                  & = & \antbang{\Theta \andtheta (x \termeq \bot)}{\ann(\Theta \andtheta (x \ntermeq \bot), t)} \\
			\ann(\Theta, \gdtguard{\grdlet{x}{e}}{t})                                & = & \ann(\Theta \andtheta (\ctlet{x}{e}), t)                                                 \\
			\ann(\Theta, \gdtguard{\grdcon{\genconapp{K}{a}{\gamma}{y:\tau}}{x}}{t}) & = & \ann(\Theta \andtheta (\ctcon{\genconapp{K}{a}{\gamma}{y:\tau}}{x}), t)                  \\
		\end{array}
	\]
\end{figure}

\begin{figure}[htbp]
	\caption{Examplary Evaluation of $\mathcal{A}$}
	\label{fig:AExample}
    \[\ann(\true,
    \vcenter{\hbox{
    \begin{forest}
		grdtree
		[
		[{$\grdbang{x_1},\, \grdcon{\mathtt{Case1}}{x_1}$} [1]]
			[
				[{$\grdbang{x_1},\, \grdcon{\mathtt{Case2}}{x_1}$} [2]]
					[
						[{$\grdlet{x}{x_1},\, \grdbang{x_2},\, \grdcon{\mathtt{True}}{x_2}$}
									[{$\grdbang{x},\, \grdcon{\mathtt{Case1}}{x}$} [3]]
									[{$\grdbang{x},\, \grdcon{\mathtt{Case2}}{x}$} [4]]
							]
							[{$\grdbang{x_1},\, \grdcon{\mathtt{Case3}}{x_1}$} [5]]
					]
			]
		]
	\end{forest}}}
	) = \]
	\[
	\begin{forest}
	    anttree
		[
		[{$\reft{\Gamma }{ x_1 \termeq \bot }$ \lightning} [$\reft{\Gamma }{ x_1 \ntermeq \bot, x_1 \termeq \mathtt{Case1} }$ 1]]
			[
				[{$\reft{\Gamma }{ x_1 \ntermeq \bot, x_1 \termeq \bot }$ \lightning} [$\reft{\Gamma }{ ..., x_1 \ntermeq \mathtt{Case1}, x_1 \termeq \mathtt{Case2}  }$ 2]]
					[
						[{$\reft{\Gamma }{ x_2 \termeq \bot }$ \lightning}
									[{$\reft{\Gamma }{ \false }$ \lightning} [$\reft{\Gamma }{ ..., x_1 \ntermeq \mathtt{Case1}, ..., x_1 \termeq \mathtt{Case1}  }$ 3]]
									[{$\reft{\Gamma }{ \false }$ \lightning} [$\reft{\Gamma }{ ..., x_1 \ntermeq \mathtt{Case2}, ..., x_1 \termeq \mathtt{Case2}  }$ 4]]
							]
							[{$\reft{\Gamma }{ \false }$ \lightning} [$\reft{\Gamma }{ ..., x_1 \termeq \mathtt{Case3}  }$ 5]]
					]
			]
		]
	\end{forest}
	\]
\end{figure}

\begin{figure}[htbp]
	\caption{Definition of $\red$. $\red$ partitions all RHSs into could-be-accessible $(\overline{k})$, inaccessible $(\overline{n})$ and $\red$edundant $(\overline{m})$ RHSs. }
	\label{fig:R}
	\centering
	\[ \ruleform{ \red(u) = (\overline{k}, \overline{n}, \overline{m}) } \]
	\[
		\begin{array}{lcl}
			\red(\antrhs{\Theta}{n})  & = & \begin{cases}
				(\epsilon, \epsilon, n), & \text{if $\generate(\Theta) = \emptyset$} \\
				(n, \epsilon, \epsilon), & \text{otherwise}                          \\
			\end{cases}                                                                                                                     \\
			\red(\antseq{t}{u})       & = & (\overline{k}\,\overline{k'}, \overline{n}\,\overline{n'}, \overline{m}\,\overline{m'}) \hspace{0.5em} \text{where} \begin{array}{l@{\,}c@{\,}l}
				(\overline{k}, \overline{n}, \overline{m})    & = & \red(t) \\
				(\overline{k'}, \overline{n'}, \overline{m'}) & = & \red(u) \\
			\end{array} \\
			\red(\antbang{\Theta}{t}) & = & \begin{cases}
				(\epsilon, m, \overline{m'}), & \text{if $\generate(\Theta) \not= \emptyset$ and $\red(t) = (\epsilon, \epsilon, m\,\overline{m'})$} \\
				\red(t),                      & \text{otherwise}                                                                                     \\
			\end{cases}                                                                                                                     \\
		\end{array}
	\]
\end{figure}

\begin{figure}[htbp]
	\caption{Examplary Evaluation of $\mathcal{R}$}
	\label{fig:RExample}
    \[
    \red(
    \vcenter{\hbox{
    	\begin{forest}
    	    anttree
    		[
    		[{$\reft{\Gamma }{ \true }$ \lightning} [$\reft{\Gamma }{ \true }$ 1]]
    			[
    				[{$\reft{\Gamma }{ \false }$ \lightning} [$\reft{\Gamma }{ \true  }$ 2]]
    					[
    						[{$\reft{\Gamma }{ \true }$ \lightning}
    									[{$\reft{\Gamma }{ \false }$ \lightning} [$\reft{\Gamma }{ \false }$ 3]]
    									[{$\reft{\Gamma }{ \false }$ \lightning} [$\reft{\Gamma }{ \false }$ 4]]
    							]
    							[{$\reft{\Gamma }{ \false }$ \lightning} [$\reft{\Gamma }{ \true }$ 5]]
    					]
    			]
    		]
    	\end{forest}
	}}
	) = (1 2 5, 3, 4)
	\]
\end{figure}

\clearpage

\section{Lean}

\subsection{The Lean Theorem Prover}

Lean is an interactive theorem prover that is based on the calculus of inductive constructions \cite{leanWebsite} \cite{moura15} and is developed by Microsoft Research.
It features dependent types, offers a high degree of automation through tactics and can also be used as a programming language.
Due to the Curry-Howard isomorphism, writing functional definitions intended to be used in proofs,
writing proofs and writing proof-generating custom tactics is very similar.

We use Lean 3 for this thesis and want to give a brief overview of its syntax. See \cite{leanDocs} for a detailed documentation.

Inductive data types can be defined with the keyword $\mathtt{inductive}$. The $\mathtt{\#check}$ instruction can be used to type-check terms:
\begin{minted}{Lean}
inductive my_nat : Type
| zero : my_nat
| succ : my_nat → my_nat

#check my_nat.succ my_nat.zero
#check my_nat.zero.succ -- equivalent term, using dot notation
\end{minted}

The keyword $\mathtt{def}$ can be used to bind terms and define recursive functions:
\begin{minted}{Lean}
def my_nat.add : my_nat → my_nat → my_nat
-- Patterns can be used in definitions
| my_nat.zero b := b
| (my_nat.succ a) b := (a.add b).succ
\end{minted}

Likewise, $\mathtt{def}$ can be used to bind proof terms to propositions.
Propositions are stated as type and proved by constructing a term of that type.
$\Pi$-types are used to introduce generalized type variables:
\begin{minted}{Lean}
-- This type states that for all a, a + zero = a
def my_nat.add_zero_eq : Π a: my_nat, a.add my_nat.zero = a :=
    -- Proof by induction
    @my_nat.rec
        -- Induction Hypothesis
        (λ a, a.add my_nat.zero = a)
        -- Case Zero
        (my_nat.add.equations._eqn_1 my_nat.zero)
        -- Case Succ
        (λ a h,
            @eq.subst my_nat
                (λ x, (my_nat.succ a).add my_nat.zero = x.succ)
                (a.add my_nat.zero)
                a
                h
                (my_nat.add.equations._eqn_2 a my_nat.zero)
        )
\end{minted}

Proofs are usually much shorter when using Leans tactic mode.
Also, definitions can be parametrized (which generalizes the parameter) and the keywords $\mathtt{lemma}$ and $\mathtt{theorem}$ can be used instead of $\mathtt{def}$:
\begin{minted}{Lean}
lemma my_nat.add_zero' (a: my_nat): a.add my_nat.zero = a :=
begin
    induction a,
    { refl, },
    { simp [my_nat.add, *], },
end
\end{minted}

\subsection{The Lean Mathematical Library}

\textit{Mathlib} \cite{mathlibOverview} is a community project that offers a rich mathematical foundation for many theories in Lean 3.
Its theories of finite sets, lists, boolean logic and permutations have been very useful for this thesis.

Mathlib also offers many advanced tactics like $\mathtt{finish}$, $\mathtt{tauto}$ or $\mathtt{linarith}$.
These tactics help significantly in proving trivial lemmas.

\chapter{Formal Definitions}\label{sec:formalization}

Before any property of LYG can be proven or even stated in Lean, all relevant definitions must be formalized.
Since nothing can be left vague in Lean, a lof of decisions had to be made to back up LYG by a fully defined model. 
This chapter discusses these decisions.

\section{Abstracting LYG: The Guard Module}
LYG does not specify an exact guard or expression syntax.
Instead, the notation ``$...$'' is often used to indicate a sensible continuation to make guards powerful enough to model all Haskell constructs.
This is rather problematic for a precise formalization and presented the first big challenge of this thesis.
As we wanted to avoid formalizing Haskell and its semantics, we had to carefully design an abstraction that is as close as possible to LYG
while pinning down guards to a closed but extendable theory.

\subsubsection{The Result Monad}

First, we defined a generic $\mathtt{Result}$ monad to capture the result of an evaluation.
Due to laziness, evaluation of guard trees can either end with a specific right hand side, not match any guard or diverge:

\begin{minted}{Lean}
inductive Result (α: Type)
| value: α → Result
| diverged: Result
| no_match: Result
\end{minted}

A $\mathtt{bind}$ operation can be easily defined on $\mathtt{Result}$ to make it a proper monad with $\mathtt{Result.value}$ as unit function:

\begin{minted}{Lean}
def Result.bind { α β: Type } (f: α → Result β): Result α → Result β
| (Result.value val) := f val
| Result.diverged := Result.diverged
| Result.no_match := Result.no_match
\end{minted}

\subsubsection{Denotational Semantics for Guards}

For some abstract environment type $\mathtt{Env}$, we would like to have a denotational semantics $\mathtt{Grd.eval}$ for guards $\mathtt{Grd}$:
\begin{minted}{Lean}
Grd.eval : Grd → Env → Result Env
\end{minted}

Abstracting $\mathtt{Grd.eval}$ would unify all guard constructs available in Haskell and those used by LYG.
However, LYG needs to recognize all guards that can lead to a diverged evaluation:
Removing all RHSs behind such a guard would inevitably remove the guard itself.
As this might change the semantics of the guard tree, LYG cannot mark all such RHSs as redundant unless there is a proof that the guard will never diverge.
As a consequence, $\mathtt{Grd.eval}$ cannot be abstracted away.

Instead, we explicitly distinguish between non-diverging (\textit{total}) $\mathtt{tgrd}$s and possibly diverging $\mathtt{bang}$ guards:
\begin{minted}{Lean}
inductive Grd
| tgrd (tgrd: TGrd)
| bang (var: Var)
\end{minted}

While $\mathtt{TGrd}$s classically represent guards and $\mathtt{Grd}$s represent guards with side effects (introduced by $\mathtt{bang}$ guards) in this context,
we decided to follow the naming conventions of LYG and chose the name $\mathtt{TGrd}$ for side-effect free (non-diverging) guards
rather than renaming $\mathtt{Grd}$.

In order to define a denotational semantics on $\mathtt{Grd}$, we postulated the functions $\mathtt{tgrd\_eval}: \mathtt{TGrd} \to \mathtt{Env} \to \mathtt{option}\;\mathtt{Env}$ and $\mathtt{is\_bottom}: \mathtt{Var} \to \mathtt{Env} \to \mathtt{bool}$ as well as a type $\mathtt{Var}$ that represents variables. While $\mathtt{tgrd}$s can change the environment,
$\mathtt{bang}$ guards cannot:

\begin{minted}{Lean}
def Grd.eval : Grd → Env → Result Env
| (Grd.tgrd grd) env :=
    match tgrd_eval grd env with
    | none := Result.no_match
    | some env' := Result.value env'
    end
| (Grd.bang var) env :=
    if is_bottom var env
    then Result.diverged
    else Result.value env
\end{minted}

Alternatively,
we could have set $\mathtt{Var} := \mathtt{Env} → \mathtt{bool}$ and $\mathtt{TGrd} := \mathtt{Env} → \mathtt{option}\;\mathtt{Env}$ and replaced $\mathtt{is\_bottom}$ and $\mathtt{tgrd\_eval}$ with $\mathtt{id}$,
yielding the following definition:
\begin{minted}{Lean}
inductive Grd'
| tgrd (grd: Env → option Env)
| bang (test: Env → bool)
\end{minted}
However, this could make the set of guard trees and refinement types uncountable.
While this is not problematic for aspects explored by this thesis,
it could make implementing a correct function $\mathcal{G}$ impossible, as it cannot reason anymore about guards in a computable way if $\mathtt{Env}$ is instantiated with a non-finite type.

\bigskip

\subsubsection{The Guard Module}

In Lean, type classes provide an ideal mechanism to define such ambient abstractions.
They can be opened so that all members of the type class become implicitly available in all definitions and theorems.
Every implicit usage pulls the type class into its signature so that consumers can provide a concrete implementation of the type class.

We defined and opened a type class \textit{GuardModule} that describes the presented abstraction:

\begin{minted}{Lean}

class GuardModule :=
    (Rhs : Type)
    [rhs_decidable: decidable_eq Rhs]
    (Env : Type)
    (TGrd : Type)
    (tgrd_eval : TGrd → Env → option Env)
    (Var : Type)
    (is_bottom : Var → Env → bool)

variable [GuardModule]
open GuardModule
\end{minted}

We also postulated a type $\mathtt{Rhs}$ to refer to right hand sides. For technical reasons, equality on this type must be decidable.
This abstracts from the numbers that are used in LYG to distinguish right hand sides.
We also require most types to be inhabited so that we can construct module-independent examples.

All following definitions and theorems implicitly make use of this abstraction.

\section{Guard Trees}\label{chap:formalGuardTrees}

\subsubsection{Syntax of Guard Trees}

With the definition of $\mathtt{Grd}$, guard trees are defined as inductive data type:

\begin{minted}{Lean}
inductive Gdt
| rhs (rhs: Rhs)
| branch (tr1: Gdt) (tr2: Gdt)
| grd (grd: Grd) (tr: Gdt)
\end{minted}

\subsubsection{Semantics of Guard Trees}

$\mathtt{Gdt.eval}$ defines a denotational semantics on guard trees, using the semantics of guards.
It returns the first RHS that matches a given environment. If a guard diverges, the entire evaluation diverges. Otherwise, if no RHSs matches, \textit{no-match} is returned.

\begin{minted}{Lean}
def Gdt.eval : Gdt → Env → Result Rhs
| (Gdt.rhs rhs) env := Result.value rhs
| (Gdt.branch tr1 tr2) env :=
    match tr1.eval env with
    | Result.no_match := tr2.eval env
    | r := r
    end
| (Gdt.grd grd tr) env := (grd.eval env).bind tr.eval
\end{minted}

\subsubsection{RHSs in Guard Trees}

Every guard tree contains a (non-empty) finite set of right hand sides:
\begin{minted}{Lean}
def Gdt.rhss: Gdt → finset Rhs
| (Gdt.rhs rhs) := { rhs }
| (Gdt.branch tr1 tr2) := tr1.rhss ∪ tr2.rhss
| (Gdt.grd grd tr) := tr.rhss
\end{minted}

In LYG, it is implicitly assumed that the right hand sides of a guard tree are numbered unambiguously.
This has to be stated explicitly in Lean with the following recursive predicate:

\begin{minted}{Lean}
def Gdt.disjoint_rhss: Gdt → Prop
| (Gdt.rhs rhs) := true
| (Gdt.branch tr1 tr2) :=
        disjoint tr1.rhss tr2.rhss
        ∧ tr1.disjoint_rhss ∧ tr2.disjoint_rhss
| (Gdt.grd grd tr) := tr.disjoint_rhss
\end{minted}

\subsubsection{Removing RHSs in Guard Trees}

$\mathtt{Gdt.remove\_rhss}$ defines how a set of RHSs can be removed from a guard tree.
This definition is required to state that all redundant RHSs can be removed without changing semantics.
Note that the resulting guard tree might be empty when all RHSs are removed!

\begin{minted}{Lean}
def Gdt.branch_option : option Gdt → option Gdt → option Gdt
| (some tr1) (some tr2) := some (Gdt.branch tr1 tr2)
| (some tr1) none := some tr1
| none (some tr2) := some tr2
| none none := none

def Gdt.grd_option : Grd → option Gdt → option Gdt
| grd (some tr) := some (Gdt.grd grd tr)
| _ none := none

def Gdt.remove_rhss : finset Rhs → Gdt → option Gdt
| rhss (Gdt.rhs rhs) := if rhs ∈ rhss then none else some (Gdt.rhs rhs)
| rhss (Gdt.branch tr1 tr2) :=
    Gdt.branch_option
        (tr1.remove_rhss rhss)
        (tr2.remove_rhss rhss)
| rhss (Gdt.grd grd tr) := Gdt.grd_option grd (tr.remove_rhss rhss)
\end{minted}

Finally, to deal with the semantics of empty guard trees,
$\mathtt{Gdt.eval\_option}$ lifts $\mathtt{Gdt.eval}$ to $\mathtt{option\;Gdt}$:

\begin{minted}{Lean}
def Gdt.eval_option : option Gdt → Env → Result
| (some gdt) env := gdt.eval env
| none env := Result.no_match
\end{minted}

\section{Refinement Types}
\label{sec:formalizationRefinementTypes}

Refinement types presented another challenge.
Defining refinement types through a proper type system would have required to model Haskell types.
Instead, we tried to rely on the same abstractions used to define guard trees in hope that guard trees and refinement types can be related.

In this formalization, a refinement type $\Phi$ denotes a predicate on environments:

\begin{minted}{Lean}
def Φ.eval: Φ → Env → bool
\end{minted}

With a proper $\mathtt{GuardModule}$ instantiation, the environment can be used to not only carry runtime values, but also their type!
A (well) typed environment can assist in proving a refinement type to be empty.

\subsubsection{Variable Scoping Rules / Incorrectness of $\mathcal{U}$}
\label{sec:formalizationVariableBindingRules}

Another problem that had to be solved was the formalization of the unconventional binding mechanism of refinement types through conjunctions, as described in chapter \ref{chap:bckgrndRefinementTypesBinding}.
In particular, this causes $\mathcal{U}$ to be incorrect (for some intuitive notion of correctness)
with regards to the guard tree semantics we defined in chapter \ref{chap:formalGuardTrees}.
While the following guard tree $gdt$ does not match for $x = \mathtt{False}$, its uncovered refinement type $\Theta$ computed by $\mathcal{U}$ is empty due to the constraints $x \ntermeq \Conid{False}$ and $x \ntermeq \Conid{True}$ that refer to the same variable and thus represent a contradiction!

\[
    \mathit{gdt} :=
    \vcenter{\hbox{
    \begin{forest}
    	grdtree
    	[
    		[{$\grdlet{x}{\mathtt{True}},\, \grdcon{\mathtt{False}}{x}$} [1]]
    		[{$\grdcon{\mathtt{True}}{x}$} [2]]
    	]
    \end{forest}
    }}
\]
\begin{align*}
    \Theta := \mathcal{U}(\true, \mathit{gdt}) =& \reft{ x{:}\ensuremath{\Conid{Bool}} }{
        ((\grdlet{x}{\Conid{True}} \andtheta
        x \ntermeq \Conid{False}) \andtheta x \ntermeq \Conid{True}
    } \\
    =& \reft{ x{:}\ensuremath{\Conid{Bool}} }{
        \grdlet{x}{\Conid{True}} \land
        (x \ntermeq \Conid{False} \land x \ntermeq \Conid{True})
    }
\end{align*}

In the example, the let binding brings a variable $x$ into scope that shadows an outer variable.
Due to the definition of $\mathcal{U}$ and the scoping rules of refinement types,
this shadowing binding of $x$ overrides the outer variable $x$ in contexts where it is incorrect to do so.

In particular, the term $\unc(\unc(\Theta, t_1), t_2)$ in the branch case of $\mathcal{U}$ is problematic:
All bindings introduced in $\Theta$ should still be exposed by $\unc(\Theta, t_1)$ so that variables in $t_2$ are resolved correctly.
However, variable definitions introduced in $t_1$ must not be visible to $t_2$ and thus must not be exposed by $\unc(\Theta, t_1)$!
This is clearly violated by the term $\unc(\Theta \andtheta (\ctlet{x}{e}), t)$ that defines the let-case in $\mathcal{U}$ - it exposes both all bindings from $\Theta$ and the new binding $\ctlet{x}{e}$.

Shadowing is unproblematic for the presented semantics of guard trees though: If the first guard tree of a branch fails to match, its environment just before the failing guard is discarded and with it possible shadowing bindings.
The second branch is always evaluated with the same environment that the first guard tree has been evaluated with.
This is consistent with Haskells semantics of pattern match expressions.

There are several ways of how this problem can be addressed.
\begin{itemize}
    \item Replace the term $\unc(\unc(\Theta, t_1), t_2)$ in the definition of $\mathcal{U}$ with $\unc((\unc(\Theta, t_1) \cup \false) \andtheta \Theta, t_2)$.
    
        ``$\andtheta$'' stops at ``$\cup$'', so the $\cup$-operator acts as scope boundary.
        To bring the variables defined by $\Theta$ into scope again, $\Theta$ is joined a second time, potentially causing a refinement type of exponential size.
        While we believe that $\mathcal{U}$ is correct with this updated definition,
        we decided against this solution as the construction to limit the scope feels like a band aid and is unnecessarily complex.
    
    \item Adjust the semantics of guard trees so that variables defined in a branch override shadowed variables in all later branches.
    
        We managed to prove correctness of $\mathcal{U}$ as stated in LYG with this updated semantics of guard trees.
        However, this semantics is not only very unconventional, but also dramatically increases the complexity
        when reasoning about the effect of removing an inaccessible RHS.
        
        Since variable bindings introduced by guards that guard only inaccessible RHSs stay visible
        until the evaluation ends (and are not only relevant for the inaccessible RHSs),
        removing such guards almost always causes a different final environment.
        In this sense, almost no inaccessible RHSs are redundant - which is not the intention of the analysis and clearly not the case for the GHC implementation of LYG. To make the analysis more meaningful, we could require each variable name to be unique.
        With this assumption, such environment modification should have no impact on the evaluation result.
        Due to the high complexity of this approach, we decided against it too.
    
    \item Limit the scope of variables in refinement types.
    
        A data constructor $\mathtt{Φ.tgrd\_in}:
        \mathtt{TGrd} \to Φ \to Φ$ is introduced that limits the scope of the guard to the nested refinement type and any scoping behavior of the $\land$-operator is removed.
        This simplifies the scoping mechanism, but requires to adapt $\mathcal{U}$, as done in chapter \ref{sec:formalizationUncoveredAnalysis}.
        We chose this approach due to its clear modeling, in hope to reduce the complexity of the formal proofs.
        
\end{itemize}

This problem does not arise in the GHC implementation of LYG as it uses a different encoding for refinement types.

\subsubsection{Syntax of Refinement Types}

Finally, this is our formalized syntax of refinement types:

\begin{minted}{Lean}
inductive Φ
| false
| true
| tgrd_in (tgrd: TGrd) (ty: Φ)
| not_tgrd (tgrd: TGrd)
| var_is_bottom (var: Var)
| var_is_not_bottom (var: Var)
| or (ty1: Φ) (ty2: Φ)
| and (ty1: Φ) (ty2: Φ)
\end{minted}

Since the negation of a guard cannot bind variables,
it does not need to have a nested refinement type that would see bound variables.
The same applies to $\mathtt{var\_is\_bottom}$ and its negation.

\subsubsection{Semantics of Refinement Types}

The semantics of refinement types is easily defined and implicitly uses the guard module:

\begin{minted}{Lean}
def Φ.eval: Φ → Env → bool
| Φ.false env := ff
| Φ.true env := tt
| (Φ.tgrd_in grd ty) env := match tgrd_eval grd env with
    | some env := ty.eval env
    | none := ff
    end
| (Φ.not_tgrd grd) env :=
    match tgrd_eval grd env with
    | some env := ff
    | none := tt
    end
| (Φ.var_is_bottom var) env := is_bottom var env
| (Φ.var_is_not_bottom var) env := !is_bottom var env
| (Φ.or t1 t2) env := t1.eval env || t2.eval env
| (Φ.and t1 t2) env := t1.eval env && t2.eval env
\end{minted}

With this definition the evaluation of the second operand of a conjunction is obviously independent
of any environment effects applied in the evaluation of the first operand!

\subsubsection{Definition of \texttt{is\_empty}}

A refinement type $\Phi$ is called \textit{empty} if it does not match any environment.
This is formalized by the predicate $\Phi\mathtt{.is\_empty}$:

\begin{minted}{Lean}
def Φ.is_empty (ty: Φ): Prop := ∀ env: Env, ¬(ty.eval env)
\end{minted}

\subsubsection{Definition of \texttt{can\_prove\_empty}}

Instead of a partial function $\generate$ with $\generate(Φ) = \varnothing$ if and only if $Φ$ is empty,
we define a total function $\mathtt{can\_prove\_empty}$ and a predicate $\mathtt{correct\_can\_prove\_empty}$ that ensures
its correctness. This abstracts from the generation of inhabitants which are superfluous in this context. It also avoids dealing with partial functions, which are not directly supported by Lean.
\begin{minted}{Lean}
variable can_prove_empty: Φ → bool
def correct_can_prove_empty : Prop :=
    ∀ ty: Φ, can_prove_empty ty = tt → ty.is_empty
\end{minted}

The subtype $\mathtt{CorrectCanProveEmpty}$ bundles a correct $\mathtt{can\_prove\_empty}$ function:
\begin{minted}{Lean}
def CorrectCanProveEmpty := {
    can_prove_empty : Φ → bool
    // correct_can_prove_empty can_prove_empty
}
\end{minted}

\newpage

\section{Uncovered Analysis}
\label{sec:formalizationUncoveredAnalysis}

As discussed in chapter \ref{sec:formalizationRefinementTypes}, LYG's definition of
$\mathcal{U}$ has problems with guard trees that define shadowing bindings.
LYG defined $\mathcal{U}$ as follows (see chapter \ref{chap:bckgrdUncoveredAnalysis} for the discussion of this definition):

\[
\begin{array}{lcl}
	\unc(\reft{\Gamma}{\Phi}, \gdtrhs{n})                                    & = & \reft{\Gamma}{\false}                                                                                                 \\
	\unc(\Theta, \gdtseq{t_1}{t_2})                                          & = & \unc(\unc(\Theta, t_1), t_2)                                                                                          \\
	\unc(\Theta, \gdtguard{\grdbang{x}}{t})                                  & = & \unc(\Theta \andtheta (x \ntermeq \bot), t)                                                                           \\
	\unc(\Theta, \gdtguard{\grdlet{x}{e}}{t})                                & = & \unc(\Theta \andtheta (\ctlet{x}{e}), t)                                                                              \\
	\unc(\Theta, \gdtguard{\grdcon{\genconapp{K}{a}{\gamma}{y:\tau}}{x}}{t}) & = & \Theta \andtheta (x \ntermeq K) \uniontheta \unc(\Theta \andtheta (\ctcon{\genconapp{K}{a}{\gamma}{y:\tau}}{x}), t) \\
\end{array}
\]

Equipped with the data constructor $\mathtt{Φ.tgrd\_in}$,
we can fix the shadowing problem and formalize $\mathcal{U}$ now.
Instead of using $\Phi$ as accumulator type, our formalization uses a function $\Phi \to \Phi$:
The new accumulator explicitly applies a context to a refinement type.
This happens implicitly in LYG's definition through the use of $\Theta \andtheta \cdot$.

\overfullrule=0pt
Note that all occuring accumulator functions are homomorphisms modulo the semantics of refinement types, i.e. $\mathtt{Gdt.eval}\;(f\;(a.\mathit{and}\;b)) = \mathtt{Gdt.eval}\;((f\;a).\mathit{and}\;(f\;b))$.
We carefully make use of this to get formalized definitions of $\mathcal{U}$ and $\mathcal{A}$ that can be interleaved, as done in LYG.

\begin{minted}{Lean}
def ω_acc : (Φ → Φ) → Gdt → Φ
| acc (Gdt.rhs _) := Φ.false
| acc (Gdt.branch tr1 tr2) := (ω_acc ((ω_acc acc tr1).and ∘ acc) tr2)
| acc (Gdt.grd (Grd.bang var) tr) :=
    ω_acc (acc ∘ (Φ.var_is_not_bottom var).and) tr
| acc (Gdt.grd (Grd.tgrd grd) tr) :=
            (acc (Φ.not_tgrd grd))
        .or (ω_acc (acc ∘ (Φ.tgrd_in grd)) tr)

def ω : Gdt → Φ := ω_acc id
\end{minted}

\newpage
\section{Redundant / Inaccessible Analysis}

\subsubsection{Formalization of Annotated Trees}

The formalization of annotated trees is straightforward.
However, we allow arbitrary annotations rather than only accepting refinement types.
This will become useful in formal proofs when we no longer care about
the specific refinement types, but only whether they are empty.

\begin{minted}{Lean}
inductive Ant (α: Type)
| rhs (a: α) (rhs: Rhs): Ant
| branch (tr1: Ant) (tr2: Ant): Ant
| diverge (a: α) (tr: Ant): Ant
\end{minted}

\subsubsection{Formalization of $\mathcal{A}$}

Similar to the formalization of $\mathcal{U}$ in chapter \ref{sec:formalizationUncoveredAnalysis}, we also need to address the shadowing problem when formalizing $\mathcal{A}$.
This is LYG's definition of $\mathcal{A}$ as stated in chapter \ref{chap:backgrdRedundantInaccAnalysis}:

\[
	\begin{array}{lcl}
		\ann(\Theta,\gdtrhs{n})                                                  & = & \antrhs{\Theta}{n}                                                                       \\
		\ann(\Theta, \gdtseq{t_1}{t_2})                                          & = & \antseq{\ann(\Theta, t_1)}{\ann(\unc(\Theta, t_1), t_2)}                                 \\
		\ann(\Theta, \gdtguard{\grdbang{x}}{t})                                  & = & \antbang{\Theta \andtheta (x \termeq \bot)}{\ann(\Theta \andtheta (x \ntermeq \bot), t)} \\
		\ann(\Theta, \gdtguard{\grdlet{x}{e}}{t})                                & = & \ann(\Theta \andtheta (\ctlet{x}{e}), t)                                                 \\
		\ann(\Theta, \gdtguard{\grdcon{\genconapp{K}{a}{\gamma}{y:\tau}}{x}}{t}) & = & \ann(\Theta \andtheta (\ctcon{\genconapp{K}{a}{\gamma}{y:\tau}}{x}), t)                  \\
	\end{array}
\]

Our formalization in Lean follows. Analogous to our formalization of $\mathcal{U}$, instead of contextualizing refinement types by
combining them with the accumulator through $\andtheta$, we model the accumulator as an explicit function
that contextualizes its argument:

\begin{minted}{Lean}
def π_acc : (Φ → Φ) → Gdt → Ant Φ
| acc (Gdt.rhs rhs) := Ant.rhs (acc Φ.true) rhs
| acc (Gdt.branch tr1 tr2) :=
    Ant.branch
        (π_acc acc tr1)
        (π_acc ((ω_acc acc tr1).and ∘ acc) tr2)
| acc (Gdt.grd (Grd.bang var) tr) :=
    Ant.diverge
        (acc (Φ.var_is_bottom var)) 
        (π_acc (acc ∘ ((Φ.var_is_not_bottom var).and)) tr)
| acc (Gdt.grd (Grd.tgrd grd) tr) :=
    (π_acc (acc ∘ (Φ.tgrd_in grd)) tr)

def π : Gdt → Ant Φ := π_acc id
\end{minted}

Note that in the branch case, $\mathcal{A}\mathtt{\_acc}$ and $\mathcal{U}\mathtt{\_acc}$
are called with the same arguments. Even more, both functions have the same recursion structure, which
makes it possible to interleave both functions.
This is done in chapter \ref{sec:formalizationInterleaving}.


%-- returns (accessible, inaccessible, redundant) rhss, given that `can_prove_empty` is correct.

\subsubsection{Formalization of $\mathcal{R}$}

It remains to formalize the function $\mathcal{R}$ that partitions
all right hand sides of an annotated guard tree into accessible, inaccessible and redundant right hand sides,
by using the function $\mathtt{can\_prove\_empty}$.

This is $\mathcal{R}$ as presented in LYG and chapter \ref{chap:backgrdRedundantInaccAnalysis}:
\[
	\begin{array}{lcl}
		\red(\antrhs{\Theta}{n})  & = & \begin{cases}
			(\epsilon, \epsilon, n), & \text{if $\generate(\Theta) = \emptyset$} \\
			(n, \epsilon, \epsilon), & \text{otherwise}                          \\
		\end{cases}                                                                                                                     \\
		\red(\antseq{t}{u})       & = & (\overline{k}\,\overline{k'}, \overline{n}\,\overline{n'}, \overline{m}\,\overline{m'}) \hspace{0.5em} \text{where} \begin{array}{l@{\,}c@{\,}l}
			(\overline{k}, \overline{n}, \overline{m})    & = & \red(t) \\
			(\overline{k'}, \overline{n'}, \overline{m'}) & = & \red(u) \\
		\end{array} \\
		\red(\antbang{\Theta}{t}) & = & \begin{cases}
			(\epsilon, m, \overline{m'}), & \text{if $\generate(\Theta) \not= \emptyset$ and $\red(t) = (\epsilon, \epsilon, m\,\overline{m'})$} \\
			\red(t),                      & \text{otherwise}                                                                                     \\
		\end{cases}                                                                                                                     \\
	\end{array}
\]

This definition has a surprisingly direct representation in Lean:
\begin{minted}{Lean}
def ρ : Ant Φ → list Rhs × list Rhs × list Rhs
| (Ant.rhs ty n) :=
    if can_prove_empty ty
    then ([], [], [n])
    else ([n], [], [])
| (Ant.branch tr1 tr2) :=
    match (ρ tr1, ρ tr2) with
    | ((k, n, m), (k', n', m')) := (k ++ k', n ++ n', m ++ m')
    end 
| (Ant.diverge ty tr) :=
    match ρ tr, can_prove_empty ty with
    | ([], [], m :: ms), ff := ([], [m], ms)
    | r, _ := r
    end
\end{minted}

\section{Interleaving $\mathcal{U}$ and $\mathcal{A}$}
\label{sec:formalizationInterleaving}

Since $\mathcal{A}\mathtt{\_acc}$ and $\mathcal{U}\mathtt{\_acc}$ have
the same recursion structure, they can be combined into a single function that shares the recursive invocations.
The following function $\mathcal{UA}\mathtt{\_acc}$ computes the uncovered refinement type and the annotated guard tree for a given guard tree at the same time.
This improves performance if a lazy evaluation strategy is used in combination with sharing as the accumulator can be fully shared.

\begin{minted}{Lean}
def ωπ_acc : (Φ → Φ) → Gdt → Φ × Ant Φ
| acc (Gdt.rhs rhs) := (Φ.false, Ant.rhs (acc Φ.true) rhs)
| acc (Gdt.branch tr1 tr2) :=
    let (U1, A1) := ωπ_acc acc tr1,
        (U2, A2) := ωπ_acc (U1.and ∘ acc) tr2
    in  (U2, Ant.branch A1 A2)
| acc (Gdt.grd (Grd.bang var) tr) :=
    let (U, A) := ωπ_acc (acc ∘ (Φ.var_is_not_bottom var).and) tr
    in (U, Ant.diverge (acc (Φ.var_is_bottom var)) A)
| acc (Gdt.grd (Grd.tgrd grd) tr) := 
    let (U, A) := ωπ_acc (acc ∘ (Φ.tgrd_in grd)) tr
    in ((acc (Φ.not_tgrd grd)).or U, A)
\end{minted}

It is surprisingly easy to show that this function is really interleaving $\mathcal{A}\mathtt{\_acc}$ and $\mathcal{U}\mathtt{\_acc}$:

\begin{minted}{Lean}
theorem ωπ_acc_eq (acc: Φ → Φ) (gdt: Gdt):
    ωπ_acc acc gdt = (ω_acc acc gdt, π_acc acc gdt) :=
by induction gdt generalizing acc;
    try { cases gdt_grd }; simp [ωπ_acc, ω_acc, π_acc, *]
\end{minted}


\chapter{Correctness Statements}
\label{sec:formalizationCorrectnessStmts}

As we have all the required definitions at this point, we can state and formalize what we expect of the presented pattern match analyses to be considered correct.
We provide proofs for all correctness propositions on GitHub \cite{leanProof}. Chapter \ref{sec:proof} will discuss parts of these proofs in more detail.

\section{Correctness of the Uncovered Analysis}
\label{sec:formalizationSemanticU}

$\unc$ should compute a refinement type that denotes exactly all values that are not covered by a given guard tree.
This does not include values under which the execution diverges!

The following theorem states correctness of $\mathcal{U}$ in Lean:

\begin{minted}{Lean}
theorem ω_semantic: ∀ gdt: Gdt, ∀ env: Env,
        (ω gdt).eval env ↔ (gdt.eval env = Result.no_match)
\end{minted}

As an obvious consequence, a guard tree always matches (or diverges) if and only if the refinement type computed by $\mathcal{U}$ is empty.
If a correct function $\mathcal{G}$ or $\mathtt{can\_prove\_empty}$ proves emptiness of such a computed refinement type, there are no uncovered cases by this theorem. Otherwise, a warning of potential uncovered cases should be issued!

Hence, this theorem implies correctness of the uncovered analysis:
The uncovered analysis should rather report a false positive than not detect an uncovered case.

Note that this theorem carries over to all semantically equivalent definitions of $\mathcal{U}$.

\subsection{Comparison to LYGs Notion Of Correctness}

LYG states that ``[...] LYG will never fail to report uncovered clauses (no false negatives), but it may report false positives'' \cite{10.1145/3408989}.
Our statement of $\mathcal{U}$s correctness is stronger:
The function $\mathcal{U}$ computes a refinement type that covers exactly all environments that are not covered by the guard tree. If $\mathcal{G}$ is assumed to be correct and used to semi-decide whether the refinement type computed by $\mathcal{U}$ is empty,
LYGs claim follows.

\section{Correctness of the Redundant/Inaccessible Analysis}
\label{sec:formalizationSemanticRA}

For a given guard tree and a given correct function $\mathtt{can\_prove\_empty}$ (which corresponds to $\generate$ in LYG), $\red$
should compute a triple $(a, i, r)$ of accessible, inaccessible and redundant right hand sides.
Whenever the given guard tree evaluates to a RHS, this RHS must be accessible and neither inaccessible nor redundant.
RHSs that are redundant can be removed from the guard tree without changing the semantics of the guard tree.
This expresses correctness of the redundant and inaccessible analysis.

\begin{minted}{Lean}
theorem ρ_semantic:
    ∀ can_prove_empty: CorrectCanProveEmpty,
    ∀ gdt: Gdt, gdt.disjoint_rhss → (
        let ⟨ a, i, r ⟩ := ρ can_prove_empty.val (π gdt)
        in
                (∀ env: Env, ∀ rhs: Rhs,
                    gdt.eval env = Result.value rhs
                      → rhs ∈ a \ (i ++ r)
                )
            ∧
                Gdt.eval_option (gdt.remove_rhss r.to_finset)
                = gdt.eval

        : Prop
    )
\end{minted}

Note that redundant RHSs could be marked as inaccessible or even accessible instead without violating this theorem.
The opposite is not true: Not all accessible RHSs can be marked as inaccessible and not all inaccessible RHSs can be marked as redundant - see chapters \ref{sec:intro} and \ref{sec:background} for counterexamples. However, we conjecture that $a$ contains no inaccessible and $i$ no redundant RHSs if $\mathtt{can\_prove\_empty}$ is both correct and complete (even though such a function is usually uncomputable).

\subsection{Comparison to LYGs Notion Of Correctness}

LYG states correctness of the redundant/inaccessible analysis as following:
``Similarly, LYG will never report accessible clauses as
redundant (no false positives), but it may fail to report clauses which are redundant when the code
involved is too close to undecidable territory.'' \cite{10.1145/3408989}.
Furthermore, LYG also states
``A redundant equation can be removed from a
function without changing its semantics, whereas an inaccessible equation cannot, [...]''.

We both improved the precision of LYGs notion of correctness by formally defining every involved concept,
but also made it more complete by stating that RHSs identified as redundant by LYG are indeed redundant.

While the predecessor of LYG, ``GADTs Meet Their Match'' \cite{10.1145/2858949.2784748} (in short \textit{GMTM}),
gives a formal statement about its correctness in theorem 1, it lacks a proof.
Also, according to our understanding,
GMTM's statement does not explicitly examine the effect of removing redundant right hand sides as we do.
\chapter{Formalized Proof}\label{sec:proof}

\section{Overview}

\section{Definitions}

\section{R is not monotonous}

\section{R is not monotonous}

\section{Definition of is\_reduntant\_set}

\section{Main Proof}

\section{Alternative Proof Ideas}

\chapter{Conclusion}\label{sec:conclusion}

We refined and formalized several correctness properties of LYG and successfully proved them in Lean.
However, we parametrized these correctness properties over a correct function $\mathcal{G}$ that semi-decides emptiness of refinement types.
While LYG defines such a function, we did not prove that it indeed is such a correct function $\mathcal{G}$.

Even though these correctness properties look seemingly easy to prove, it turned out to be a very involved undertaking.
After all, it took us 48 definitions and 143 lemmas and theorems to formalize these proofs in Lean!

We believe that this complexity is caused by the amount of details required to describe LYG and the rigorousness of Lean.
In fact, we discovered a minor flaw in LYG's definition of $\mathcal{U}$, buried in the details of the let binding semantics.
Luckily, this flaw has no impact on the GHC implementation of LYG, as the implementation uses a different encoding of refinement types.
Still, this flaw was not discovered in peer reviews of the LYG paper,
showing that LYG's correctness is not obvious at all and making a strong point for verification, yet formal verification.

Finally, as our proofs are formally verified by Lean, it is highly unlikely that LYG has any other flaws,
except in the definition of the presented function $\mathcal{G}$ that we did not check.

We can strongly recommend to use Lean for formal verification!

\bibliographystyle{ieeetr}
\bibliography{thesis/bib}

\begin{otherlanguage}{ngerman}
	\chapter*{Erklärung}
	\pagestyle{empty}

	\vspace{20mm}
	Hiermit erkläre ich, \theauthor, dass ich die vorliegende Masterarbeit selbst\-ständig
	verfasst habe und keine anderen als die angegebenen Quellen und Hilfsmittel
	benutzt habe, die wörtlich oder inhaltlich übernommenen Stellen als solche kenntlich gemacht und
	die Satzung des KIT zur Sicherung guter wissenschaftlicher Praxis beachtet habe.
	\vspace{20mm}
	\begin{tabbing}
		\rule{7cm}{.4pt}\hspace{1cm} \= \rule{6.8cm}{.4pt} \\
		Ort, Datum \> Unterschrift
	\end{tabbing}
\end{otherlanguage}

\chapter*{Danke}
\pagestyle{empty}

Ich danke meinen Betreuern Sebastian Graf und Sebastian Ullrich, die mich in jeglicher Hinsicht unterstützt haben.
Außerdem will ich mich bei der Lean Community bedanken, die mir bei Fragen zu Lean viel geholfen hat.

\pagestyle{fancy}

\end{document}
