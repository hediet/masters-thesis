\documentclass[parskip=no,12pt,a4paper,twoside,headings=openright]{scrreprt}
% switch to scrbook if you want roman page numbers for the front matter
% however scrbook has no 'abstract' environment!
% if your thesis is in english, use "parskip=no" instead

% binding correction (BCOR) von 1cm für Leimbindung
\KOMAoptions{BCOR=1cm}
\KOMAoptions{draft=yes}

\usepackage[utf8]{inputenc} % encoding of sources
\usepackage[T1]{fontenc}
\usepackage{./thesis/style/studarbeit}
\usepackage{listings}
\usepackage[edges]{forest}
\usepackage{xspace}
\usepackage{mathtools}
\usepackage{amssymb}
\usepackage{ifsym}
\usepackage{wasysym}
\usepackage{minted}
\usepackage{fontspec}
\usepackage{unicode-math}
\usepackage{mathdesign}
\usepackage{svg}

% https://tex.stackexchange.com/questions/343494/minted-red-box-around-greek-characters/343506#343506
\makeatletter
\AtBeginEnvironment{minted}{\dontdofcolorbox}
\def\dontdofcolorbox{\renewcommand\fcolorbox[4][]{##4}}
\makeatother

% https://leanprover.github.io/lean4/doc/syntax_highlight_in_latex.html
\usepackage{newunicodechar}
\newfontfamily{\freeserif}{DejaVu Sans}
\newunicodechar{∈}{\freeserif{\smallin}}
\newunicodechar{∧}{\freeserif{\land}}
\newunicodechar{∨}{\freeserif{\lor}}
\newunicodechar{∪}{\freeserif{\cup}}
\newunicodechar{∘}{\freeserif{\circ}}
\newunicodechar{ω}{\ensuremath{\mathcal{U}}}
\newunicodechar{ρ}{\ensuremath{\mathcal{R}}}
\newunicodechar{π}{\ensuremath{\mathcal{A}}}
\newunicodechar{∀}{\ensuremath{\forall}}
\newunicodechar{↔}{\ensuremath{\leftrightarrow}}
\newunicodechar{⟨}{\ensuremath{\langle}}
\newunicodechar{⟩}{\ensuremath{\rangle}}
\newunicodechar{∅}{\ensuremath{\varnothing}}
\newunicodechar{∉}{\ensuremath{\not\in}}
\newunicodechar{∩}{\ensuremath{\cap}}
\newunicodechar{⊆}{\ensuremath{\subseteq}}
\newunicodechar{∃}{\ensuremath{\exists}}
\newunicodechar{≠}{\ensuremath{\neq}}
\newunicodechar{ℕ}{\ensuremath{\mathbb{N}}}
\newunicodechar{Γ}{\ensuremath{\Rightarrow}}


\defaultfontfeatures{Scale = MatchLowercase}
\setmainfont{CMU Serif}[Scale = 1.0]
\setsansfont{CMU Sans Serif}
\setmonofont{CMU Typewriter Text}
\setmathfont{Latin Modern Math}

\title{Formal Verification of Pattern Matching Analyses}
\author{Henning Dieterichs}
\thesistype{Masterarbeit}
\zweitgutachter{Prof.~Dr.~rer.~nat.~Bernhard~Beckert}
\betreuer{M.~Sc.~Sebastian~Graf}
\coverimage{thesis/cover.png}
\abgabedatum{{\year=2021 \month=3 \day=7 \today}}


\begin{document}

\begin{otherlanguage}{ngerman} % Titelseite ist immer auf Deutsch
	\mytitlepage
\end{otherlanguage}

\begin{abstract}
	\begin{center}\Huge\textbf{\textsf{Zusammenfassung}}
	\end{center}
	\vfill
    Die in Lower Your Guards vorgestellten Algorithmen analysieren Pattern Matching Definitionen
	und erkennen nicht abgedeckte Fälle, aber auch unzugängliche und redundante rechte Seiten.
	
	Ihre Implementierung in GHC entdeckte erfolgreich bisher unbekannte Fehler in Haskell Quellcode.
    Während die empirische Validierung über einen große Menge von Haskell-Code die Behauptung der Korrektheit untermauert,
    fehlt den Autoren eine präzise Formalisierung sowie ein Beweis für diese Behauptung.
	
	Diese Arbeit etabliert einen präzisen Begriff von Korrektheit und
	präsentiert formale Beweise, dass diese Algorithmen tatsächlich korrekt sind.
	Diese Beweise sind in Lean 3 formalisiert.

	\vfill

	The algorithms presented in Lower Your Guards analyze pattern matching definitions
	and detect uncovered cases, but also inaccessible and redundant right hand sides.
	
	Their implementation in the GHC spotted previously unknown bugs in real world code.
    While empirical validation over a large corpus of Haskell code corroborates the claim of correctness, the authors lack a precise formalization as well as a proof of that claim.
	
	This thesis establishes a precise notion of correctness and
	presents formal proofs that these algorithms are indeed correct.
	These proofs are formalized in Lean 3.
	\vfill

\end{abstract}

\tableofcontents



%%%%%%%%% HERE THE MACROS START
% Highlight changes and keyword undone
\newcommand{\highlight}[1]{\setlength{\fboxsep}{2pt}\colorbox[gray]{0.8}{\ensuremath{#1}}} % less height

\newcommand{\TODOI}[1]{\textcolor{red}{#1}}
\newcommand{\UNDONE}{\begin{color}{red}{\bf UNDONE }\end{color}}
\newcommand{\DONE}{\begin{color}{green}{\bf DONE}\end{color}}
\newcommand{\NOTE}[1]{\bf #1}
\newcommand{\TODO}[1]{{\bf{\begin{color}{red}{TODO: }\end{color} #1}}}
\newcommand{\lyg}{LYG\xspace}
\newcommand{\gmtm}{GMTM\xspace}

\newcommand{\cf}{cf.\@\xspace}
\newcommand{\eg}{e.g.,\@\xspace}
\newcommand{\ie}{i.e.\@\xspace}
\newcommand{\vs}{vs.\@\xspace}
\newcommand{\etc}{etc.\@\xspace}
\newcommand{\keyword}[1]{\textsf{\textbf{#1}}}
\newcommand{\id}[1]{\textsf{\textsl{#1}}\xspace}

% Useful macros that are often needed
\newcommand{\typeeq}{\sim}          % type equality
\newcommand{\termeq}{\approx}       % positive term equality
\newcommand{\ntermeq}{\not\approx}  % negative term equality

\newcommand{\ruleform}[1]{\fbox{$#1$}}
\newcommand{\rulename}[1]{\textsc{[#1]}}
\newcommand{\freein}{\;\#\;}

\newcommand{\ticket}[1]{\href{https://ghc.haskell.org/trac/ghc/ticket/#1}{\##1}}
\newcommand{\extension}[1]{\texttt{#1}}

% \newcommand{\sg}[1]{\begin{color}{red}{\bf SG:} #1\end{color}}
% \newcommand{\simon}[1]{\begin{color}{purple}{\bf SLPJ:} #1\end{color}}
% \newcommand{\ryan}[1]{\begin{color}{orange}{\bf Ryan:} #1\end{color}}
\newcommand{\sg}[1]{}
\newcommand{\simon}[1]{}
\newcommand{\ryan}[1]{}

\newcommand{\Conid}[1]{\mathit{#1}}
\newcommand{\Varid}[1]{\mathit{#1}}

% Types and Grd syntax
\newcommand{\ty}[1]{\textsf{#1}\xspace}
\newcommand{\Pat}{\ty{Pat}}
\newcommand{\Grd}{\ty{Grd}}
\newcommand{\Con}{\ty{Con}}
\newcommand{\Var}{\ty{Var}}
\newcommand{\Expr}{\ty{Expr}}
\newcommand{\Type}{\ty{Type}}
\newcommand{\Kind}{\ty{Kind}}
\newcommand{\TyCt}{\ty{TyCt}}
\newcommand{\NT}{\ty{NT}}
\newcommand{\PS}{\ty{PS}}
\newcommand{\CL}{\ty{CL}}
\newcommand{\grdlet}[2]{\textsf{let}\;#1\,\textsf{=}\,#2}
\newcommand{\grdbang}[1]{\textsf{!}#1}
\newcommand{\grdcon}[2]{#1 \leftarrow #2}
\newcommand{\ctlet}[2]{\textsf{let}\;#1\,\textsf{=}\,#2}
\newcommand{\ctcon}[2]{#1 \leftarrow #2}
\newcommand{\genconapp}[4]{#1\;\overline{#2}\;\overline{#3}\;\overline{#4}}
\newcommand{\expconapp}[4]{#1\;\overline{#2}\;\overline{#3}\;\overline{#4}}
\newcommand{\deltaconapp}[3]{#1\;\overline{#2}\;\overline{#3}}
\newcommand{\ntconapp}[3]{#1\;\overline{#2}\;#3}
\newcommand{\false}{\times}
\newcommand{\true}{\checked}

% GrdTree Gdt
\newcommand{\Gdt}{\ty{Gdt}}
\newcommand{\gdtrhs}[1]{
	\vcenter{\hbox{\begin{forest}
				grdtree,
				[ [{$#1$}] ]
			\end{forest}}}}
\newcommand{\gdtseq}[2]{
	\vcenter{\hbox{\begin{forest}
				grdtree,
				for tree={delay={edge={-}}},
				[ [{$#1$}] [{$#2$}] ]
			\end{forest}}}}
\newcommand{\gdtguard}[2]{
	\vcenter{\hbox{\begin{forest}
				grdtree,
				grhs/.style={tier=rhs,edge={-}},
				[ [{$#1$} [{$#2$}] ] ]
			\end{forest}}}}
\newcommand{\gdtempty}{\bullet_{\Gdt}}

% AnnTree Ant
\newcommand{\Ant}{\ty{Ant}}
\newcommand{\antrhs}[2]{
	\vcenter{\hbox{\begin{forest}
				anttree,
				[ [{$#1$\,$#2$}] ]
			\end{forest}}}}
\newcommand{\antseq}[2]{
	\vcenter{\hbox{\begin{forest}
				anttree,
				for tree={delay={edge={-}}},
				[ [{$#1$}] [{$#2$}] ]
			\end{forest}}}}
\newcommand{\antbang}[2]{
	\vcenter{\hbox{\begin{forest}
				anttree,
				for tree={delay={edge={-}}},
				[ [{$#1$\,\lightning} [{$#2$}] ] ]
			\end{forest}}}}
\newcommand{\antempty}{\bullet_{\Ant}}

% Graphic notation for trees
\forestset{%
	clausetree/.style={
			for tree={
					grow'=0,
					calign=first,
					anchor=parent,
					line width=0.2mm, % This one only affects layout, not appearance of lines. Hence we set this again in delay
					inner sep=2pt,
					s sep=0pt,
					delay={
							edge={line width=0.2mm}},
					l=0em % fix some excessive horizontal space usage for empty nodes
				},
			forked edges
		},
	grdtree/.style={
			clausetree,
			guards/.style={edge={-Bar}},
			grhs/.style={tier=rhs,edge={->}},
			% Everything except the root and the leaves is guards
			for descendants={delay={if n children=0{grhs}{guards}}}
		},
	anttree/.style={
			clausetree,
			arhs/.style={tier=rhs,edge={->}},
			for descendants={delay={if n children=0{arhs}{}}}
		},
}

% Desugaring function
\newcommand{\ds}{\mathcal{D}}

% Checking functions
\newcommand{\unc}{\mathcal{U}}
\newcommand{\ann}{\mathcal{A}}
\newcommand{\red}{\mathcal{R}}
\newcommand{\uncann}{\unc\hspace{-0.35em}\ann}

% Refinement type functions
\newcommand{\generate}{\mathcal{G}}
\newcommand{\normalise}{\mathcal{N}}
\newcommand{\expand}{\mathcal{E}}

% Refinement type syntax
\newcommand{\reft}[2]{\langle \, #1 \mid #2 \, \rangle}
\newcommand{\andtheta}{\,\dot{\wedge}\,}
\newcommand{\uniontheta}{\,\cup\,}

% Normalised refinement types
\newcommand{\inv}[1]{I#1}
\newcommand{\nreft}[2]{\langle #1 \!\parallel\! #2 \rangle}
\newcommand{\adddelta}{\,\oplus_{\delta}\,}
\newcommand{\restrict}[2]{#1 \! \mid_{#2}}
\newcommand{\addphi}{\,\oplus_{\varphi}\,}
\newcommand{\inhabited}[2]{#1 \vdash #2 \, \mathsf{inh}}
\newcommand{\inhabitedbot}{\textsc{$\vdash$Bot}\xspace}
\newcommand{\inhabitednocpl}{\textsc{$\vdash$NoCpl}\xspace}
\newcommand{\inhabitedinst}{\textsc{$\vdash$Inst}\xspace}
\newcommand{\inhabitednt}{\textsc{$\vdash$NT}\xspace}
\newcommand{\cons}{\mathsf{Cons}}
\newcommand{\inst}{\mathsf{Inst}}
\newcommand{\rep}[2]{#1(#2)}
\newcommand{\repnt}[2]{#1_{\text{\tiny NT}}(#2)}
\newcommand{\addphiv}{\,\dot{\oplus}_{\varphi}\,} % "Vectorised" \addphi, hence \addphiv
\newcommand{\throttle}[2]{\left\lfloor#2\right\rfloor_{#1}}
\chapter{Introduction}\label{sec:intro}

LYG detects right hand sides of
pattern matching expressions that are
redundant.

-> Siehe Paper
-> Leaf -> Rhs
\chapter{Background}\label{sec:background}

\section{Lower Your Guards}

Lower Your Guards (LYG) is an algorithm that analyzes pattern matching expressions.
The analysis reports uncovered cases, but also redundant and inaccessible right hand sides.

LYG was designed for use in the Glasgow Haskell Compiler,
but the algorithm and its data structures are so universal
that they can be leveraged for other programming languages with pattern matching constructs too.

% Redundant and inaccessible right hand sides are both never evaluated.
% However, while removing redundant right hand sides does not change the semantics of the pattern matching expression,
% removing inaccessible right hand sides might lead to observable side effects.

\subsection{Motivation}

In functional programming, pattern matching is a very popular feature.
This is particularly true for Haskell, where you can define algebraic data types
and easily match on them in function definitions.
With increasingly complex data types and function definitions however,
pattern matching can be yet another source of mistakes.

Listing \ref{lst:haskell} showcases common types of mistakes that can arise with pattern matching.

Most importantly, the function $f$ is not defined on all values.
The case $f\;\mathrm{Case4}\;\mathrm{False}$ is uncovered and will cause a runtime error.
Also, the right hand sides (RHS) $3$ and $4$ are inaccessible - $f$ will never evaluate to $3$ or $4$!

\begin{lstlisting}[caption={A Pattern Matching Example In Haskell},label={lst:haskell},language=Haskell]
data Cases = Case1 | Case2 | Case3 | Case4

f :: Cases -> Bool -> Integer
f Case1 _ = 1
f Case2 _ = 2
f x True    | Case1 <- x = 3
            | Case2 <- x = 4
f Case3 _ = 5
\end{lstlisting}

LYG provides a robust algorithm that is able to detect such mistakes. LYG can also deal with the intricacies of lazy evaluation as discussed in the following chapter.

\subsection{Lazy Evaluation}

A closer look at listing \ref{lst:haskell} reveals that while both RHS 3 and 4 are inaccessible,
the semantics of $f$ changes if both are removed.
This means that an automated refactoring is not allowed to just remove all inaccessible leaves!

The reason for this is the term $t := f\;\mathrm{Case3}\;\mathrm{undefined}$ and the fact that Haskell uses a lazy evaluation strategy.
If both RHSs $3$ and $4$ are removed, $t$ evaluates to $5$ - the term $\mathrm{undefined}$ is never evaluated as no pattern matches against it.
However, if nothing or only one of the RHSs $3$ or $4$ is removed, $\mathrm{undefined}$ will be matched with $\mathrm{True}$ and thus $t$ will throw an error!

To communicate this difference, LYG introduces the concept of $\mathit{redundant}$ and $\mathit{inaccessible}$ RHSs:
A redundant RHS can be removed from its pattern matching expression without any observable difference.
An inaccessible RHS is never evaluated, but its removal might lead to observable changes.
This definition implies that redundant RHSs are inaccessible.

As of listing \ref{lst:haskell}, LYG will mark RHS $3$ as inaccessible and RHS $4$ as redundant.
This choice is somewhat arbitrary, as RHS $3$ could be marked as redundant and RHS $4$ as inaccessible as well, and will be discussed in more detail chapter \TODOI{ref}.

\subsection{Guard Trees}

For all analyses, LYG first transforms Haskell specific pattern match expressions to simpler \textit{guard trees}.
This transformation removes a lot of complexity, as many different Haskell constructs can be desugared
to the same guard tree. Guard trees also simplify adapting LYG to other programming languages
and they enable studying LYG mostly independent from Haskell.
Their syntax is defined in figure \ref{fig:guardTrees}.

Guard trees (Gdts) are made of three elements: Uniquely numbered right hand sides, branches and guarded trees.
Guarded trees refer to Haskell specific guards (Grd) that control the execution.
\textit{Let guards} can bind a term to a variable in a new lexical scope,
\textit{pattern match guards} can destructure a value into variables if the pattern matches or otherwise prevent the
execution from entering the tree behind the guard
and \textit{bang guards} can stop the entire execution when the value of a variable does not reduce to a head normal form.

\begin{figure}[htbp]
	\caption{Definition of Guard Trees}
	\label{fig:guardTrees}
	\centering
	\[ \textbf{Guard Syntax} \]
	\[
		\begin{array}{cc}
			\begin{array}{rlcl}
				k,n,m       \in & \mathbb{N} &           &                                                 \\
				K           \in & \Con       &           &                                                 \\
				x,y,a,b     \in & \Var       &           &                                                 \\
				\tau,\sigma \in & \Type      & \Coloneqq & a \mid ...                                      \\
				e \in           & \Expr      & \Coloneqq & x \mid  \genconapp{K}{\tau}{\gamma}{e} \mid ... \\
			\end{array} &
			\begin{array}{rlcl}
				\gamma \in & \TyCt & \Coloneqq & \tau_1 \typeeq \tau_2 \mid ...               \\
				p \in      & \Pat  & \Coloneqq & \_ \mid K \; \overline{p} \mid ...           \\
				g \in      & \Grd  & \Coloneqq & \grdlet{x:\tau}{e}                           \\
				           &       & \mid      & \grdcon{\genconapp{K}{a}{\gamma}{y:\tau}}{x} \\
				           &       & \mid      & \grdbang{x}                                  \\
			\end{array}
		\end{array}
	\]

	\[ \textbf{Guard Tree Syntax} \]
	\[
		\begin{array}{rcll}
			t \in \Gdt & \Coloneqq & \gdtrhs{k} \mid \gdtseq{t_1}{t_2} \mid \gdtguard{g}{t} \\
		\end{array}
	\]
\end{figure}

The evaluation of a guard tree selects the first right hand side that execution reaches.
If the execution stops at a bang guard, the evaluation diverges, otherwise, if execution falls through, the evaluation ends with a no-match.
A formal semantic for guard trees will be defined in chapter \TODOI{ref}.

The transformation from Haskell pattern matches to guard trees is not of much interest for this thesis and can be found in \TODOI{ref}.
To preserve semantics, it is important that the transformation inserts bang guards whenever a variable is matched against a data constructor.

The figure \ref{fig:desugaringExample} represents the guard tree of listing \ref{lst:haskell}.


\begin{figure}[htbp]
	\caption{Desugaring Example}
	\label{fig:desugaringExample}
	\begin{lstlisting}[language=Haskell]
data Cases = Case1 | Case2 | Case3 | Case4

f :: Cases -> Bool -> Integer
f Case1 _ = 1
f Case2 _ = 2
f x True    | Case1 <- x = 3
            | Case2 <- x = 4
f Case3 _ = 5
\end{lstlisting}

	$\Downarrow$

	\begin{forest}
		grdtree
		[
		[{$\grdbang{x_1},\, \grdcon{\mathtt{Case1}}{x_1}$} [1]]
			[
				[{$\grdbang{x_1},\, \grdcon{\mathtt{Case2}}{x_1}$} [2]]
					[
						[{$\grdlet{x}{x_1},\, \grdbang{x_2},\, \grdcon{\mathtt{True}}{x_2}$}
									[{$\grdbang{x},\, \grdcon{\mathtt{Case1}}{x}$} [3]]
									[{$\grdbang{x},\, \grdcon{\mathtt{Case2}}{x}$} [4]]
							]
							[{$\grdbang{x_1},\, \grdcon{\mathtt{Case3}}{x_1}$} [5]]
					]
			]
		]
	\end{forest}
\end{figure}

It is usually straightforward to define a transformation from pattern matching expressions to guard trees
that also preserves uncovered cases and inaccessible and redundant RHSs.
This makes guard trees an ideal abstraction for the following analysis steps.

\subsection{Refinement Types}

\textit{Refinement types} describe vectors of values $x_1, ..., x_n$ that satisfy a given predicate $\Phi$.
Their syntax is defined in figure \ref{fig:refinementTypes}.

\begin{figure}[htbp]
	\caption{Definition of Refinement Types}
	\label{fig:refinementTypes}
	\centering
	\[
		\begin{array}{rcll}
			\Gamma  & \Coloneqq & \varnothing \mid \Gamma, x:\tau \mid \Gamma, a                                                                                                  & \text{Context}         \\
			\varphi & \Coloneqq & \true \mid \false \mid \ctcon{\genconapp{K}{a}{\gamma}{y:\tau}}{x} \mid x \ntermeq K \mid x \termeq \bot \mid x \ntermeq \bot \mid \ctlet{x}{e} & \text{Literals}        \\
			\Phi    & \Coloneqq & \varphi \mid \Phi \wedge \Phi \mid \Phi \vee \Phi                                                                                               & \text{Formula}         \\
			\Theta  & \Coloneqq & \reft{\Gamma}{\Phi}                                                                                                                             & \text{Refinement type} \\
		\end{array}
	\]
\end{figure}

Refinement types are built from literals $\phi$ and closed under conjunction and disjunction.
The literal $\true$ means ``true'', while $\false$ means ``false''. For example:
$$
	\begin{array}{rcl}
		\reft{ x{:}\ensuremath{\Conid{Bool}}}{ \true }                                                                                                           & \text{denotes} & \{ \bot, \ensuremath{\Conid{True}}, \ensuremath{\Conid{False}} \}                       \\
		\reft{ x{:}\ensuremath{\Conid{Bool}}}{ x \ntermeq \bot }                                                                                                 & \text{denotes} & \{ \ensuremath{\Conid{True}}, \ensuremath{\Conid{False}} \}                             \\
		\reft{ x{:}\ensuremath{\Conid{Bool}}}{ x \ntermeq \bot \wedge \ctcon{\ensuremath{\Conid{True}}}{x} }                                                     & \text{denotes} & \{ \ensuremath{\Conid{True}} \}                                                         \\
		\reft{ mx{:}\ensuremath{\Conid{Maybe}\;\Conid{Bool}}}{ mx \ntermeq \bot \wedge \ctcon{\ensuremath{\Conid{Just}\;\Varid{x}}}{mx} \wedge x \ntermeq \bot } & \text{denotes} & \{ \ensuremath{\Conid{Just}\;\Conid{True}}, \ensuremath{\Conid{Just}\;\Conid{False}} \} \\
	\end{array}
$$

Unconventionally, a literal can bind one or more variables in a way that such a binding is in scope in all literals on its right.
Thus, $(\ctlet{x}{y} \land z \ntermeq \bot) \land x \ntermeq \bot$ is equivalent to $z \ntermeq \bot \land y \ntermeq \bot$.
While this conjunction operator is still associative, it is clearly not commutative!

LYG also describes a partially defined function $\generate$
with $\generate(\Theta) = \varnothing \implies \Theta \text{ denotes } \varnothing$ for all refinement types $\Theta$.
$\generate$ is used to $\generate$enerate inhabitants of a refinement type to build elaborate error messages
and to get a guarantee that a refinement type is empty.
A precise $\generate$ function is undecidable!
This thesis just assumes that ``interesting'' correct $\generate$ functions exist,
so the details of $\generate$ as proposed by LYG do not matter.

\subsection{Uncovered Analysis}

The goal of the uncovered analysis is to detect all cases that are not covered by a given guard tree.
Refinement types are used to capture the result of this analysis.

The function $\unc(\reft{\Gamma}{\true}, \cdot)$ in figure \ref{fig:U}
computes a refinement type that captures all uncovered values for a given guard tree.
This refinement type is empty if and only if there are not any uncovered cases.
If $\generate$ is used to test for emptiness, this yields an algorithm to test for uncovered cases.

\begin{figure}[htbp]
	\caption{Definition of $\unc$}
	\label{fig:U}
	\[ \ruleform{ \unc(\Theta, t) = \Theta } \]
	\[
		\begin{array}{lcl}
			\unc(\reft{\Gamma}{\Phi}, \gdtrhs{n})                                    & = & \reft{\Gamma}{\false}                                                                                                 \\
			\unc(\Theta, \gdtseq{t_1}{t_2})                                          & = & \unc(\unc(\Theta, t_1), t_2)                                                                                          \\
			\unc(\Theta, \gdtguard{\grdbang{x}}{t})                                  & = & \unc(\Theta \andtheta (x \ntermeq \bot), t)                                                                           \\
			\unc(\Theta, \gdtguard{\grdlet{x}{e}}{t})                                & = & \unc(\Theta \andtheta (\ctlet{x}{e}), t)                                                                              \\
			\unc(\Theta, \gdtguard{\grdcon{\genconapp{K}{a}{\gamma}{y:\tau}}{x}}{t}) & = & (\Theta \andtheta (x \ntermeq K)) \uniontheta \unc(\Theta \andtheta (\ctcon{\genconapp{K}{a}{\gamma}{y:\tau}}{x}), t) \\
		\end{array}
	\]
\end{figure}

\begin{figure}[htbp]
	\caption{Example of $\unc$}
	\label{fig:UEx}
	\begin{array}{rcl}
		\unc(
		\reft{\Gamma}{\true},
		\begin{forest}
			grdtree
			[
			[{$\grdbang{x_1},\, \grdcon{\mathtt{Case1}}{x_1}$} [1]]
				[
					[{$\grdbang{x_1},\, \grdcon{\mathtt{Case2}}{x_1}$} [2]]
						[
							[{$\grdlet{x}{x_1},\, \grdbang{x_2},\, \grdcon{\mathtt{True}}{x_2}$}
										[{$\grdbang{x},\, \grdcon{\mathtt{Case1}}{x}$} [3]]
										[{$\grdbang{x},\, \grdcon{\mathtt{Case2}}{x}$} [4]]
								]
								[{$\grdbang{x_1},\, \grdcon{\mathtt{Case3}}{x_1}$} [5]]
						]
				]
			]
		\end{forest}
		) \\
		= \unc(
		\unc(
			\reft{\Gamma}{\true},
			\begin{forest}
				grdtree
				[
				[{$\grdbang{x_1},\, \grdcon{\mathtt{Case1}}{x_1}$} [1]]
				]
			\end{forest}
			),
		\begin{forest}
				grdtree
				[
				[
						[{$\grdbang{x_1},\, \grdcon{\mathtt{Case2}}{x_1}$} [2]]
							[
								[{$\grdlet{x}{x_1},\, \grdbang{x_2},\, \grdcon{\mathtt{True}}{x_2}$}
											[{$\grdbang{x},\, \grdcon{\mathtt{Case1}}{x}$} [3]]
											[{$\grdbang{x},\, \grdcon{\mathtt{Case2}}{x}$} [4]]
									]
									[{$\grdbang{x_1},\, \grdcon{\mathtt{Case3}}{x_1}$} [5]]
							]
					]
				]
			\end{forest}
		)
	\end{array}
\end{figure}

\subsection{Annotated Guard Trees}

\textit{Annotated guard trees} represent simplified guard trees that have been annotated with refinement types.
Their syntax is defined in figure \ref{fig:annotatedGuardTree}.

\begin{figure}[htbp]
	\caption{Definition of Annotated Guard Trees}
	\label{fig:annotatedGuardTree}
	\centering
	\[
		u \in \Ant \Coloneqq \antrhs{\Theta}{k} \mid \antseq{u_1}{u_2} \mid \antbang{\Theta}{u}
	\]
\end{figure}

\subsection{Redundant/Inaccessible Analysis}

The goal of the redundant/inaccessible analysis is to report as much RHSs
as possible that are guaranteed to be redundant or inaccessible.
This is done by annotating a guard tree with refinement types and then checking these refinement types for emptiness.
If a RHS is associated with an empty refinement type, the RHS is inaccessible and in some circumstances even redundant.

\begin{figure}[htbp]
	\caption{Definition of $\ann$}
	\label{fig:A}
	\[ \ruleform{ \ann(\Theta, t) = u } \]
	\[
		\begin{array}{lcl}
			\ann(\Theta,\gdtrhs{n})                                                  & = & \antrhs{\Theta}{n}                                                                       \\
			\ann(\Theta, \gdtseq{t_1}{t_2})                                          & = & \antseq{\ann(\Theta, t_1)}{\ann(\unc(\Theta, t_1), t_2)}                                 \\
			\ann(\Theta, \gdtguard{\grdbang{x}}{t})                                  & = & \antbang{\Theta \andtheta (x \termeq \bot)}{\ann(\Theta \andtheta (x \ntermeq \bot), t)} \\
			\ann(\Theta, \gdtguard{\grdlet{x}{e}}{t})                                & = & \ann(\Theta \andtheta (\ctlet{x}{e}), t)                                                 \\
			\ann(\Theta, \gdtguard{\grdcon{\genconapp{K}{a}{\gamma}{y:\tau}}{x}}{t}) & = & \ann(\Theta \andtheta (\ctcon{\genconapp{K}{a}{\gamma}{y:\tau}}{x}), t)                  \\
		\end{array}
	\]
\end{figure}

\begin{figure}[htbp]
	\caption{Definition of $\red$}
	\centering
	%\[ \textbf{Collect accessible $(\overline{k})$, inaccessible $(\overline{n})$ and $\red$edundant $(\overline{m})$ RHSs} \]
	\[ \ruleform{ \red(u) = (\overline{k}, \overline{n}, \overline{m}) } \]
	\[
		\begin{array}{lcl}
			\red(\antrhs{\Theta}{n})  & = & \begin{cases}
				(\epsilon, \epsilon, n), & \text{if $\generate(\Theta) = \emptyset$} \\
				(n, \epsilon, \epsilon), & \text{otherwise}                          \\
			\end{cases}                                                                                                                     \\
			\red(\antseq{t}{u})       & = & (\overline{k}\,\overline{k'}, \overline{n}\,\overline{n'}, \overline{m}\,\overline{m'}) \hspace{0.5em} \text{where} \begin{array}{l@{\,}c@{\,}l}
				(\overline{k}, \overline{n}, \overline{m})    & = & \red(t) \\
				(\overline{k'}, \overline{n'}, \overline{m'}) & = & \red(u) \\
			\end{array} \\
			\red(\antbang{\Theta}{t}) & = & \begin{cases}
				(\epsilon, m, \overline{m'}), & \text{if $\generate(\Theta) \not= \emptyset$ and $\red(t) = (\epsilon, \epsilon, m\,\overline{m'})$} \\
				\red(t),                      & \text{otherwise}                                                                                     \\
			\end{cases}                                                                                                                     \\
		\end{array}
	\]
\end{figure}

\section{Lean}

\subsection{Lean Prover}

\subsection{Mathlib}

\chapter{Formalization}\label{sec:formalization}

Before any property of LYG can be proven or even stated in Lean, all relevant definitions must be formalized.
Since nothing can be left vague in Lean, a lof of decisions had to be made to back up LYG by a fully defined model. 
This chapter discusses these decisions.

\section{Definitions}

\subsection{Abstracting LYG: The Guard Module}
LYG does not specify an exact guard or expression syntax.
Instead, the notation ``$...$'' is often used to indicate a sensible continuation to make guards powerful enough to model all Haskell constructs.
This is rather problematic for a precise formalization and presented the first big challenge of this thesis.
As we wanted to avoid formalizing Haskell and its semantics, we had to carefully design an abstraction that is as close as possible to LYG
while pinning down guards to a closed but extendable theory.

\subsubsection{The Result Monad}

First, we defined a generic $\mathtt{Result}$ monad to capture the result of an evaluation.
Due to laziness, evaluation of guard trees can either end with a specific right hand side, not match any guard or diverge:

\begin{minted}{Lean}
inductive Result (α: Type)
| value: α → Result
| diverged: Result
| no_match: Result
\end{minted}

A $\mathtt{bind}$ operation can be easily defined on $\mathtt{Result}$ to make it a proper monad with $\mathtt{Result.value}$ as unit function:

\begin{minted}{Lean}
def Result.bind { α β: Type } (f: α → Result β): Result α → Result β
| (Result.value val) := f val
| Result.diverged := Result.diverged
| Result.no_match := Result.no_match
\end{minted}

\subsubsection{Denotational Semantic for Guards}

For some abstract environment type $\mathtt{Env}$, we would like to have a denotational semantic $\mathtt{Grd.eval}$ for guards $\mathtt{Grd}$:
\begin{minted}{Lean}
Grd.eval : Grd → Env → Result Env
\end{minted}

Abstracting $\mathtt{Grd.eval}$ would unify all guard constructs available in Haskell and those used by LYG.
However, LYG needs to recognize all guards that can lead to a diverged evaluation:
Removing all RHSs behind such a guard would inevitably remove the guard itself.
As this might change the semantic of the guard tree, LYG cannot mark all such RHSs as redundant unless there is a proof that the guard will never diverge.
As a consequence, $\mathtt{Grd.eval}$ cannot be abstracted away.

Instead, we explicitly distinguished between non-diverging (\textit{total}) $\mathtt{tgrd}$s and non-no-matching $\mathtt{bang}$ guards:
\begin{minted}{Lean}
inductive Grd
| tgrd (tgrd: TGrd)
| bang (var: Var)
\end{minted}

While $\mathtt{TGrd}$s classically represent guards and $\mathtt{Grd}$s represent guards with side effects in this context,
we decided to follow the naming conventions of LYG and chose the name $\mathtt{TGrd}$ for side-effect free (non-diverging) guards
rather than renaming $\mathtt{Grd}$.

In order to define a denotational semantic on $\mathtt{Grd}$, we postulated the functions $\mathtt{tgrd\_eval}: \mathtt{TGrd} \to \mathtt{Env} \to \mathtt{option}\;\mathtt{Env}$ and $\mathtt{is\_bottom}: \mathtt{Var} \to \mathtt{Env} \to \mathtt{bool}$ as well as a type $\mathtt{Var}$ that represents variables. While $\mathtt{tgrd}$s can change the environment,
$\mathtt{bang}$ guards cannot:

\begin{minted}{Lean}
def Grd.eval : Grd → Env → Result Env
| (Grd.tgrd grd) env :=
    match tgrd_eval grd env with
    | none := Result.no_match
    | some env' := Result.value env'
    end
| (Grd.bang var) env :=
    if is_bottom var env
    then Result.diverged
    else Result.value env
\end{minted}

Alternatively,
we can set $\mathtt{Var} := \mathtt{Env} → \mathtt{bool}$ and $\mathtt{TGrd} := \mathtt{Env} → \mathtt{bool}$ and replace $\mathtt{is\_bottom}$ and $\mathtt{tgrd\_eval}$ with $\mathtt{id}$,
yielding the following definition:
\begin{minted}{Lean}
inductive Grd'
| tgrd (grd: Env → option Env)
| bang (test: Env → bool)
\end{minted}
However, this could make the set of guard trees and refinement types uncountable.
While this is not problematic for aspects explored by this thesis,
it could make implementing a correct function $\mathcal{G}$ impossible, as it cannot reason anymore about guards in a computable way if $\mathtt{Env}$ is instantiated with a non-finite type.

\bigskip

\subsubsection{The Guard Module}

In Lean, type classes provide an ideal mechanism to define such ambient abstractions.
They can be opened, so that all members of the type class become implicitly available in all definitions and theorems.
Every implicit usage pulls the type class into its signature so that consumers can provide a concrete implementation of the type class.

We defined and opened a type class \textit{GuardModule} that describes the presented abstraction:

\begin{minted}{Lean}

class GuardModule :=
    (Rhs : Type)
    [rhs_decidable: decidable_eq Rhs]
    [rhs_inhabited: inhabited Rhs]
    (Env : Type)
    (TGrd : Type)
    [tgrd_inhabited: inhabited TGrd]
    (tgrd_eval : TGrd → Env → option Env)
    (Var : Type)
    [var_inhabited: inhabited Var]
    (is_bottom : Var → Env → bool)

variable [GuardModule]
open GuardModule
\end{minted}

We also postulated a type $\mathtt{Rhs}$ to refer to right hand sides. For technical reasons, equality on this type must be decidable.
This abstracts from the numbers that are used in LYG to distinguish right hand sides.
We also require most types to be inhabited so that we can construct module-independent examples.

All following definitions and theorems implicitly make use of this abstraction.

\subsection{Guard Trees}\label{chap:formalGuardTrees}

\subsubsection{Syntax of Guard Trees}

With the definition of $\mathtt{Grd}$, guard trees are defined as inductive data type:

\begin{minted}{Lean}
inductive Gdt
| rhs (rhs: Rhs)
| branch (tr1: Gdt) (tr2: Gdt)
| grd (grd: Grd) (tr: Gdt)
\end{minted}

\subsubsection{Semantic of Guard Trees}

$\mathtt{Gdt.eval}$ defines a denotational semantic on guard trees, using the semantic of guards.
It returns the first RHS that matches a given environment. If a guard diverges, the entire evaluation diverges. Otherwise, if no RHSs matches, \textit{no-match} is returned.

\begin{minted}{Lean}
def Gdt.eval : Gdt → Env → Result Rhs
| (Gdt.rhs rhs) env := Result.value rhs
| (Gdt.branch tr1 tr2) env :=
    match tr1.eval env with
    | Result.no_match := tr2.eval env
    | r := r
    end
| (Gdt.grd grd tr) env := (grd.eval env).bind tr.eval
\end{minted}

\subsubsection{RHSs in Guard Trees}

Every guard tree contains a (non-empty) finite set of right hand sides:
\begin{minted}{Lean}
def Gdt.rhss: Gdt → finset Rhs
| (Gdt.rhs rhs) := { rhs }
| (Gdt.branch tr1 tr2) := tr1.rhss ∪ tr2.rhss
| (Gdt.grd grd tr) := tr.rhss
\end{minted}

In LYG, it is implicitly assumed that the right hand sides of a guard tree are numbered unambiguously.
This has to be stated explicitly in Lean with the following recursive predicate:

\begin{minted}{Lean}
def Gdt.disjoint_rhss: Gdt → Prop
| (Gdt.rhs rhs) := true
| (Gdt.branch tr1 tr2) :=
        disjoint tr1.rhss tr2.rhss
        ∧ tr1.disjoint_rhss ∧ tr2.disjoint_rhss
| (Gdt.grd grd tr) := tr.disjoint_rhss
\end{minted}

\subsubsection{Removing RHSs in Guard Trees}

$\mathtt{Gdt.remove\_rhss}$ defines how a set of RHSs can be removed from a guard tree.
This definition is required to state that all redundant RHSs can be removed without changing semantics.
Note that the resulting guard tree might be empty when all RHSs are removed!

\begin{minted}{Lean}
def Gdt.branch_option : option Gdt → option Gdt → option Gdt
| (some tr1) (some tr2) := some (Gdt.branch tr1 tr2)
| (some tr1) none := some tr1
| none (some tr2) := some tr2
| none none := none

def Gdt.grd_option : Grd → option Gdt → option Gdt
| grd (some tr) := some (Gdt.grd grd tr)
| _ none := none

def Gdt.remove_rhss : finset Rhs → Gdt → option Gdt
| rhss (Gdt.rhs rhs) := if rhs ∈ rhss then none else some (Gdt.rhs rhs)
| rhss (Gdt.branch tr1 tr2) :=
    Gdt.branch_option
        (tr1.remove_rhss rhss)
        (tr2.remove_rhss rhss)
| rhss (Gdt.grd grd tr) := Gdt.grd_option grd (tr.remove_rhss rhss)
\end{minted}

Finally, to deal with the semantic of empty guard trees,
$\mathtt{Gdt.eval\_option}$ lifts $\mathtt{Gdt.eval}$ to $\mathtt{option\;Gdt}$:

\begin{minted}{Lean}
def Gdt.eval_option : option Gdt → Env → Result
| (some gdt) env := gdt.eval env
| none env := Result.no_match
\end{minted}

\subsection{Refinement Types}
\label{sec:formalizationRefinementTypes}

Refinement types presented another challenge.
Defining refinement types through a proper type system would have required to model some Haskell types.
Instead, we tried to rely on the same abstractions used to define guard trees in hope that guard trees and refinement types can be related.

In this formalization, a refinement type $\Phi$ denotes a predicate on environments:

\begin{minted}{Lean}
def Φ.eval: Φ → Env → bool
\end{minted}

With a proper $\mathtt{GuardModule}$ instantiation, the environment can be used to not only carry runtime values, but also their type!
A (well) typed environment can assist in proving a refinement type to be empty.


\subsubsection{Variable Binding Rules}
\label{sec:formalizationVariableBindingRules}

Another problem that had to be solved was the formalization of the unconventional binding mechanism of refinement types through conjunctions, as described in chapter \ref{chap:bckgrndRefinementTypesBinding}.
In particular, this causes $\mathcal{U}$ to be not correct (for some intuitive notion of correctness)
with regards to the guard tree semantic we defined in chapter \ref{chap:formalGuardTrees}.
While the following guard tree $gdt$ does not match for any environment, its uncovered refinement type $\Theta$ computed by $\mathcal{U}$ is empty (the empty vector does not match)!

\[
    \mathit{gdt} :=
    \vcenter{\hbox{
    \begin{forest}
    	grdtree
    	[
    	[
    		{$\grdlet{x}{\mathtt{False}}$}
    		[{$\grdlet{x}{\mathtt{True}},\, \grdcon{\mathtt{False}}{x}$} [1]]
    		[{$\grdcon{\mathtt{False}}{x}$} [2]]
    	]
    	]
    \end{forest}
    }}
\]
\begin{align*}
    \Theta := \mathcal{U}(\true, \mathit{gdt}) =& \reft{}{
        (((\grdlet{x}{\Conid{False}} \andtheta \grdlet{x}{\Conid{True}}) \andtheta
        x \ntermeq \Conid{False}) \andtheta x \ntermeq \Conid{False}
    } \\
    =& \reft{}{
        \grdlet{x}{\Conid{False}} \land (\grdlet{x}{\Conid{True}} \land
        (x \ntermeq \Conid{False} \land x \ntermeq \Conid{False}))
    }
\end{align*}

Shadowing is unproblematic for the semantic of guard trees though: If the first guard tree of a branch fails to match, its environment just before the failing guard is discarded and with it possible shadowing bindings.
The second branch is always evaluated with the same environment that the first guard tree has been evaluated with.

This imbalance between refinement types and the semantic of guard trees could be fixed by either adjusting the semantics of guard trees or by
introducing a guard operator for refinement types with a restricted binding scope.
However, if the semantic of guard trees would not undo the effect of no-matching branches to the environment,
an inner let binding would prevent an inaccessible right hand side from being redundant. Even though it never matches, the let binding would always have an effect
on the final environment and removing it would be observable!

This problem does not arise in the GHC implementation of LYG as it uses a different encoding for refinement types.

We introduced a data constructor $\mathtt{Φ.tgrd\_in}: \mathtt{TGrd} \to Φ \to Φ$ that limits the scope of the guard to the nested refinement type and removed any scoping behaviour of the $\land$-operator.
This simplifies the scoping mechanism and allows to fix the problem of shadowing bindings in $\mathcal{U}$.

\subsubsection{Syntax of Refinement Types}

Finally, this is our formalized syntax of refinement types:

\begin{minted}{Lean}
inductive Φ
| false
| true
| tgrd_in (tgrd: TGrd) (ty: Φ)
| not_tgrd (tgrd: TGrd)
| var_is_bottom (var: Var)
| var_is_not_bottom (var: Var)
| or (ty1: Φ) (ty2: Φ)
| and (ty1: Φ) (ty2: Φ)
\end{minted}

Since the negation of a guard cannot bind variables,
it does not need to have a nested refinement type that would see bound variables.
The same applies to $\mathtt{var\_is\_bottom}$ and its negation.

\subsubsection{Semantic of Refinement Types}

The semantic of refinement types is easily defined and implicitly uses the guard module:

\begin{minted}{Lean}
def Φ.eval: Φ → Env → bool
| Φ.false env := ff
| Φ.true env := tt
| (Φ.tgrd_in grd ty) env := match tgrd_eval grd env with
    | some env := ty.eval env
    | none := ff
    end
| (Φ.not_tgrd grd) env :=
    match tgrd_eval grd env with
    | some env := ff
    | none := tt
    end
| (Φ.var_is_bottom var) env := is_bottom var env
| (Φ.var_is_not_bottom var) env := !is_bottom var env
| (Φ.or t1 t2) env := t1.eval env || t2.eval env
| (Φ.and t1 t2) env := t1.eval env && t2.eval env
\end{minted}

With this definition the evaluation of the second operand of a conjunction is obviously independent
of any environment effects applied in the evaluation of the first operand!

\subsubsection{Definition of \texttt{is\_empty}}

A refinement type $\Phi$ is called \textit{empty} if it does not match any environment.
This is formalized by the predicate $\Phi\mathtt{.is\_empty}$:

\begin{minted}{Lean}
def Φ.is_empty (ty: Φ): Prop := ∀ env: Env, ¬(ty.eval env)
\end{minted}

\subsubsection{Definition of \texttt{can\_prove\_empty}}

Instead of a partial function $\generate$ with $\generate(Φ) = \varnothing$ if and only if $Φ$ is empty,
we define a total function $\mathtt{can\_prove\_empty}$ and a predicate $\mathtt{correct\_can\_prove\_empty}$ that ensures
its correctness. This abstracts from the in this context unneeded generation of inhabitants. It also avoids dealing with partial functions, which are not supported by Lean.
\begin{minted}{Lean}
variable can_prove_empty: Φ → bool
def correct_can_prove_empty : Prop :=
    ∀ ty: Φ, can_prove_empty ty = tt → ty.is_empty
\end{minted}

The subtype $\mathtt{CorrectCanProveEmpty}$ bundles a correct $\mathtt{can\_prove\_empty}$ function:
\begin{minted}{Lean}
def CorrectCanProveEmpty := {
    can_prove_empty : Φ → bool
    // correct_can_prove_empty can_prove_empty
}
\end{minted}

\subsection{$\mathcal{U}$ncovered Analysis}
\label{sec:formalizationUncoveredAnalysis}

As discussed in chapter \ref{sec:formalizationRefinementTypes}, LYG's definition of
$\mathcal{U}$ has problems with guard trees that define shadowing bindings.
LYG defined $\mathcal{U}$ as follows (see chapter \ref{chap:bckgrdUncoveredAnalysis} for the discussion of this definition):

\[
\begin{array}{lcl}
	\unc(\reft{\Gamma}{\Phi}, \gdtrhs{n})                                    & = & \reft{\Gamma}{\false}                                                                                                 \\
	\unc(\Theta, \gdtseq{t_1}{t_2})                                          & = & \unc(\unc(\Theta, t_1), t_2)                                                                                          \\
	\unc(\Theta, \gdtguard{\grdbang{x}}{t})                                  & = & \unc(\Theta \andtheta (x \ntermeq \bot), t)                                                                           \\
	\unc(\Theta, \gdtguard{\grdlet{x}{e}}{t})                                & = & \unc(\Theta \andtheta (\ctlet{x}{e}), t)                                                                              \\
	\unc(\Theta, \gdtguard{\grdcon{\genconapp{K}{a}{\gamma}{y:\tau}}{x}}{t}) & = & \Theta \andtheta (x \ntermeq K) \uniontheta \unc(\Theta \andtheta (\ctcon{\genconapp{K}{a}{\gamma}{y:\tau}}{x}), t) \\
\end{array}
\]

Equipped with the data constructor $\mathtt{Φ.tgrd\_in}$ we can fix the shadowing problem problem and formalize $\mathcal{U}$ now.
Instead of using $\Phi$ as accumulator type, our formalization uses the type $\Phi \to \Phi$:
The new accumulator explicitly applies a context to a refinement type.
This happens implicitly in LYG's definition through the use of $\Theta \andtheta \cdot$.
Note that all occuring accumulator functions are homomorphic under the semantic of refinement types.
We carefully make use of this to get formalized definitions of $\mathcal{U}$ and $\mathcal{A}$ that can be interleaved, as done in LYG.

\begin{minted}{Lean}
def ω_acc : (Φ → Φ) → Gdt → Φ
| acc (Gdt.rhs _) := Φ.false
| acc (Gdt.branch tr1 tr2) := (ω_acc ((ω_acc acc tr1).and ∘ acc) tr2)
| acc (Gdt.grd (Grd.bang var) tr) :=
    ω_acc (acc ∘ (Φ.var_is_not_bottom var).and) tr
| acc (Gdt.grd (Grd.tgrd grd) tr) :=
            (acc (Φ.not_tgrd grd))
        .or (ω_acc (acc ∘ (Φ.tgrd_in grd)) tr)

def ω : Gdt → Φ := ω_acc id
\end{minted}

\subsection{$\mathcal{R}$edundant / Inaccessible Analysis}

\subsubsection{Formalization of Annotated Trees}

The formalization of annotated trees is straightforward.
However, we allow arbitrary annotations rather than only accepting refinement types.
This will become useful in formal proofs when we no longer care about
the specific refinement types but only if they are empty.

\begin{minted}{Lean}
inductive Ant (α: Type)
| rhs (a: α) (rhs: Rhs): Ant
| branch (tr1: Ant) (tr2: Ant): Ant
| diverge (a: α) (tr: Ant): Ant
\end{minted}

\subsubsection{Formalization of $\mathcal{A}$}

Similar to the formalization of $\mathcal{U}$ in chapter \ref{sec:formalizationUncoveredAnalysis}, we also need to address the shadowing problem when formalizing $\mathcal{A}$.
This is LYG's definition of $\mathcal{A}$ as stated in chapter \ref{chap:backgrdRedundantInaccAnalysis}:

\[
	\begin{array}{lcl}
		\ann(\Theta,\gdtrhs{n})                                                  & = & \antrhs{\Theta}{n}                                                                       \\
		\ann(\Theta, \gdtseq{t_1}{t_2})                                          & = & \antseq{\ann(\Theta, t_1)}{\ann(\unc(\Theta, t_1), t_2)}                                 \\
		\ann(\Theta, \gdtguard{\grdbang{x}}{t})                                  & = & \antbang{\Theta \andtheta (x \termeq \bot)}{\ann(\Theta \andtheta (x \ntermeq \bot), t)} \\
		\ann(\Theta, \gdtguard{\grdlet{x}{e}}{t})                                & = & \ann(\Theta \andtheta (\ctlet{x}{e}), t)                                                 \\
		\ann(\Theta, \gdtguard{\grdcon{\genconapp{K}{a}{\gamma}{y:\tau}}{x}}{t}) & = & \ann(\Theta \andtheta (\ctcon{\genconapp{K}{a}{\gamma}{y:\tau}}{x}), t)                  \\
	\end{array}
\]

Our formalization in Lean follows. Analogous to our formalization of $\mathcal{U}$, instead of contextualizing refinement types by
combining them with the accumulator through $\andtheta$, we model the accumulator as an explicit function
that contextualizes its argument:

\begin{minted}{Lean}
def π_acc : (Φ → Φ) → Gdt → Ant Φ
| acc (Gdt.rhs rhs) := Ant.rhs (acc Φ.true) rhs
| acc (Gdt.branch tr1 tr2) :=
    Ant.branch
        (π_acc acc tr1)
        (π_acc ((ω_acc acc tr1).and ∘ acc) tr2)
| acc (Gdt.grd (Grd.bang var) tr) :=
    Ant.diverge
        (acc (Φ.var_is_bottom var)) 
        (π_acc (acc ∘ ((Φ.var_is_not_bottom var).and)) tr)
| acc (Gdt.grd (Grd.tgrd grd) tr) :=
    (π_acc (acc ∘ (Φ.tgrd_in grd)) tr)

def π : Gdt → Ant Φ := π_acc id
\end{minted}

Note that in the branch case, $\mathcal{A}\mathtt{\_acc}$ and $\mathcal{U}\mathtt{\_acc}$
are called with the same arguments. Even more, both functions have the same recursion structure, which
makes it possible to interleave both functions.
This is done in chapter \ref{sec:formalizationInterleaving}.


%-- returns (accessible, inaccessible, redundant) rhss, given that `can_prove_empty` is correct.

\subsubsection{Formalization of $\mathcal{R}$}

It remains to formalize the function $\mathcal{R}$ that partitions
all right hand sides of an annotated guard tree into accessible, inaccessible and redundant right hand sides,
by using the function $\mathtt{can\_prove\_empty}$.

This is $\mathcal{R}$ as presented in LYG and chapter \ref{chap:backgrdRedundantInaccAnalysis}:
\[
	\begin{array}{lcl}
		\red(\antrhs{\Theta}{n})  & = & \begin{cases}
			(\epsilon, \epsilon, n), & \text{if $\generate(\Theta) = \emptyset$} \\
			(n, \epsilon, \epsilon), & \text{otherwise}                          \\
		\end{cases}                                                                                                                     \\
		\red(\antseq{t}{u})       & = & (\overline{k}\,\overline{k'}, \overline{n}\,\overline{n'}, \overline{m}\,\overline{m'}) \hspace{0.5em} \text{where} \begin{array}{l@{\,}c@{\,}l}
			(\overline{k}, \overline{n}, \overline{m})    & = & \red(t) \\
			(\overline{k'}, \overline{n'}, \overline{m'}) & = & \red(u) \\
		\end{array} \\
		\red(\antbang{\Theta}{t}) & = & \begin{cases}
			(\epsilon, m, \overline{m'}), & \text{if $\generate(\Theta) \not= \emptyset$ and $\red(t) = (\epsilon, \epsilon, m\,\overline{m'})$} \\
			\red(t),                      & \text{otherwise}                                                                                     \\
		\end{cases}                                                                                                                     \\
	\end{array}
\]

This definition has a surprisingly direct representation in Lean:
\begin{minted}{Lean}
def ρ : Ant Φ → list Rhs × list Rhs × list Rhs
| (Ant.rhs ty n) :=
    if can_prove_empty ty
    then ([], [], [n])
    else ([n], [], [])
| (Ant.branch tr1 tr2) :=
    match (ρ tr1, ρ tr2) with
    | ((k, n, m), (k', n', m')) := (k ++ k', n ++ n', m ++ m')
    end 
| (Ant.diverge ty tr) :=
    match ρ tr, can_prove_empty ty with
    | ([], [], m :: ms), ff := ([], [m], ms)
    | r, _ := r
    end
\end{minted}

\subsection{Interleaving $\mathcal{U}$ and $\mathcal{A}$}
\label{sec:formalizationInterleaving}

Since $\mathcal{A}\mathtt{\_acc}$ and $\mathcal{U}\mathtt{\_acc}$ have
the same recursion structure, they can be combined into a single function that shares the recursive invocations.
The following function $\mathcal{UA}\mathtt{\_acc}$ computes the uncovered refinement type and the annotated guard tree for a given guard tree at the same time.
This improves performance if a lazy evaluation strategy is used in combination with sharing as the accumulator can be fully shared.

\begin{minted}{Lean}
def ωπ_acc : (Φ → Φ) → Gdt → Φ × Ant Φ
| acc (Gdt.rhs rhs) := (Φ.false, Ant.rhs (acc Φ.true) rhs)
| acc (Gdt.branch tr1 tr2) :=
    let (U1, A1) := ωπ_acc acc tr1,
        (U2, A2) := ωπ_acc (U1.and ∘ acc) tr2
    in  (U2, Ant.branch A1 A2)
| acc (Gdt.grd (Grd.bang var) tr) :=
    let (U, A) := ωπ_acc (acc ∘ (Φ.var_is_not_bottom var).and) tr
    in (U, Ant.diverge (acc (Φ.var_is_bottom var)) A)
| acc (Gdt.grd (Grd.tgrd grd) tr) := 
    let (U, A) := ωπ_acc (acc ∘ (Φ.tgrd_in grd)) tr
    in ((acc (Φ.not_tgrd grd)).or U, A)
\end{minted}

It is surprisingly easy to show that this function is really interleaving $\mathcal{A}\mathtt{\_acc}$ and $\mathcal{U}\mathtt{\_acc}$:

\begin{minted}{Lean}
theorem ωπ_acc_eq (acc: Φ → Φ) (gdt: Gdt):
    ωπ_acc acc gdt = (ω_acc acc gdt, π_acc acc gdt) :=
by induction gdt generalizing acc;
    try { cases gdt_grd }; simp [ωπ_acc, ω_acc, π_acc, *]
\end{minted}


\section{Correctness Statements}
\label{sec:formalizationCorrectnessStmts}

As we have all the required definitions at this point, we can state and formalize what we expect of the presented pattern match analyses to be considered correct.
We provide proofs for all correctness propositions on GitHub \cite{leanProof}. Chapter \ref{sec:proof} will discuss parts of these proofs in more detail.

\subsection{Correctness of the $\mathcal{U}$ncovered Analysis}
\label{sec:formalizationSemanticU}

$\unc$ should compute a refinement type that denotes exactly all values that are not covered by a given guard tree.
This does not include values under which the execution diverges!

\begin{minted}{Lean}
theorem ω_semantic: ∀ gdt: Gdt, ∀ env: Env,
        (ω gdt).eval env ↔ (gdt.eval env = Result.no_match)
\end{minted}

As an obvious consequence, a guard tree always matches (or diverges) if and only if this refinement type is empty. If a correct function $\mathcal{G}$ or $\mathtt{can\_prove\_empty}$ proves emptiness of such a computed refinement type, there clearly are no uncovered cases. Otherwise, a warning of potential uncovered cases should be issued! Hence, this theorem implies correctness of the uncovered analysis:
The uncovered analysis should rather report a false positive than not detect an uncovered case.

Note that this theorem carries over to all semantically equivalent definitions of $\mathcal{U}$.

\subsection{Correctness of the $\mathcal{R}$edundant/Inaccessible Analysis}
\label{sec:formalizationSemanticRA}

For a given guard tree and a given correct function $\mathtt{can\_prove\_empty}$ (which corresponds to $\generate$ in LYG), $\red$
should compute a triple $(a, i, r)$ of accessible, inaccessible and redundant right hand sides.
Whenever the given guard tree evaluates to a RHS, this RHS must be accessible and neither inaccessible nor redundant.
RHSs that are redundant can be removed without changing the semantic of the guard tree.
This expresses correctness of the redundant and inaccessible analysis.

\begin{minted}{Lean}
theorem ρ_semantic:
    ∀ can_prove_empty: CorrectCanProveEmpty,
    ∀ gdt: Gdt, gdt.disjoint_rhss → (
        let ⟨ a, i, r ⟩ := ρ can_prove_empty.val (π gdt)
        in
                (∀ env: Env, ∀ rhs: Rhs,
                    gdt.eval env = Result.value rhs
                      → rhs ∈ a \ (i ++ r)
                )
            ∧
                Gdt.eval_option (gdt.remove_rhss r.to_finset)
                = gdt.eval

        : Prop
    )
\end{minted}

Note that redundant RHSs could be marked as inaccessible or even accessible instead without violating this theorem.
The opposite is not true: Not all accessible RHSs can be marked as inaccessible and not all inaccessible RHSs can be marked as redundant - see chapters \ref{sec:intro} and \ref{sec:background} for counterexamples. However, we conjecture that $a$ contains no inaccessible and $i$ no redundant RHSs if $\mathtt{can\_prove\_empty}$ is both correct and complete.
\chapter{Formalized Proofs}\label{sec:proof}

This chapter gives an overview of the formal proofs of the correctness statements from the previous chapter.
The full Lean proofs can be found on GitHub \cite{leanProof}.


To reduce the complexity of the definitions from chapter \ref{sec:formalization}, we came up with several internal definitions.
They include accumulator-free alternatives $\mathtt{U}$ and $\mathtt{A}$ for the functions $\mathcal{U}$ and $\mathcal{A}$.

%However, these alternatives might compute syntactically different refinement types that are only semantically equivalent to the source definition.
%Since $\mathtt{can\_prove\_empty}$ might yield different results for refinement types that describe the same set of values, care needs to be taken.

Correctness of $\mathtt{U}$ can be shown directly and this result can be transferred easily to $\mathcal{U}$ too, as $\mathcal{U}$'s correctness only depends on the semantic of the computed refinement type (see chapter \ref{sec:formalizationSemanticU}).
It is much more difficult to show correctness of $\red$/$\mathcal{A}$ though, so we will discuss this in more detail.

In chapter \ref{sec:proofRedRemovable}, we show that redundant RHSs can be removed without changing semantics.
Then, in chapter \ref{sec:proofAcc}, we show that if a guard tree evaluates to a RHS, this RHS must be marked as accessible. Together, these properties form the correctness statement of the uncovered/redundant analysis as presented in chapter \ref{sec:formalizationSemanticRA}.

In total, we declared 48 definitions and proved 143 lemmas and theorems, resulting in 2009 lines of Lean code!

\section{Simplification $A$ of $\mathcal{A}$}\label{chap:A_A}

It is difficult to reason about $\mathcal{A}\_acc$ and thus $\mathcal{A}$,
as we are only interested in certain well behaving accumulator values (in particular homomorphisms) and not arbitrary functions.
Let us have another look at the definition of $\mathcal{A}$:

\begin{minted}{Lean}
def π_acc : (Φ → Φ) → Gdt → Ant Φ
| acc (Gdt.rhs rhs) := Ant.rhs (acc Φ.true) rhs
| acc (Gdt.branch tr1 tr2) := Ant.branch
        (π_acc acc tr1)
        (π_acc ((ω_acc acc tr1).and ∘ acc) tr2)
| acc (Gdt.grd (Grd.bang var) tr) := Ant.diverge
        (acc (Φ.var_is_bottom var)) 
        (π_acc (acc ∘ ((Φ.var_is_not_bottom var).and)) tr)
| acc (Gdt.grd (Grd.tgrd grd) tr) :=
    (π_acc (acc ∘ (Φ.tgrd_in grd)) tr)

def π : Gdt → Ant Φ := π_acc id
\end{minted}

Since $\mathcal{A}$ is central to many propositions, we define a much simpler function $A$ that does not need an accumulator:

\begin{minted}{Lean}
def A : Gdt → Ant Φ
| (Gdt.rhs rhs) := Ant.rhs Φ.true rhs
| (Gdt.branch tr1 tr2) := Ant.branch (A tr1) $ (A tr2).map ((U tr1).and)
| (Gdt.grd (Grd.bang var) tr) := Ant.diverge (Φ.var_is_bottom var)
                    $ (A tr).map ((Φ.var_is_not_bottom var).and)
| (Gdt.grd (Grd.tgrd grd) tr) := (A tr).map (Φ.tgrd_in grd)
\end{minted}

However, $\mathcal{A}(\mathrm{gdt})$ is not syntactically equal to $A(\mathrm{gdt})$ for every $\mathrm{gdt}$,
as the following example shows:

\overfullrule=0pt

\begin{align*}
    gdt :=&\vcenter{
    	\begin{forest}
    		grdtree
			[{$\grdcon{\mathtt{True}}{x}$}
						[[3]]
						[[4]]
			]
    	\end{forest}
	}\\
	A(gdt) :=&\vcenter{
    	\begin{forest}
    		anttree
			[
						[$x \termeq \mathtt{True} \; \mathrm{in} \; \true$ [3]]
						[$x \termeq \mathtt{True} \; \mathrm{in} \; (\false \land \true)$ [4]]
			]
    	\end{forest}
	}\\
	\mathcal{A}(gdt) :=&\vcenter{
    	\begin{forest}
    		anttree
			[
						[$x \termeq \mathtt{True} \; \mathrm{in} \; \true$ [3]]
						[$x \termeq \mathtt{True} \; \mathrm{in} \; (\false) \land x \termeq \mathtt{True} \; \mathrm{in} \; (\true)$ [4]]
			]
    	\end{forest}
	}
\end{align*}

This counterexample can easily be verified by Lean:
\begin{minted}{Lean}
theorem A_neq_π (r: Rhs) (g: TGrd): A ≠ π :=
begin
    intro,
    replace a := congr_fun a (Gdt.grd (Grd.tgrd g)
        (
                    (Gdt.rhs r)
            .branch (Gdt.rhs r)
        )),
    finish [A, π, π_acc, Ant.map],
end
\end{minted}

We are unsure whether the definition of $A$ and $\mathcal{A}$ can be adapted to get syntactical equality
while maintaining the simplicity of $A$ and aligning the recursion structure of $\mathcal{A}$ and $\mathcal{U}$ (see chapter \ref{sec:formalizationInterleaving}).

Instead, we define a semantics on $\mathtt{Ant}\;\Phi$ and show that $A$ and $\mathcal{A}$ have the same semantics:
\begin{minted}{Lean}
def Ant.eval_rhss (ant: Ant Φ) (env: Env): Ant bool :=
    ant.map (λ ty, ty.eval env)
    
theorem A_sem_eq_π (gdt: Gdt):
    (A gdt).eval_rhss = (π gdt).eval_rhss
\end{minted}

When only relying on semantical equivalence, care has to be taken when getting insights on $\mathcal{A}$ by studying $A$,
as $\mathtt{can\_prove\_empty}$ does not have to be \textit{well defined}
on refinement types modulo semantical equivalence. If two refinement types are semantically equal,
$\mathtt{can\_prove\_empty}$ could be $\mathrm{true}$ for the former, but $\mathrm{false}$ for the latter type.
A function $\mathtt{can\_prove\_empty}$ that is correct and has this well defined property is uncomputable if it returns $\mathtt{true}$ for the refinement type $\false$ - it would need to return $\mathtt{true}$ for all refinement types that are empty!
Thus, $\mathtt{can\_prove\_empty}$ must operate on the refinement types of $\mathcal{A}$.

\section{Redundant RHSs Can Be Removed Without Changing Semantics}\label{sec:proofRedRemovable}

\subsection{Proof Idea}

Given a guard tree $gdt$ with disjoint RHSs
and an annotated guard tree $Agdt$ that semantically equals $\mathtt{A}\;\mathtt{gdt}$,
all redundant leaves reported by $\mathcal{R}$ (on $Agdt$, using a correct function $\mathtt{can\_prove\_empty}$) can be removed from $gdt$ without changing its semantics.
We will later instantiate $\mathtt{Agdt}$ with $\mathcal{A}\;\mathtt{gdt}$.
The indirection introduced by $\mathtt{Agdt}$ allows to use the simpler definition of $A$ while $\mathtt{can\_prove\_empty}$ still computes emptiness for refinement types in $\mathtt{Agdt}$ (see chapter \ref{chap:A_A} for why this is important).
This internal statement forms the second part of the correctness property defined in chapter \ref{sec:formalizationSemanticRA} and is formalized as follows:

\begin{minted}{Lean}
theorem R_red_removable
    (can_prove_empty: CorrectCanProveEmpty)
    { gdt: Gdt } (gdt_disjoint: gdt.disjoint_rhss)
    { Agdt: Ant Φ }
    (ant_def: Agdt.mark_inactive_rhss = (A gdt).mark_inactive_rhss):
        Gdt.eval_option (gdt.remove_rhss (
            (R (Agdt.map can_prove_empty.val)).red.to_finset
        ))
        = gdt.eval
\end{minted}

The general idea is to focus on a particular but arbitrary environment $\mathrm{env}$:
Reasoning about which RHSs can be removed while preserving semantics is much simpler when only considering a single environment.

In fact, we can just evaluate the given guard tree on $\mathrm{env}$ and safely remove all RHSs except the one the evaluation ended with.
We call RHSs that play no role in the evaluation on $\mathrm{env}$ \textit{inactive}, the resulting RHS is called \textit{active}.
If the evaluation diverged however, the diverging bang guard must not be removed; thus, all RHSs behind the diverging bang operator except one can be removed. In this case, the bang guard is \textit{active} and all RHSs are inactive. Clearly, at most one node (RHS or bang guard) is active.

The function $\mathtt{Gdt.mark\_inactive}$ directly computes a boolean annotated tree
that marks inactive nodes for a given guard tree and environment.
The definition of $\mathtt{Gdt.mark\_inactive}$ is very similar to the definition of the denotational semantic of guard trees - this helps proofs that bring these concepts together.
This function equals the negation of the semantic of trees annotated with refinement types!

It remains to relate the set of RHSs $r := \mathcal{R}(\mathcal{A}(\mathrm{gdt})).\mathrm{red}$
to the RHSs that can be removed when focusing on a particular environment.

Figure \ref{fig:proofOverview} sketches the proof idea.
Thin arrows mark the data flow, fat arrows the flow of reasoning.

\begin{figure}[htbp]
    \caption{Proof Overview: Redundant RHSs can be removed without changing semantics.}
    \label{fig:proofOverview}
	\centering
	\fontsize{8}{10}\selectfont
    \centerline{
        \includesvg[width = 490pt]{proof-overview}
	}
\end{figure}

\subsubsection{Step 1: Defining $\mathrm{gdt}$ and $\mathcal{A}(\mathrm{gdt})$}

We start with a guard tree $\mathrm{gdt}$ and its annotated tree $\mathcal{A}(\mathrm{gdt})$.

As a detail in the formal proof, we actually use $\mathtt{Agdt}$ instead of $\mathcal{A}(\mathrm{gdt})$, but since $\mathrm{ant_2} := \mathcal{A}(\mathrm{gdt}).\mathrm{map}(\neg \circ \Phi.\mathrm{eval}_{\mathrm{env}})$
only depends on the semantic of $\mathcal{A}(\mathrm{gdt})$ and $\mathrm{Agdt}$ has the same semantic, this does not change the proof idea.

% ($\mathrm{ant}_3$ in the figure)

\subsubsection{Step 2: Decomposing $\mathcal{R}$ into $R$ and $\mathrm{Ant.map}(\mathrm{can\_prove\_empty})$, Defining $\mathrm{ant}_1$}

To better understand $\mathcal{R}$, we decompose $\mathcal{R}$, which takes an $\mathrm{Ant}\;\Phi$ and needs a function $\mathtt{can\_prove\_empty}$,
into a function $R$ that takes an $\mathrm{Ant}\;\mathrm{bool}$
and a function $f :=  \mathrm{map}(\mathrm{can\_prove\_empty})$ that computes an
$\mathrm{Ant}\;\mathrm{bool}$ from an $\mathrm{Ant}\;\Phi$ so that $\mathcal{R} = R \circ f$.

In figure \ref{fig:proofOverview}, $\mathrm{ant}_1 := f(\mathcal{A}(\mathrm{gdt}))$ represents the object that $R$ works on.
Clearly, $\mathcal{R}(\mathcal{A}(\mathrm{gdt}))\mathrm{.red} = R(\mathrm{ant_1})\mathrm{.red}$.
In this particular example, only the refinement type associated with RHS 1 is recognized as empty and we have
$\mathcal{R}(\mathcal{A}(\mathrm{gdt}))\mathrm{.red} = \{ \mathtt{RHS\;1} \}$, as indicated by the ellipsis.


\subsubsection{Step 3: Defining $\mathrm{ant}_3$ and $\mathrm{ant}_2$}

$\mathrm{ant}_3$ in figure \ref{fig:proofOverview} is a boolean annotated tree whose nodes indicate inactivity under $\mathrm{env}$ (true if they are inactive, otherwise false).
It is much easier to reason about the effect of removing selected RHSs from this tree due to the closely related definitions of $\mathtt{Gdt.mark\_inactive}$ and $\mathtt{Gdt.eval}$, especially if the selection of RHSs is done by only looking at $\mathrm{ant}_3$.

Is easy to relate $\mathrm{ant_1}$ with $\mathrm{ant_3}$ if we define $\mathrm{ant_2} := \mathcal{A}(\mathrm{gdt}).\mathrm{map}(\neg \circ \Phi.\mathrm{eval}_{\mathrm{env}})$ as the negation of the evaluation of each refinement type under $\mathrm{env}$.

\subsubsection{Step 4: Relating $\mathrm{ant}_1$, $\mathrm{ant}_2$ and $\mathrm{ant}_3$}

We can show that each boolean annotation in $\mathrm{ant_1}$ implies (``$\Rightarrow$'') the corresponding boolean annotation in $\mathrm{ant_2}$ pointwise (P1):
If a refinement type is empty, it must not match any environment.

We can also show $\mathrm{ant}_2 = \mathrm{ant}_3$ (P3), since a node is active under $\mathrm{env}$ if and only if the corresponding refinement type matches $\mathrm{env}$.

\subsubsection{Step 5: Exploiting the Relationship}

It is easy to show that any subset of RHSs $R(\mathrm{ant_3}).\mathrm{red}$ can be removed from $gdt$ without changing its semantic on $\mathrm{env}$.
We have to show the same for $\mathrm{R}(\mathrm{ant_1}).\mathrm{red}$.
We hoped that $\mathrm{R}(\mathrm{ant_1}).\mathrm{red}$ would be a subset of $\mathrm{R}(\mathrm{ant_2}).\mathrm{red}$ (due to $\mathrm{ant_1} \Rightarrow \mathrm{ant_2}$) to complete the proof. However, this is not the case! See chapter \ref{chap:isRedundantSet} for a counterexample.

To repair the proof idea, we defined a predicate $\mathtt{is\_redundant\_set}$ (for brevity called $\mathtt{is\_redundant}$ in figure \ref{fig:proofOverview}) on sets of RHSs for a given boolean annotated tree.
This predicate has the property that $R(\mathrm{ant_1}).\mathrm{red}$ is a redundant set (P2, hence $\{ \mathtt{RHS\;1} \}$ is a redundant set) and that if $r$ is a redundant set in $\mathrm{ant_1}$ and if $\mathrm{ant_1} \Rightarrow \mathrm{ant_2}$,
then $r$ is also a redundant set in $\mathrm{ant_2}$ (P3).

Finally, we show that RHSs that are redundant in $\mathtt{gdt.mark\_inactive}_\mathtt{env}$ can be removed from guard trees without changing their semantic under $\mathtt{env}$ (P5).
This finishes the proof!

\subsection{Generalization of $R(\_).\mathrm{red}$}\label{chap:isRedundantSet}

Given two boolean annotated trees $\mathrm{ant_a}$ and $\mathrm{ant_b}$ with $\mathrm{ant_a} \Rightarrow \mathrm{ant_b}$,
we would like to transfer insights on redundant sets in $\mathrm{ant_a}$ to $\mathrm{ant_b}$ as stated in the previous chapter.

\subsubsection{$R$ is not suitable}

We cannot use $\mathrm{R}$ directly: $\mathrm{R}(\mathrm{ant_a}).\mathrm{red}$ does not need to be a subset of $\mathrm{R}(\mathrm{ant_b}).\mathrm{red}$! In fact, they can be disjoint, as the following counterexample shows.


\begin{align*}
	\mathrm{ant_a} :=&\vcenter{
    	\begin{forest}
    		anttree
			[
				[{false \lightning} [
				    [{true} [1]]
				    [{false} [2]]
				]]
			]
    	\end{forest}
	}\\
	\mathrm{ant_b} :=&\vcenter{
    	\begin{forest}
    		anttree
			[
				[{false \lightning} [
				    [{true} [1]]
				    [{true} [2]]
				]]
			]
    	\end{forest}
	}
\end{align*}

Clearly, it is $\mathrm{ant}_a \Rightarrow \mathrm{ant}_b$,
but $\mathrm{R}(\mathrm{ant_a}) = \{ 1 \}$ and $\mathrm{R}(\mathrm{ant_b}) = \{ 2 \}$.
This counterexample can easily be verified with Lean, a proof is included in \cite{leanProof}.

% We use Lean to construct and validate a counterexample. As we need to distinguish RHSs, we instantiate an exemplary guard module that uses $\mathbb{N}$ to identify them.
% It shows that there are boolean annotated guard trees $\mathrm{ant_a}$ and $\mathrm{ant_b}$ so that $\mathrm{ant_a}$ implies $\mathrm{ant_b}$ pointwise, but
% their redundant sets computed by $R$ are disjoint:
% 
% \begin{minted}{Lean}
% instance exmpl : GuardModule := {
%     Rhs := ℕ,
%     TGrd := ℕ,
%     Env := ℕ,
%     tgrd_eval := λ grd env, some env,
%     Var := ℕ,
%     is_bottom := λ var env, false
% }
% 
% def ant_a := Ant.diverge ff ((Ant.rhs tt 1).branch (Ant.rhs ff 2))
% def ant_b := Ant.diverge ff ((Ant.rhs tt 1).branch (Ant.rhs tt 2))
% 
% example : ant_a Γ ant_b
%     ∧ (R ant_a).red = [1] ∧ (R ant_b).red = [2] :=
% by finish [ant_a, ant_b, R,
%     Ant.implies.rhs, Ant.implies.branch, Ant.implies.diverge]
% \end{minted}

This issue is caused by the freedom of how \textit{critical sets} of RHSs can be avoided and that $R$ does not always consider this freedom.
A set of RHSs is critical if removing all its RHSs necessarily also removes a bang guard associated with a non-empty refinement type.
Clearly, a set of redundant right hand sides must not contain a critical set - otherwise, a possibly active bang guard might be removed!

Hence, if all RHSs behind a possibly active bang guard are inaccessible (as in $\mathrm{ant}_b$ in the counterexample), not all of them can be marked as redundant.
In such cases, $R$ marks all RHSs as redundant except the first.
However, $R$ could have excluded the second RHS instead, which would equally avoid the critical set caused by the bang guard!
If the second RHS is not inaccessible (as in $\mathrm{ant}_a$ in the counterexample), $R$ does not have to exclude a RHS and marks the first as redundant. In this case, $R$ makes use of the freedom of how critical sets can be avoided by using an active RHS instead of just the first RHS as witness.

\subsubsection{Definition of $\mathtt{is\_redundant\_set}$}

To overcome this issue, we generalize $R(\_).\mathrm{red}$ to a precidate $\mathtt{is\_redundant\_set}$ as follows:

\begin{minted}{Lean}
def Ant.critical_rhs_sets : Ant bool → finset (finset Rhs)
| (Ant.rhs inactive n) := ∅
| (Ant.diverge inactive tr) := tr.critical_rhs_sets ∪ if inactive
    then ∅
    else { tr.rhss }
| (Ant.branch tr1 tr2) := tr1.critical_rhs_sets ∪ tr2.critical_rhs_sets

def Ant.inactive_rhss : Ant bool → finset Rhs
| (Ant.rhs inactive n) := if inactive then { n } else ∅
| (Ant.diverge inactive tr) := tr.inactive_rhss
| (Ant.branch tr1 tr2) := tr1.inactive_rhss ∪ tr2.inactive_rhss

def Ant.is_redundant_set (a: Ant bool) (rhss: finset Rhs) :=
    rhss ∩ a.rhss ⊆ a.inactive_rhss
    ∧ ∀ c ∈ a.critical_rhs_sets, ∃ l ∈ c, l ∉ rhss
\end{minted}

A redundant set consists of RHSs that are annotated with $\mathtt{false}$ and avoid critical sets.
If a diverge node is annotated with $\mathtt{true}$, all its RHSs form a critical set.
Each critical set must have one RHS that is not contained in a given redundant set.
The purpose of critical sets is to ensure that active diverge nodes do not disappear when a redundant set is removed from a guard tree.

We show that all RHSs marked as redundant by $\mathcal{R}$ indeed form a redundant set: Clearly, $R$ avoids all critical sets and only marks inactive RHSs as redundant.

We believe that $\mathcal{R}(\_).\mathrm{red}$ actually computes a largest redundant set given a boolean annotated tree.
However, a largest redundant set does not need to be unique!
If $R$ would exclude the last inaccessible RHS instead of the first from being redundant, $R$ would compute a different redundant set of equal size.

It is also simple to show that the predicate becomes less strict the more nodes are marked as inactive, as the amount of critical sets decreases and the amount of inactive RHSs increases.

\subsection{Formal Proof}

The complete formal proof follows.
All used lemmas and their proofs can be found at \cite{leanProof}.
\begin{minted}{Lean}
theorem R_red_removable
    (can_prove_empty: CorrectCanProveEmpty)
    { gdt: Gdt } (gdt_disjoint: gdt.disjoint_rhss)
    { Agdt: Ant Φ }
    (ant_def: Agdt.mark_inactive_rhss = (A gdt).mark_inactive_rhss):
        Gdt.eval_option
            (gdt.remove_rhss 
                (R (Agdt.map can_prove_empty.val)).red.to_finset
            )
        = gdt.eval :=
begin

ext env:1,

-- `can_prove_empty` approximates emptiness for a
-- single refinement type.
-- `ant_empt` approximates emptiness of the
-- refinement types in `Agdt` for every `env`.
-- It also approximates inactive rhss of `gdt` in
-- context of `env` (ant_empt_imp_gdt).
let ant_empt := Agdt.map can_prove_empty.val,
have ant_empt_imp_gdt := calc
    ant_empt Γ Agdt.mark_inactive_rhss env
        : can_prove_empty_implies_inactive can_prove_empty Agdt env
    ...      Γ (A gdt).mark_inactive_rhss env
        : by simp [Ant.implies_refl, ant_def]
    ...      Γ gdt.mark_inactive_rhss env 
        : by simp [Ant.implies_refl, A_mark_inactive_rhss gdt env],

-- Since `gdt` has disjoint rhss, `ant_empt` has disjoint rhss too.
have ant_empt_disjoint : ant_empt.disjoint_rhss
    := by simp [Ant.disjoint_rhss_of_gdt_disjoint_rhss gdt_disjoint,
            Ant.disjoint_rhss_iff_of_mark_inactive_rhss_eq
                (function.funext_iff.1 ant_def env)],

-- The set of rhss `R_red` is redundant in `ant_empt` (red_in_ant_empt).
-- This means that these rhss are inactive and
-- not all rhss of possibly active diverge nodes are redundant.
let R_red := (R ant_empt).red.to_finset,
have red_in_ant_empt: ant_empt.is_redundant_set R_red
    := R_red_redundant ant_empt_disjoint,

-- Since `redundant_in` is monotone and `ant_empt`
-- approximates inactive rhss on `gdt`,
-- `R_red` is also redundant in `gdt` (red_in_gdt).
have red_in_gdt: (gdt.mark_inactive_rhss env).is_redundant_set R_red
    := is_redundant_set_monotone _ ant_empt_imp_gdt red_in_ant_empt,

-- Since `R_red` is a redundant set, it can be removed from `gdt` without
-- changing the semantics. Note that `R_red` is independent of env.
show Gdt.eval_option (Gdt.remove_rhss R_red gdt) env = gdt.eval env,
from redundant_rhss_removable gdt gdt_disjoint env _ red_in_gdt,

end
\end{minted}

\section{Accessible RHSs Must Be Detected as Accessible}\label{sec:proofAcc}

For the correctness of the inaccessible/redundant analysis,
it remains to show that the analysis correctly identifies all potentially accessible RHSs.

This is formalized by the following lemma:
\begin{minted}{Lean}
lemma R_acc_mem_of_reachable
    { gdt: Gdt } { env: Env } { rhs: Rhs } { ant: Ant Φ }
    (gdt_disjoint: gdt.disjoint_rhss)
    (can_prove_empty: CorrectCanProveEmpty)
    (Agdt: ant.mark_inactive_rhss env = (A gdt).mark_inactive_rhss env)
    (h: gdt.eval env = Result.value rhs)
    { r: RhsPartition }
    (r_def: r = R (ant.map can_prove_empty.val)):
    rhs ∈ r.acc \ (r.inacc ++ r.red) 
\end{minted}
As in chapter \ref{sec:proofRedRemovable}, $\mathtt{Agdt}$ abstracts from the syntactical structure of the refinement types in $\mathtt{A}\;gdt$.
Given that $\mathit{gdt}$ evaluates to $\mathit{rhs}$ under $\mathit{env}$,
we want to show that $R$ marks $\mathit{rhs}$ as accessible and not as inaccessible or redundant.

This proof is very technical, so we concentrate on the key insights.

First, we show that the accessible, inaccessible and redundant RHSs as identified by $R$ form a partition of all RHSs.
This is simple to prove and expressed by the following lemma ($a \sim b$ denotes that the list $a$ is a permutation of $b$):
\begin{minted}{Lean}
lemma R_rhss_perm { ant: Ant bool }:
    ((R ant).acc ++ (R ant).inacc ++ (R ant).red) ~ ant.rhss_list
\end{minted}

Clearly, $\mathit{rhs}$ is a RHS in $\mathit{gdt}$ and thus $\mathit{ant}$ and $\mathtt{ant.map\;can\_prove\_empty.val}$.
Since $R$ computes a partition of all RHSs, it remains to show that $\mathit{rhs}$ is neither contained in $\mathtt{r.inacc}$ nor in $\mathtt{r.red}$.

With the following lemma we only need to show that $\mathit{rhs}$ is not an inactive RHS in $\mathtt{ant.map\;can\_prove\_empty.val}$:
\begin{minted}{Lean}
lemma R_inacc_unon_R_red_subseteq_inactive (ant: Ant bool):
    (R ant).inacc.to_finset ∪ (R ant).red.to_finset
    ⊆ ant.inactive_rhss
\end{minted}

In fact, $\mathit{rhs}$ is the only active RHS in $\mathtt{gdt.mark\_inactive\_rhss\;env}$, as the following lemma shows:

\begin{minted}{Lean}
lemma gdt_mark_inactive_rhss_inactive_rhss_of_rhs_match
    { gdt: Gdt } { env: Env } { rhs: Rhs }
    (gdt_disjoint: gdt.disjoint_rhss):
    gdt.rhss \ (gdt.mark_inactive_rhss env).inactive_rhss = { rhs }
    ↔ gdt.eval env = Result.value rhs
\end{minted}

From chapter \ref{sec:proofRedRemovable}, we know that $(\mathtt{ant.map\;can\_prove\_empty.val})$ pointwise implies $(\mathtt{gdt.mark\_inactive\_rhss\;env})$:
empty refinement types imply inactivity.
When we proved that $\mathit{ant}_b$ is a redundant set if $\mathit{ant}_a$ is a redundant set and $\mathit{ant}_a \Rightarrow \mathit{ant}_b$ (as discussed in chapter \ref{chap:isRedundantSet}),
we first showed a stronger result that we can reuse now to relate inactive RHSs of $\mathtt{ant.map\;can\_prove\_empty.val}$ and $\mathtt{gdt.mark\_inactive\_rhss\;env}$:
\begin{minted}{Lean}
lemma is_redundant_set_monotone' { a b: Ant bool } (h: a Γ b): 
        a.inactive_rhss ⊆ b.inactive_rhss
        ∧ b.critical_rhs_sets ⊆ a.critical_rhs_sets
\end{minted}

We can use this fact and the previous lemmas to show that $\mathit{rhs}$ must be an active RHS in $\mathtt{ant.map\;can\_prove\_empty.val}$.
This closes the proof.
\chapter{Conclusion}\label{sec:conclusion}

We refined and formalized several correctness properties of LYG and successfully proved them in Lean.
However, we parametrized these correctness properties over a correct function $\mathcal{G}$ that semi-decides emptiness of refinement types.
While LYG defines such a function, we did not prove that it indeed is such a correct function $\mathcal{G}$.

Even though these correctness properties look seemingly easy to prove, it turned out to be a very involved undertaking.
After all, it took us 48 definitions and 143 lemmas and theorems to formalize these proofs in Lean!

We believe that this complexity is caused by the amount of details required to describe LYG and the rigorousness of Lean.
In fact, we discovered a minor flaw in LYG's definition of $\mathcal{U}$, buried in the details of the let binding semantics.
Luckily, this flaw has no impact on the GHC implementation of LYG, as the implementation uses a different encoding of refinement types.
Still, this flaw was not discovered in peer reviews of the LYG paper,
showing that LYG's correctness is not obvious at all and making a strong point for verification, yet formal verification.

Finally, as our proofs are formally verified by Lean, it is highly unlikely that LYG has any other flaws,
except in the definition of the presented function $\mathcal{G}$ that we did not check.

We can strongly recommend to use Lean for formal verification!

\bibliographystyle{ieeetr}
\bibliography{thesis/bib}

\begin{otherlanguage}{ngerman}
	\chapter*{Erklärung}
	\pagestyle{empty}

	\vspace{20mm}
	Hiermit erkläre ich, \theauthor, dass ich die vorliegende Masterarbeit selbst\-ständig
	verfasst habe und keine anderen als die angegebenen Quellen und Hilfsmittel
	benutzt habe, die wörtlich oder inhaltlich übernommenen Stellen als solche kenntlich gemacht und
	die Satzung des KIT zur Sicherung guter wissenschaftlicher Praxis beachtet habe.
	\vspace{20mm}
	\begin{tabbing}
		\rule{7cm}{.4pt}\hspace{1cm} \= \rule{6.8cm}{.4pt} \\
		Ort, Datum \> Unterschrift
	\end{tabbing}
\end{otherlanguage}

\chapter*{Danke}
\pagestyle{empty}

Ich danke meinen Betreuern Sebastian Graf und Sebastian Ullrich, die mich in jeglicher Hinsicht unterstützt haben.
Außerdem will ich mich bei der Lean Community bedanken, die mir bei Fragen zu Lean viel geholfen hat.

\pagestyle{fancy}

\end{document}
